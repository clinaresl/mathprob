%%
%% main.tex
%% 
%% Made by 
%% Login   <clinares@atlas>
%% 
%% Started on  Sat May  4 14:24:58 2019 
%% Last update Sat May  4 14:24:58 2019 
%%

\documentclass[svgnames,addpoints]{exam}

\usepackage[T1]{fontenc}
\usepackage[utf8]{inputenc}
\usepackage[spanish]{babel}

\usepackage{tb}

\usepackage{amsfonts}
\usepackage{amssymb}
\usepackage{mathtools}

\usepackage{pifont}

\usepackage{cancel}
\usepackage{array}

\usepackage{tikz}
\usepackage{pgflibraryarrows}
\usepackage{pgflibrarysnakes}

\usetikzlibrary{calc,matrix,patterns,fadings,positioning}

\usepackage{xcolor}

\usepackage{array}
\usepackage{eurosym}

\usepackage{booktabs}
\usepackage{url}

\usepackage{rotating}

\newlength{\zerowidth}
\settowidth{\zerowidth}{\huge 0}
\newlength{\zeroheight}
\settoheight{\zeroheight}{\huge 0}

\makeatletter
\def\convertto#1#2{\strip@pt\dimexpr #2*65536/\number\dimexpr 1#1}
\makeatother


\begin{document}

\titulacion{Esquemas}
\asignatura{Tablas de multiplicar}

\convocatoria{\today}
\tiempo{4 horas}

\begin{tabular}{cc}

  \begin{minipage}{7.5cm}

    \begin{tabular}{c}

      \includegraphics[scale=0.15]{tablas-multiplicar}\\

    \end{tabular}

  \end{minipage}
  &
  \begin{minipage}{8.5cm}

    \begin{center}

      {\huge \bf \titu}\\
      {\Large \bf \asig}\\ \ \\
      
      \convo

    \end{center}

  \end{minipage}\\ & \\ & \\  & \\

\end{tabular}

{\Large\bf Tablas de multiplicar}

Todas las representaciones que se muestran a continuación se hacen
empleando las siguientes medidas:

\begin{center}
  \begin{tabular}{ll}
    \texttt{zeroheight}   & \convertto{cm}{\the\zeroheight}\ cm \\
    \texttt{zerowidth}    & \convertto{cm}{\the\zerowidth}\ cm \\
  \end{tabular}
\end{center}

\noindent
que son el alto y ancho del {\huge 0} (tamaño \texttt{$\backslash$huge}), y que
representan genéricamente el alto y ancho de cualquier carácter; y el
\texttt{baselineskip} (\convertto{cm}{\the\baselineskip}\ cm), que representa el
espacio natural entre líneas. Las dos primeras medidas mostradas arriba deben
definirse en el preámbulo del documento \LaTeX\ (por ejemplo, con un fichero
\texttt{.sty} que se incluya automáticamente en el fichero).

Hay dos tipos de operaciones diferentes: \textit{type 0} y \textit{type 1}. En
el primer caso, la caja de respuesta se muestra en la posición del resultado de
cada producto; en el segundo caso, la caja de respuesta se muestra en la
posición del segundo operando. Las tablas de multiplicar consisten en un número
cualquiera de multiplicaciones (no necesariamente 10) entre dos límites dados
por el usuario.

Las tablas de multiplicar se caracterizan con los siguientes parámetros:

\begin{itemize}

  \item Número maximo de dígitos de la primera columna: 1

  \item Número maximo de dígitos de la segunda columna: 1

  \item Número maximo de dígitos necesarios para representar cada resultado: 1

  \item Número de multiplicaciones a presentar en la tabla: 3

\end{itemize}

El siguiente esquema muestra, para los valores de los parámetros dados
anteriormente, un ejemplo donde la caja de respuesta se ha dispuesto en cada
posición posible:

\noindent\begin{minipage}{0.33\linewidth}
  \begin{center}
    \begin{tikzpicture}

      % ----------------------------------------------------------------------------------

      \coordinate (bottom) at (0,0);
      \fill [blue] (bottom) circle (1pt);

      % el desplazamiento vertical desde la esquina inferior izquierda hasta
      % donde se escribe el primer operando de la última serie incluye una
      % distancia adicional de \baselineskip para facilitar que unas tablas se
      % puedan apilar sobre otras. El desplazamiento horizontal es igual al
      % 150% del ancho de un dígito para dejar un poco de espacio a derecha e
      % izquierda en el caso de que ahí haya que escribir una respuesta
      \coordinate (op11) at ($(bottom) + (1.5\zerowidth, 0.5\zeroheight+1.0\baselineskip)$);
      \fill [red] (op11) circle (1pt);
      \draw (op11) node {\huge 3};

      % El operador de multiplicación se situa dejando 2.5 veces el ancho de un
      % dígito porque: uno, el primer operador (op11) ocupa (1+2)=3 dígitos y
      % como el marcador está en el medio, debe desplazarse ahora 3/2=1.5 veces
      % el ancho de un dígito; segundo, el operador de multiplicación ocupa
      % exactamente una posición. Por lo tanto, 1.5+1=2.5
      \coordinate (operator1) at ($(op11) + (2.5*\zerowidth, 0.0)$);
      \fill [red] (operator1) circle (1pt);
      \draw (operator1) node {\huge$\times$};

      % El segundo operando se situa asimismo a 2.5 veces el ancho de un dígito
      % por los mismos motivos de antes puesto que: uno, el operador ocupa un
      % dígito; dos, el operando ocupa (1+2)=3 dígitos
      \coordinate (op12) at ($(operator1) + (2.5*\zerowidth, 0.0)$);
      \fill [red] (op12) circle (1pt);
      \draw (op12) node {\huge 3};

      % el signo de igualdad sigue las mismas reglas, y vuelve a desplazarse 2.5
      % veces el ancho de un dígito exactamente por los mismos motivos que lo
      % hizo el operador de multiplicación
      \coordinate (equal1) at ($(op12) + (2.5*\zerowidth, 0.0)$);
      \fill [red] (equal1) circle (1pt);
      \draw (equal1) node {\huge$=$};

      % Por fin, el resultado de la multiplicacion se escribe desplazándose
      % como antes 2.5 veces el ancho de un dígito respecto de la posición del
      % último carácter (el símbolo de igual), porque el signo de igual ocupa
      % exactamente un dígito y porque la caja de respuesta ocupa (1+2)=3 veces
      % el ancho de un dígito
      \coordinate (answer1) at ($(equal1) + (2.5*\zerowidth, 0.0)$);
      \fill [red] (answer1) circle (1pt);
      \draw(answer1) node [rounded corners, rectangle, minimum width=3.0*\zerowidth, minimum height = \zeroheight + \baselineskip, draw] {\textcolor{lightgray}{\huge 9}};

      % ----------------------------------------------------------------------------------

      % A continuación se muestra, en la segunda fila el caso en el que la
      % caja de respuesta está en el segundo operando
      \coordinate (op21) at ($(bottom) + (1.5\zerowidth, 1.5\zeroheight+2.5\baselineskip)$);
      \fill [red] (op21) circle (1pt);
      \draw (op21) node {\huge 3};

      \coordinate (operator2) at ($(op21) + (2.5*\zerowidth, 0.0)$);
      \fill [red] (operator2) circle (1pt);
      \draw (operator2) node {\huge$\times$};

      \coordinate (op22) at ($(operator2) + (2.5*\zerowidth, 0.0)$);
      \fill [red] (op22) circle (1pt);
      \draw(op22) node [rounded corners, rectangle, minimum width=3.0*\zerowidth, minimum height = \zeroheight + \baselineskip, draw] {\textcolor{lightgray}{\huge 2}};

      \coordinate (equal2) at ($(op22) + (2.5*\zerowidth, 0.0)$);
      \fill [red] (equal2) circle (1pt);
      \draw (equal2) node {\huge$=$};

      \coordinate (answer2) at ($(equal2) + (2.5*\zerowidth, 0.0)$);
      \fill [red] (answer2) circle (1pt);
      \draw (answer2) node {\huge 6};

      % ---------------------------------------------------------------------

      % La tercera fila desde abajo (y, por lo tanto, la primera fila de la
      % tabla de multiplicar) muestra el caso en el que la caja que debe
      % rellenarse está localizada en la posición del primer operador
      \coordinate (op31) at ($(bottom) + (1.5\zerowidth, 2.5\zeroheight+4.0\baselineskip)$);
      \fill [red] (op31) circle (1pt);
      \draw(op31) node [rounded corners, rectangle, minimum width=3.0*\zerowidth, minimum height = \zeroheight + \baselineskip, draw] {\textcolor{lightgray}{\huge 3}};

      \coordinate (operator3) at ($(op31) + (2.5*\zerowidth, 0.0)$);
      \fill [red] (operator3) circle (1pt);
      \draw (operator3) node {\huge$\times$};

      \coordinate (op32) at ($(operator3) + (2.5*\zerowidth, 0.0)$);
      \fill [red] (op32) circle (1pt);
      \draw (op32) node {\huge 1};

      \coordinate (equal3) at ($(op32) + (2.5*\zerowidth, 0.0)$);
      \fill [red] (equal3) circle (1pt);
      \draw (equal3) node {\huge$=$};

      \coordinate (answer3) at ($(equal3) + (2.5*\zerowidth, 0.0)$);
      \fill [red] (answer3) circle (1pt);
      \draw (answer3) node {\huge 3};

      % ----------------------------------------------------------------------------------

      % El bounding box reserva un poco de espacio por encima de la primera
      % fila de la tabla de multiplicar para facilitar el apilamiento de
      % tablas. El ancho es exactamente igual al requerido para dibujar todos
      % los elementos, y que incluye un
      % 150% del ancho de un dígito respecto del elemento (en cualquier fila) dibujado
      % más a la derecha
      \coordinate (right) at ($(answer3) + (1.5*\zerowidth, 0.5\zeroheight+1.0\baselineskip)$);
      \fill [blue] (right) circle (1pt);
      \draw [lightgray] (bottom) rectangle (right);

    \end{tikzpicture}
  \end{center}
\end{minipage}


A continuación se muestra otro ejemplo donde cambia el número de dígitos
necesarios por columna:

\begin{itemize}

  \item Número maximo de dígitos de la primera columna: 2

  \item Número maximo de dígitos de la segunda columna: 1

  \item Número maximo de dígitos necesarios para representar cada resultado: 3

  \item Número de multiplicaciones a presentar en la tabla: 3

\end{itemize}

\noindent\begin{minipage}{0.33\linewidth}
  \begin{center}
    \begin{tikzpicture}

      % --- Coordinates -----------------------------------------------------

      \coordinate (bottom) at (0,0);
      \fill [blue] (bottom) circle (1pt);

      \coordinate (op11) at ($(bottom) + (2.0*\zerowidth, 0.5\zeroheight+1.0\baselineskip)$);
      \fill [red] (op11) circle (1pt);
      \draw (op11) node {\huge 36};

      \coordinate (operator1) at ($(op11) + (3.0*\zerowidth, 0.0)$);
      \fill [red] (operator1) circle (1pt);
      \draw (operator1) node {\huge$\times$};

      \coordinate (op12) at ($(operator1) + (2.5*\zerowidth, 0.0)$);
      \fill [red] (op12) circle (1pt);
      \draw (op12) node {\huge 3};

      \coordinate (equal1) at ($(op12) + (2.5*\zerowidth, 0.0)$);
      \fill [red] (equal1) circle (1pt);
      \draw (equal1) node {\huge$=$};

      \coordinate (answer1) at ($(equal1) + (3.5*\zerowidth, 0.0)$);
      \fill [red] (answer1) circle (1pt);
      \draw(answer1) node [rounded corners, rectangle, minimum width=5.0*\zerowidth, minimum height = \zeroheight + \baselineskip, draw] {\textcolor{lightgray}{\huge 108}};

      % ----------------------------------------------------------------------------------

      % A continuación se muestra, en la segunda fila el caso en el que la
      % caja de respuesta está en el segundo operando
      \coordinate (op21) at ($(bottom) + (2.0\zerowidth, 1.5\zeroheight+2.5\baselineskip)$);
      \fill [red] (op21) circle (1pt);
      \draw (op21) node {\huge 36};

      \coordinate (operator2) at ($(op21) + (3.0*\zerowidth, 0.0)$);
      \fill [red] (operator2) circle (1pt);
      \draw (operator2) node {\huge$\times$};

      \coordinate (op22) at ($(operator2) + (2.5*\zerowidth, 0.0)$);
      \fill [red] (op22) circle (1pt);
      \draw(op22) node [rounded corners, rectangle, minimum width=3.0*\zerowidth, minimum height = \zeroheight + \baselineskip, draw] {\textcolor{lightgray}{\huge 2}};

      \coordinate (equal2) at ($(op22) + (2.5*\zerowidth, 0.0)$);
      \fill [red] (equal2) circle (1pt);
      \draw (equal2) node {\huge$=$};

      \coordinate (answer2) at ($(equal2) + (3.5*\zerowidth, 0.0)$);
      \fill [red] (answer2) circle (1pt);
      \draw (answer2) node {\huge 72};

      % ---------------------------------------------------------------------

      % La tercera fila desde abajo (y, por lo tanto, la primera fila de la
      % tabla de multiplicar) muestra el caso en el que la caja que debe
      % rellenarse está localizada en la posición del primer operador
      \coordinate (op31) at ($(bottom) + (2.0\zerowidth, 2.5\zeroheight+4.0\baselineskip)$);
      \fill [red] (op31) circle (1pt);
      \draw(op31) node [rounded corners, rectangle, minimum width=4.0*\zerowidth, minimum height = \zeroheight + \baselineskip, draw] {\textcolor{lightgray}{\huge 36}};

      \coordinate (operator3) at ($(op31) + (3.0*\zerowidth, 0.0)$);
      \fill [red] (operator3) circle (1pt);
      \draw (operator3) node {\huge$\times$};

      \coordinate (op32) at ($(operator3) + (2.5*\zerowidth, 0.0)$);
      \fill [red] (op32) circle (1pt);
      \draw (op32) node {\huge 1};

      \coordinate (equal3) at ($(op32) + (2.5*\zerowidth, 0.0)$);
      \fill [red] (equal3) circle (1pt);
      \draw (equal3) node {\huge$=$};

      \coordinate (answer3) at ($(equal3) + (3.5*\zerowidth, 0.0)$);
      \fill [red] (answer3) circle (1pt);
      \draw (answer3) node {\huge 36};

      % ----------------------------------------------------------------------------------

      % El bounding box reserva un poco de espacio por encima de la primera
      % fila de la tabla de multiplicar para facilitar el apilamiento de
      % tablas. El ancho es exactamente igual al requerido para dibujar todos
      % los elementos, y que incluye un
      % 150% del ancho de un dígito respecto del elemento (en cualquier fila) dibujado
      % más a la derecha
      \coordinate (right) at ($(answer3) + (2.5*\zerowidth, 0.5\zeroheight+1.0\baselineskip)$);
      \fill [blue] (right) circle (1pt);
      \draw [lightgray] (bottom) rectangle (right);

    \end{tikzpicture}
  \end{center}
\end{minipage}

Todos los elementos de una tabla emplean como punto de referencia, el elemento
anterior, salvo el primero que usa la esquina inferior izquierda,
\texttt{bottom}, independientemente de la fila en la que se encuentre. La
primera fila que se escribe es la última de la tabla (esto es, la que tiene las
coordenadas menores de $y$), y las siguientes (que son las anteriores en la
tabla de multiplicar) dejan un espacio con la línea siguiente igual a la altura
de un dígito y un 150\% del \texttt{baselineskip}, donde el 50\% se añade para
evitar que las cajas se toquen.

La primera fila (la última de la tabla de multiplicar) deja un pequeño espacio
extra que se utiliza para poder apilar correctamente diferentes elementos
\LaTeX{} en vertical. Este espacio vertical, como en el caso de otros elementos
de \texttt{mathprob}, es igual a la mitad de la altura de un dígito más el
\texttt{baselineskip}.

Para posicionar unos elementos horizontalmente junto a otros basta con
determinar el ancho de cada elemento: los operandos y el resultado tienen un
ancho que es exactamente igual al número de dígitos (tomando el máximo del
número de dígitos necesario en la misma columna para todas las filas) más otros
dos, uno que se deja a la derecha, y otro a la izquierda; el operador de
multiplicar, y el signo de igualdad, por su parte, tienen un ancho exactamente
igual a un dígito.

La esquina superior derecha se calcula sumando a la posición de la respuesta de
la primera fila (o la que tiene los mayores valores de $y$) una distancia igual
a la que se deja entre la esquina inferior izquierda y la última fila de la
tabla de multiplicar, la mitad de la altura de un dígito más el
\texttt{baselineskip}.

La siguiente tabla muestra la forma de calcular las coordenadas $x$ e $y$ de
cada uno de los puntos de referencia de la fila $i$-ésimia (donde la fila 0 se
corresponde con la última fila de la tabla de multiplicar, la fila 1 con la
penúltima, y así sucesivamente ) junto con una descripción de su utilidad:

\begin{center}
  \begin{tabular}{c|p{2.0cm}|c|c|c}
    Etiqueta & Descripción & Referencia & $\delta x$ & $\delta y$ \\ \toprule
    \texttt{bottom} & Esquina inferior izquierda & -- & 0 & 0 \\ \midrule
    \texttt{op}$i$\texttt{1} & Posición del primer operando & \texttt{bottom}&  $\left(\frac{2+\textrm{nbdigits}_{1}}{2}\right)\backslash\mathit{width}$ &
                                       $(i-\frac{1}{2})\backslash\mathit{height} + \frac{1}{2}(3i - 1)\backslash lineskip$ \\  \midrule
    \texttt{operator}$i$ & Posición del operador & \texttt{op}$i$\texttt{1} & $\left(1+\frac{2+\textrm{nbdigits}_{1}}{2}\right)\backslash\mathit{width}$ & 0 \\  \midrule
    \texttt{op}$i$\texttt{2} & Posición del segundo operando & \texttt{operator}$i$ & $\left(1+\frac{2+\textrm{nbdigits}_{3}}{2}\right)\backslash\mathit{width}$ & 0 \\  \midrule
    \texttt{equal}$i$ & Posición del igual & \texttt{op}$i$\texttt{2} & $\left(1+\frac{2+\textrm{nbdigits}_{3}}{2}\right)\backslash\mathit{width}$ & 0 \\  \midrule
    \texttt{answer}$i$ & Posición del resultado & \texttt{equal}$i$ & $\left(1+\frac{2+\textrm{nbdigits}_{5}}{2}\right)\backslash\mathit{width}$ & 0 \\  \midrule
    \texttt{right} & Esquina superior derecha & \texttt{answer}$n$ & $\left(\frac{2+\textrm{nbdigits}_{5}}{2}\right)\backslash\mathit{width}$ & $0.5\backslash\mathit{height}+1.0\backslash lineskip$ \\ \bottomrule
  \end{tabular}
\end{center}

\noindent
donde los valores del ancho y alto de un carácter o de una línea se han
abreviado por comodidad como $\backslash\mathit{width}$ y
$\backslash\mathit{height}$ respectivamente; $\mathit{nbdigits}_{i}$ representa
el número máximo de dígitos usados en la columna $i$-ésima (contando a partir de
1), y $n$ es el número de filas de la tabla de multiplicar.



\end{document}

%%% Local Variables:
%%% mode: latex
%%% TeX-master: t
%%% End:
