%%
%% main.tex
%% 
%% Made by 
%% Login   <clinares@atlas>
%% 
%% Started on  Sat May  4 14:24:58 2019 
%% Last update Sat May  4 14:24:58 2019 
%%

\documentclass[svgnames,addpoints]{exam}

\usepackage[T1]{fontenc}
\usepackage[utf8]{inputenc}
\usepackage[spanish]{babel}

\usepackage{tb}

\usepackage{amsfonts}
\usepackage{amssymb}
\usepackage{mathtools}

\usepackage{pifont}

\usepackage{cancel}
\usepackage{array}

\usepackage{tikz}
\usepackage{pgflibraryarrows}
\usepackage{pgflibrarysnakes}

\usetikzlibrary{calc,matrix,patterns,fadings,positioning}

\usepackage{xcolor}

\usepackage{array}
\usepackage{eurosym}

\usepackage{booktabs}
\usepackage{url}

\usepackage{rotating}

\newlength{\zerowidth}
\settowidth{\zerowidth}{\huge 0}
\newlength{\zeroheight}
\settoheight{\zeroheight}{\huge 0}

\makeatletter
\def\convertto#1#2{\strip@pt\dimexpr #2*65536/\number\dimexpr 1#1}
\makeatother


\begin{document}

\titulacion{Esquemas}
\asignatura{Divisiones}

\convocatoria{\today}
\tiempo{4 horas}

\begin{tabular}{cc}

  \begin{minipage}{7.5cm}

    \begin{tabular}{c}

      \includegraphics[scale=0.6]{images/math-calculator}\\

    \end{tabular}

  \end{minipage}
  &
  \begin{minipage}{8.5cm}

    \begin{center}

      {\huge \bf \titu}\\
      {\Large \bf \asig}\\ \ \\
      
      \convo

    \end{center}

  \end{minipage}\\ & \\ & \\  & \\

\end{tabular}

{\Large\bf Divisiones}

Todas las representaciones que se muestran a continuación se hacen
empleando las siguientes medidas:

\begin{center}
  \begin{tabular}{ll}
    \texttt{zeroheight}   & \convertto{cm}{\the\zeroheight}\ cm \\
    \texttt{zerowidth}    & \convertto{cm}{\the\zerowidth}\ cm \\
  \end{tabular}
\end{center}

\noindent
y el \texttt{baselineskip} (\convertto{cm}{\the\baselineskip}\
cm). Las dos mostradas arriba deben definirse en el preámbulo del
documento \LaTeX\ (por ejemplo, con un fichero \texttt{.sty} que se
incluya automáticamente en el fichero). Además, cada operación se
muestra en una \textit{minipage} con un ancho igual a un cuarto el
ancho de la página, salvo el último (con dividendos más largos) que
requieren que cada una se haga en un tercio de la longitud de la
línea.

\begin{questions}

  \question
  Primero, se muestran varias divisiones con las siguientes características:

  \begin{itemize}
    
  \item Número de dígitos del dividendo: 1
    
  \item Número de dígitos del divisor: 1
    
  \item Número de dígitos necesarios para representar el cociente: 1
    
  \end{itemize}

  El esquema general se muestra a continuación:

  \noindent\begin{minipage}{0.25\linewidth}
    \begin{center}
      \begin{tikzpicture}

        % in the following, let:
        %
        % nbdvdigits: number of digits required for displaying the
        % dividend
        %
        % nbdrdigits: number of digits required for displaying the
        % divisor
        %
        % nbqdigits: number of digits necessary for writing the
        % quotient
        % 
        % nbrows: number of rows required for making the
        % calculations. This number is exactly equal to twice the
        % number of digits of the quotient, as each digit in the
        % quotient requires two lines for making calculations.

        % parameters:
        %
        % \zerowidth: width of a digit
        % \zeroheight: height of a digit
        %
        % Also, \baselineskip is used to take into account the natural
        % space between lines
        
        % --- Coordinates -------------------------------------------------------
        
        % The first label gets at x=0 and it marks where the dividend
        % should be written.
        %
        % Let Y denote the maximum vertical space required to draw the
        % whole picture. Y is computed as the total number of rows of
        % the figure plus 0.5 to give some slack. The total number of
        % rows required is nbrows + 1 (+1 for displaying the dividend)
        \coordinate (label1) at (0, 3.5);
        \fill [red] (label1) circle (1pt);

        % the location of the divisor is computed by adding the space
        % of *two* digits to nbdvdigits to leave some space between
        % the dividend and the divisor
        \coordinate (label2) at ($(label1) + 3.0*(\zerowidth, 0.0)$);
        \fill [red] (label2) circle (1pt);

        % The third coordinate defines the location which splits the
        % divisor and the quotient. Let W denote the width of the box
        % that encloses the divisor, W = 2 + max (nbdrdigts, nbqdigits)
        %
        % W is computed taking into account the number of digits
        % required for writing both the dividend and the quotient,
        % which is max (nbdrdigits, nbqdigits), plus 2 digits to leave
        % some space around. Then the third label is computed as the
        % midpoint.
        %
        % Also, this label is lowered the height of a digit. 
        \coordinate (label3) at ($(label2) + (1.5*\zerowidth, -\zeroheight)$);
        \fill [red] (label3) circle (1pt);
        % -----------------------------------------------------------------------

        % --- Ancilliary reference points

        % the first line below the dividend is shown next. It is
        % aligned with the left margin of the whole bounding box, so
        % that the number 3 substracts the same horizontal distance
        % added when computing label2. Twice the height of a digit
        % plus 0.15 cm is substracted *always* so that the first line
        % gets properly aligned with the quotient
        \coordinate (line1) at ($(label2) + (-3.0\zerowidth, -2*\zeroheight-0.15 cm)$);
        \fill [blue] (line1) circle (1pt);

        % the second line below the dividend is shown next. All these
        % lines are computed just by simply leaving a vertical space
        % equal to the sum of the height of a digit and the
        % baselineskip
        \coordinate (line2) at ($(line1) + 1*(0.0, -\zeroheight-\baselineskip)$);
        \fill [blue] (line2) circle (1pt);
        % -----------------------------------------------------------------------

        % --- Bounding Box ------------------------------------------------------

        % the bottom of the whole bounding box can be computed from
        % the first line just by decrementing the height as many lines
        % as required for making all the calculations (minus one,
        % because we start in the first line) minus half the
        % height of a digit for the last row.
        %
        % The number of rows required for making all the calculations
        % is twice the number of digits in the quotient, but one has
        % to be substracted because we start from the first line, so
        % that the factor used next is 2*nbqdigits - 1. Note that the
        % same factor is used to divide the last part of the
        % computation of the y-shift so that it is always equal to
        % half the height of a digit
        \coordinate (bottom) at ($(line1) + 1.0*(0.0, -\zeroheight-\baselineskip-0.5/1.0*\zeroheight)$);
        \fill [green] (bottom) circle (1pt);

        % the right border is computed just by adding to the second
        % label as many digits as required to draw both the quotient
        % and the divisor plus two ---+2 to leave some space
        % around. This number was also used above to compute the
        % location of label3 and it is equal to:
        % 2 + max (nbdrdigits, nbqdigits)
        \coordinate (right) at ($(label2) + (3*\zerowidth, \zeroheight)$);
        \fill [green] (right) circle (1pt);
        
        % draw an invisible box used to properly align all sequences
        % \draw [lightgray] (0,0) rectangle (3.5,5.0);
        \draw [lightgray] (bottom) rectangle (right);
        
        % -----------------------------------------------------------------------

        % show the box enclosing the divisor

        % the width of the box has a width equal to 2 + max
        % (nbdrdigits, nbqdigits), and it has a height equal to twice
        % the height of a digit. Note that label2 falls in the middle
        % of the vertical bar and label3 falls in the middle of the
        % horizontal bar
        \draw [thick, rounded corners] ($(label2) + (0.0, \zeroheight)$) -- ($(label2) + (0.0, -\zeroheight)$) -- ($(label2) + 3.0*(\zerowidth, -\zeroheight/3.0)$);

        % show the box for writing the quotient

        % the width of the box equals 2 + max (nbdrdigits, nbqdigits),
        % and its height equals the sum of the height of a digit plus
        % the baselineskip. It is drawn 0.15 cms from the line
        % separating the divisor from the quotient
        \node [rounded corners, rectangle, minimum width=3.0*\zerowidth, minimum height = \zeroheight+\baselineskip, draw, below=0.15 cm of label3] {\textcolor{lightgray}{\huge 4}};

        
        % --- Text

        % Dividend
        \node [right=0.0 cm of label1] (dividend) {\huge 9};

        % Divisor
        \node [right=0.0 cm of label2] (divisor) {\huge 2};

        % calculations
        \node [lightgray,right=0.0cm of line1] (result1) {\huge 8};
        \node [lightgray,right=0.0cm of line2] (result2) {\huge 1};
        
      \end{tikzpicture}
    \end{center}
  \end{minipage}

  A continuación se muestra una vista de la composición de diferentes
  divisiones de este tipo:
  
  \noindent\begin{minipage}{0.25\linewidth}
    \begin{center}
      \begin{tikzpicture}

        % --- Coordinates -------------------------------------------------------
        \coordinate (label1) at (0, 3.5);
        \coordinate (label2) at ($(label1) + 3.0*(\zerowidth, 0.0)$);
        \coordinate (label3) at ($(label2) + (1.5*\zerowidth, -\zeroheight)$);
        % -----------------------------------------------------------------------

        % --- Ancilliary reference points
        \coordinate (line1) at ($(label2) + (-3.0\zerowidth, -2*\zeroheight-0.15 cm)$);
        % -----------------------------------------------------------------------

        % --- Bounding Box ------------------------------------------------------
        \coordinate (bottom) at ($(line1) + 1.0*(0.0, -\zeroheight-\baselineskip-0.5/1.0*\zeroheight)$);
        \coordinate (right) at ($(label2) + (3*\zerowidth, \zeroheight)$);
        \draw [white] (bottom) rectangle (right);
        % -----------------------------------------------------------------------

        % show the box enclosing the divisor
        \draw [thick, rounded corners] ($(label2) + (0.0, \zeroheight)$) -- ($(label2) + (0.0, -\zeroheight)$) -- ($(label2) + 3.0*(\zerowidth, -\zeroheight/3.0)$);

        % show the box for writing the quotient
        \node [rounded corners, rectangle, minimum width=3.0*\zerowidth, minimum height = \zeroheight+\baselineskip, draw, below=0.15 cm of label3] {};
        % -----------------------------------------------------------------------
        
        % --- Text ------------------------------------------------------------

        % Dividend
        \node [right=0.0 cm of label1] (dividend) {\huge 9};

        % Divisor
        \node [right=0.0 cm of label2] (divisor) {\huge 2};
        % -----------------------------------------------------------------------
        
      \end{tikzpicture}
    \end{center}
  \end{minipage}
  \noindent\begin{minipage}{0.25\linewidth}
    \begin{center}
      \begin{tikzpicture}

        % --- Coordinates -------------------------------------------------------
        \coordinate (label1) at (0, 3.5);
        \coordinate (label2) at ($(label1) + 3.0*(\zerowidth, 0.0)$);
        \coordinate (label3) at ($(label2) + (1.5*\zerowidth, -\zeroheight)$);
        % -----------------------------------------------------------------------

        % --- Ancilliary reference points
        \coordinate (line1) at ($(label2) + (-3.0\zerowidth, -2*\zeroheight-0.15 cm)$);
        % -----------------------------------------------------------------------

        % --- Bounding Box ------------------------------------------------------
        \coordinate (bottom) at ($(line1) + 1.0*(0.0, -\zeroheight-\baselineskip-0.5/1.0*\zeroheight)$);
        \coordinate (right) at ($(label2) + (3*\zerowidth, \zeroheight)$);
        \draw [white] (bottom) rectangle (right);
        % -----------------------------------------------------------------------

        % show the box enclosing the divisor
        \draw [thick, rounded corners] ($(label2) + (0.0, \zeroheight)$) -- ($(label2) + (0.0, -\zeroheight)$) -- ($(label2) + 3.0*(\zerowidth, -\zeroheight/3.0)$);

        % show the box for writing the quotient
        \node [rounded corners, rectangle, minimum width=3.0*\zerowidth, minimum height = \zeroheight+\baselineskip, draw, below=0.15 cm of label3] {};
        % -----------------------------------------------------------------------
        
        % --- Text ------------------------------------------------------------

        % Dividend
        \node [right=0.0 cm of label1] (dividend) {\huge 8};

        % Divisor
        \node [right=0.0 cm of label2] (divisor) {\huge 3};
        % -----------------------------------------------------------------------
        
      \end{tikzpicture}
    \end{center}
  \end{minipage}
  \noindent\begin{minipage}{0.25\linewidth}
    \begin{center}
      \begin{tikzpicture}

        % --- Coordinates -------------------------------------------------------
        \coordinate (label1) at (0, 3.5);
        \coordinate (label2) at ($(label1) + 3.0*(\zerowidth, 0.0)$);
        \coordinate (label3) at ($(label2) + (1.5*\zerowidth, -\zeroheight)$);
        % -----------------------------------------------------------------------

        % --- Ancilliary reference points
        \coordinate (line1) at ($(label2) + (-3.0\zerowidth, -2*\zeroheight-0.15 cm)$);
        % -----------------------------------------------------------------------

        % --- Bounding Box ------------------------------------------------------
        \coordinate (bottom) at ($(line1) + 1.0*(0.0, -\zeroheight-\baselineskip-0.5/1.0*\zeroheight)$);
        \coordinate (right) at ($(label2) + (3*\zerowidth, \zeroheight)$);
        \draw [white] (bottom) rectangle (right);
        % -----------------------------------------------------------------------

        % show the box enclosing the divisor
        \draw [thick, rounded corners] ($(label2) + (0.0, \zeroheight)$) -- ($(label2) + (0.0, -\zeroheight)$) -- ($(label2) + 3.0*(\zerowidth, -\zeroheight/3.0)$);

        % show the box for writing the quotient
        \node [rounded corners, rectangle, minimum width=3.0*\zerowidth, minimum height = \zeroheight+\baselineskip, draw, below=0.15 cm of label3] {};
        % -----------------------------------------------------------------------
        
        % --- Text ------------------------------------------------------------

        % Dividend
        \node [right=0.0 cm of label1] (dividend) {\huge 7};

        % Divisor
        \node [right=0.0 cm of label2] (divisor) {\huge 4};
        % -----------------------------------------------------------------------
        
      \end{tikzpicture}
    \end{center}
  \end{minipage}
  \noindent\begin{minipage}{0.25\linewidth}
    \begin{center}
      \begin{tikzpicture}

        % --- Coordinates -------------------------------------------------------
        \coordinate (label1) at (0, 3.5);
        \coordinate (label2) at ($(label1) + 3.0*(\zerowidth, 0.0)$);
        \coordinate (label3) at ($(label2) + (1.5*\zerowidth, -\zeroheight)$);
        % -----------------------------------------------------------------------

        % --- Ancilliary reference points
        \coordinate (line1) at ($(label2) + (-3.0\zerowidth, -2*\zeroheight-0.15 cm)$);
        % -----------------------------------------------------------------------

        % --- Bounding Box ------------------------------------------------------
        \coordinate (bottom) at ($(line1) + 1.0*(0.0, -\zeroheight-\baselineskip-0.5/1.0*\zeroheight)$);
        \coordinate (right) at ($(label2) + (3*\zerowidth, \zeroheight)$);
        \draw [white] (bottom) rectangle (right);
        % -----------------------------------------------------------------------

        % show the box enclosing the divisor
        \draw [thick, rounded corners] ($(label2) + (0.0, \zeroheight)$) -- ($(label2) + (0.0, -\zeroheight)$) -- ($(label2) + 3.0*(\zerowidth, -\zeroheight/3.0)$);

        % show the box for writing the quotient
        \node [rounded corners, rectangle, minimum width=3.0*\zerowidth, minimum height = \zeroheight+\baselineskip, draw, below=0.15 cm of label3] {};
        % -----------------------------------------------------------------------
        
        % --- Text ------------------------------------------------------------

        % Dividend
        \node [right=0.0 cm of label1] (dividend) {\huge 5};

        % Divisor
        \node [right=0.0 cm of label2] (divisor) {\huge 1};
        % -----------------------------------------------------------------------
        
      \end{tikzpicture}
    \end{center}
  \end{minipage}
  \noindent\begin{minipage}{0.25\linewidth}
    \begin{center}
      \begin{tikzpicture}

        % --- Coordinates -------------------------------------------------------
        \coordinate (label1) at (0, 3.5);
        \coordinate (label2) at ($(label1) + 3.0*(\zerowidth, 0.0)$);
        \coordinate (label3) at ($(label2) + (1.5*\zerowidth, -\zeroheight)$);
        % -----------------------------------------------------------------------

        % --- Ancilliary reference points
        \coordinate (line1) at ($(label2) + (-3.0\zerowidth, -2*\zeroheight-0.15 cm)$);
        % -----------------------------------------------------------------------

        % --- Bounding Box ------------------------------------------------------
        \coordinate (bottom) at ($(line1) + 1.0*(0.0, -\zeroheight-\baselineskip-0.5/1.0*\zeroheight)$);
        \coordinate (right) at ($(label2) + (3*\zerowidth, \zeroheight)$);
        \draw [white] (bottom) rectangle (right);
        % -----------------------------------------------------------------------

        % show the box enclosing the divisor
        \draw [thick, rounded corners] ($(label2) + (0.0, \zeroheight)$) -- ($(label2) + (0.0, -\zeroheight)$) -- ($(label2) + 3.0*(\zerowidth, -\zeroheight/3.0)$);

        % show the box for writing the quotient
        \node [rounded corners, rectangle, minimum width=3.0*\zerowidth, minimum height = \zeroheight+\baselineskip, draw, below=0.15 cm of label3] {};
        % -----------------------------------------------------------------------
        
        % --- Text ------------------------------------------------------------

        % Dividend
        \node [right=0.0 cm of label1] (dividend) {\huge 3};

        % Divisor
        \node [right=0.0 cm of label2] (divisor) {\huge 2};
        % -----------------------------------------------------------------------
        
      \end{tikzpicture}
    \end{center}
  \end{minipage}
  \noindent\begin{minipage}{0.25\linewidth}
    \begin{center}
      \begin{tikzpicture}

        % --- Coordinates -------------------------------------------------------
        \coordinate (label1) at (0, 3.5);
        \coordinate (label2) at ($(label1) + 3.0*(\zerowidth, 0.0)$);
        \coordinate (label3) at ($(label2) + (1.5*\zerowidth, -\zeroheight)$);
        % -----------------------------------------------------------------------

        % --- Ancilliary reference points
        \coordinate (line1) at ($(label2) + (-3.0\zerowidth, -2*\zeroheight-0.15 cm)$);
        % -----------------------------------------------------------------------

        % --- Bounding Box ------------------------------------------------------
        \coordinate (bottom) at ($(line1) + 1.0*(0.0, -\zeroheight-\baselineskip-0.5/1.0*\zeroheight)$);
        \coordinate (right) at ($(label2) + (3*\zerowidth, \zeroheight)$);
        \draw [white] (bottom) rectangle (right);
        % -----------------------------------------------------------------------

        % show the box enclosing the divisor
        \draw [thick, rounded corners] ($(label2) + (0.0, \zeroheight)$) -- ($(label2) + (0.0, -\zeroheight)$) -- ($(label2) + 3.0*(\zerowidth, -\zeroheight/3.0)$);

        % show the box for writing the quotient
        \node [rounded corners, rectangle, minimum width=3.0*\zerowidth, minimum height = \zeroheight+\baselineskip, draw, below=0.15 cm of label3] {};
        % -----------------------------------------------------------------------
        
        % --- Text ------------------------------------------------------------

        % Dividend
        \node [right=0.0 cm of label1] (dividend) {\huge 8};

        % Divisor
        \node [right=0.0 cm of label2] (divisor) {\huge 4};
        % -----------------------------------------------------------------------
        
      \end{tikzpicture}
    \end{center}
  \end{minipage}
  \noindent\begin{minipage}{0.25\linewidth}
    \begin{center}
      \begin{tikzpicture}

        % --- Coordinates -------------------------------------------------------
        \coordinate (label1) at (0, 3.5);
        \coordinate (label2) at ($(label1) + 3.0*(\zerowidth, 0.0)$);
        \coordinate (label3) at ($(label2) + (1.5*\zerowidth, -\zeroheight)$);
        % -----------------------------------------------------------------------

        % --- Ancilliary reference points
        \coordinate (line1) at ($(label2) + (-3.0\zerowidth, -2*\zeroheight-0.15 cm)$);
        % -----------------------------------------------------------------------

        % --- Bounding Box ------------------------------------------------------
        \coordinate (bottom) at ($(line1) + 1.0*(0.0, -\zeroheight-\baselineskip-0.5/1.0*\zeroheight)$);
        \coordinate (right) at ($(label2) + (3*\zerowidth, \zeroheight)$);
        \draw [white] (bottom) rectangle (right);
        % -----------------------------------------------------------------------

        % show the box enclosing the divisor
        \draw [thick, rounded corners] ($(label2) + (0.0, \zeroheight)$) -- ($(label2) + (0.0, -\zeroheight)$) -- ($(label2) + 3.0*(\zerowidth, -\zeroheight/3.0)$);

        % show the box for writing the quotient
        \node [rounded corners, rectangle, minimum width=3.0*\zerowidth, minimum height = \zeroheight+\baselineskip, draw, below=0.15 cm of label3] {};
        % -----------------------------------------------------------------------
        
        % --- Text ------------------------------------------------------------

        % Dividend
        \node [right=0.0 cm of label1] (dividend) {\huge 9};

        % Divisor
        \node [right=0.0 cm of label2] (divisor) {\huge 3};
        % -----------------------------------------------------------------------
        
      \end{tikzpicture}
    \end{center}
  \end{minipage}
  \noindent\begin{minipage}{0.25\linewidth}
    \begin{center}
      \begin{tikzpicture}

        % --- Coordinates -------------------------------------------------------
        \coordinate (label1) at (0, 3.5);
        \coordinate (label2) at ($(label1) + 3.0*(\zerowidth, 0.0)$);
        \coordinate (label3) at ($(label2) + (1.5*\zerowidth, -\zeroheight)$);
        % -----------------------------------------------------------------------

        % --- Ancilliary reference points
        \coordinate (line1) at ($(label2) + (-3.0\zerowidth, -2*\zeroheight-0.15 cm)$);
        % -----------------------------------------------------------------------

        % --- Bounding Box ------------------------------------------------------
        \coordinate (bottom) at ($(line1) + 1.0*(0.0, -\zeroheight-\baselineskip-0.5/1.0*\zeroheight)$);
        \coordinate (right) at ($(label2) + (3*\zerowidth, \zeroheight)$);
        \draw [white] (bottom) rectangle (right);
        % -----------------------------------------------------------------------

        % show the box enclosing the divisor
        \draw [thick, rounded corners] ($(label2) + (0.0, \zeroheight)$) -- ($(label2) + (0.0, -\zeroheight)$) -- ($(label2) + 3.0*(\zerowidth, -\zeroheight/3.0)$);

        % show the box for writing the quotient
        \node [rounded corners, rectangle, minimum width=3.0*\zerowidth, minimum height = \zeroheight+\baselineskip, draw, below=0.15 cm of label3] {};
        % -----------------------------------------------------------------------
        
        % --- Text ------------------------------------------------------------

        % Dividend
        \node [right=0.0 cm of label1] (dividend) {\huge 7};

        % Divisor
        \node [right=0.0 cm of label2] (divisor) {\huge 1};
        % -----------------------------------------------------------------------
        
      \end{tikzpicture}
    \end{center}
  \end{minipage}
  

  \question
  En segundo lugar, se muestran varias divisiones con las siguientes características:

  \begin{itemize}
    
  \item Número de dígitos del dividendo: 2
    
  \item Número de dígitos del divisor: 1
    
  \item Número de dígitos necesarios para representar el cociente: 1
    
  \end{itemize}

  El esquema general se muestra a continuación:

  \noindent\begin{minipage}{0.25\linewidth}
    \begin{center}
      \begin{tikzpicture}

        % in the following, let:
        %
        % nbdvdigits: number of digits required for displaying the
        % dividend
        %
        % nbdrdigits: number of digits required for displaying the
        % divisor
        %
        % nbqdigits: number of digits necessary for writing the
        % quotient
        % 
        % nbrows: number of rows required for making the
        % calculations. This number is exactly equal to twice the
        % number of digits of the quotient, as each digit in the
        % quotient requires two lines for making calculations.

        % parameters:
        %
        % \zerowidth: width of a digit
        % \zeroheight: height of a digit
        %
        % Also, \baselineskip is used to take into account the natural
        % space between lines
        
        % --- Coordinates -------------------------------------------------------
        
        % The first label gets at x=0 and it marks where the dividend
        % should be written.
        %
        % Let Y denote the maximum vertical space required to draw the
        % whole picture. Y is computed as the total number of rows of
        % the figure plus 0.5 to give some slack. The total number of
        % rows required is nbrows + 1 (+1 for displaying the dividend)
        \coordinate (label1) at (0, 3.5);
        \fill [red] (label1) circle (1pt);

        % the location of the divisor is computed by adding the space
        % of *two* digits to nbdvdigits to leave some space between
        % the dividend and the divisor
        \coordinate (label2) at ($(label1) + 4.0*(\zerowidth, 0.0)$);
        \fill [red] (label2) circle (1pt);

        % The third coordinate defines the location which splits the
        % divisor and the quotient. Let W denote the width of the box
        % that encloses the divisor, W = 2 + max (nbdrdigts, nbqdigits)
        %
        % W is computed taking into account the number of digits
        % required for writing both the dividend and the quotient,
        % which is max (nbdrdigits, nbqdigits), plus 2 digits to leave
        % some space around. Then the third label is computed as the
        % midpoint.
        %
        % Also, this label is lowered the height of a digit. 
        \coordinate (label3) at ($(label2) + (1.5*\zerowidth, -\zeroheight)$);
        \fill [red] (label3) circle (1pt);
        % -----------------------------------------------------------------------

        % --- Ancilliary reference points

        % the first line below the dividend is shown next. It is
        % aligned with the left margin of the whole bounding box, so
        % that the number 4 substracts the same horizontal distance
        % added when computing label2. Twice the height of a digit
        % plus 0.15 cm is substracted *always* so that the first line
        % gets properly aligned with the quotient
        \coordinate (line1) at ($(label2) + (-4.0\zerowidth, -2*\zeroheight-0.15 cm)$);
        \fill [blue] (line1) circle (1pt);

        % the second line below the dividend is shown next. All these
        % lines are computed just by simply leaving a vertical space
        % equal to the sum of the height of a digit and the
        % baselineskip
        \coordinate (line2) at ($(line1) + 1*(0.0, -\zeroheight-\baselineskip)$);
        \fill [blue] (line2) circle (1pt);
        % -----------------------------------------------------------------------

        % --- Bounding Box ------------------------------------------------------

        % the bottom of the whole bounding box can be computed from
        % the first line just by decrementing the height as many lines
        % as required for making all the calculations (minus one,
        % because we start in the first line) minus half the
        % height of a digit for the last row.
        %
        % The number of rows required for making all the calculations
        % is twice the number of digits in the quotient, but one has
        % to be substracted because we start from the first line, so
        % that the factor used next is 2*nbqdigits - 1. Note that the
        % same factor is used to divide the last part of the
        % computation of the y-shift so that it is always equal to
        % half the height of a digit
        \coordinate (bottom) at ($(line1) + 1.0*(0.0, -\zeroheight-\baselineskip-0.5/1.0*\zeroheight)$);
        \fill [green] (bottom) circle (1pt);

        % the right border is computed just by adding to the second
        % label as many digits as required to draw both the quotient
        % and the divisor plus two ---+2 to leave some space
        % around. This number was also used above to compute the
        % location of label3 and it is equal to:
        % 2 + max (nbdrdigits, nbqdigits)
        \coordinate (right) at ($(label2) + (3*\zerowidth, \zeroheight)$);
        \fill [green] (right) circle (1pt);
        
        % draw an invisible box used to properly align all sequences
        % \draw [lightgray] (0,0) rectangle (3.5,5.0);
        \draw [lightgray] (bottom) rectangle (right);
        
        % -----------------------------------------------------------------------

        % show the box enclosing the divisor

        % the width of the box has a width equal to 2 + max
        % (nbdrdigits, nbqdigits), and it has a height equal to twice
        % the height of a digit. Note that label2 falls in the middle
        % of the vertical bar and label3 falls in the middle of the
        % horizontal bar
        \draw [thick, rounded corners] ($(label2) + (0.0, \zeroheight)$) -- ($(label2) + (0.0, -\zeroheight)$) -- ($(label2) + 3.0*(\zerowidth, -\zeroheight/3.0)$);

        % show the box for writing the quotient

        % the width of the box equals 2 + max (nbdrdigits, nbqdigits),
        % and its height equals the sum of the height of a digit plus
        % the baselineskip. It is drawn 0.15 cms from the line
        % separating the divisor from the quotient
        \node [rounded corners, rectangle, minimum width=3.0*\zerowidth, minimum height = \zeroheight+\baselineskip, draw, below=0.15 cm of label3] {\textcolor{lightgray}{\huge 5}};

        
        % --- Text

        % Dividend
        \node [right=0.0 cm of label1] (dividend) {\huge 37};

        % Divisor
        \node [right=0.0 cm of label2] (divisor) {\huge 7};

        % calculations
        \node [lightgray,right=0.0cm of line1] (result1) {\huge 35};
        \node [lightgray,right=\zerowidth of line2] (result2) {\huge 2};
        
      \end{tikzpicture}
    \end{center}
  \end{minipage}

  A continuación se muestra una vista de la composición de diferentes
  divisiones de este tipo:
  
  \noindent\begin{minipage}{0.25\linewidth}
    \begin{center}
      \begin{tikzpicture}

        % --- Coordinates -------------------------------------------------------
        \coordinate (label1) at (0, 3.5);
        \coordinate (label2) at ($(label1) + 4.0*(\zerowidth, 0.0)$);
        \coordinate (label3) at ($(label2) + (1.5*\zerowidth, -\zeroheight)$);
        % -----------------------------------------------------------------------

        % --- Ancilliary reference points
        \coordinate (line1) at ($(label2) + (-4.0\zerowidth, -2*\zeroheight-0.15 cm)$);
        % -----------------------------------------------------------------------

        % --- Bounding Box ------------------------------------------------------
        \coordinate (bottom) at ($(line1) + 1.0*(0.0, -\zeroheight-\baselineskip-0.5/1.0*\zeroheight)$);
        \coordinate (right) at ($(label2) + (3*\zerowidth, \zeroheight)$);
        \draw [white] (bottom) rectangle (right);
        % -----------------------------------------------------------------------

        % show the box enclosing the divisor
        \draw [thick, rounded corners] ($(label2) + (0.0, \zeroheight)$) -- ($(label2) + (0.0, -\zeroheight)$) -- ($(label2) + 3.0*(\zerowidth, -\zeroheight/3.0)$);

        % show the box for writing the quotient
        \node [rounded corners, rectangle, minimum width=3.0*\zerowidth, minimum height = \zeroheight+\baselineskip, draw, below=0.15 cm of label3] {};
        % -----------------------------------------------------------------------
        
        % --- Text ------------------------------------------------------------

        % Dividend
        \node [right=0.0 cm of label1] (dividend) {\huge 37};

        % Divisor
        \node [right=0.0 cm of label2] (divisor) {\huge 7};
        % -----------------------------------------------------------------------
        
      \end{tikzpicture}
    \end{center}
  \end{minipage}
  \noindent\begin{minipage}{0.25\linewidth}
    \begin{center}
      \begin{tikzpicture}

        % --- Coordinates -------------------------------------------------------
        \coordinate (label1) at (0, 3.5);
        \coordinate (label2) at ($(label1) + 4.0*(\zerowidth, 0.0)$);
        \coordinate (label3) at ($(label2) + (1.5*\zerowidth, -\zeroheight)$);
        % -----------------------------------------------------------------------

        % --- Ancilliary reference points
        \coordinate (line1) at ($(label2) + (-4.0\zerowidth, -2*\zeroheight-0.15 cm)$);
        % -----------------------------------------------------------------------

        % --- Bounding Box ------------------------------------------------------
        \coordinate (bottom) at ($(line1) + 1.0*(0.0, -\zeroheight-\baselineskip-0.5/1.0*\zeroheight)$);
        \coordinate (right) at ($(label2) + (3*\zerowidth, \zeroheight)$);
        \draw [white] (bottom) rectangle (right);
        % -----------------------------------------------------------------------

        % show the box enclosing the divisor
        \draw [thick, rounded corners] ($(label2) + (0.0, \zeroheight)$) -- ($(label2) + (0.0, -\zeroheight)$) -- ($(label2) + 3.0*(\zerowidth, -\zeroheight/3.0)$);

        % show the box for writing the quotient
        \node [rounded corners, rectangle, minimum width=3.0*\zerowidth, minimum height = \zeroheight+\baselineskip, draw, below=0.15 cm of label3] {};
        % -----------------------------------------------------------------------
        
        % --- Text ------------------------------------------------------------

        % Dividend
        \node [right=0.0 cm of label1] (dividend) {\huge 43};

        % Divisor
        \node [right=0.0 cm of label2] (divisor) {\huge 9};
        % -----------------------------------------------------------------------
        
      \end{tikzpicture}
    \end{center}
  \end{minipage}
  \noindent\begin{minipage}{0.25\linewidth}
    \begin{center}
      \begin{tikzpicture}

        % --- Coordinates -------------------------------------------------------
        \coordinate (label1) at (0, 3.5);
        \coordinate (label2) at ($(label1) + 4.0*(\zerowidth, 0.0)$);
        \coordinate (label3) at ($(label2) + (1.5*\zerowidth, -\zeroheight)$);
        % -----------------------------------------------------------------------

        % --- Ancilliary reference points
        \coordinate (line1) at ($(label2) + (-4.0\zerowidth, -2*\zeroheight-0.15 cm)$);
        % -----------------------------------------------------------------------

        % --- Bounding Box ------------------------------------------------------
        \coordinate (bottom) at ($(line1) + 1.0*(0.0, -\zeroheight-\baselineskip-0.5/1.0*\zeroheight)$);
        \coordinate (right) at ($(label2) + (3*\zerowidth, \zeroheight)$);
        \draw [white] (bottom) rectangle (right);
        % -----------------------------------------------------------------------

        % show the box enclosing the divisor
        \draw [thick, rounded corners] ($(label2) + (0.0, \zeroheight)$) -- ($(label2) + (0.0, -\zeroheight)$) -- ($(label2) + 3.0*(\zerowidth, -\zeroheight/3.0)$);

        % show the box for writing the quotient
        \node [rounded corners, rectangle, minimum width=3.0*\zerowidth, minimum height = \zeroheight+\baselineskip, draw, below=0.15 cm of label3] {};
        % -----------------------------------------------------------------------
        
        % --- Text ------------------------------------------------------------

        % Dividend
        \node [right=0.0 cm of label1] (dividend) {\huge 16};

        % Divisor
        \node [right=0.0 cm of label2] (divisor) {\huge 2};
        % -----------------------------------------------------------------------
        
      \end{tikzpicture}
    \end{center}
  \end{minipage}
  \noindent\begin{minipage}{0.25\linewidth}
    \begin{center}
      \begin{tikzpicture}

        % --- Coordinates -------------------------------------------------------
        \coordinate (label1) at (0, 3.5);
        \coordinate (label2) at ($(label1) + 4.0*(\zerowidth, 0.0)$);
        \coordinate (label3) at ($(label2) + (1.5*\zerowidth, -\zeroheight)$);
        % -----------------------------------------------------------------------

        % --- Ancilliary reference points
        \coordinate (line1) at ($(label2) + (-4.0\zerowidth, -2*\zeroheight-0.15 cm)$);
        % -----------------------------------------------------------------------

        % --- Bounding Box ------------------------------------------------------
        \coordinate (bottom) at ($(line1) + 1.0*(0.0, -\zeroheight-\baselineskip-0.5/1.0*\zeroheight)$);
        \coordinate (right) at ($(label2) + (3*\zerowidth, \zeroheight)$);
        \draw [white] (bottom) rectangle (right);
        % -----------------------------------------------------------------------

        % show the box enclosing the divisor
        \draw [thick, rounded corners] ($(label2) + (0.0, \zeroheight)$) -- ($(label2) + (0.0, -\zeroheight)$) -- ($(label2) + 3.0*(\zerowidth, -\zeroheight/3.0)$);

        % show the box for writing the quotient
        \node [rounded corners, rectangle, minimum width=3.0*\zerowidth, minimum height = \zeroheight+\baselineskip, draw, below=0.15 cm of label3] {};
        % -----------------------------------------------------------------------
        
        % --- Text ------------------------------------------------------------

        % Dividend
        \node [right=0.0 cm of label1] (dividend) {\huge 13};

        % Divisor
        \node [right=0.0 cm of label2] (divisor) {\huge 3};
        % -----------------------------------------------------------------------
        
      \end{tikzpicture}
    \end{center}
  \end{minipage}
  \noindent\begin{minipage}{0.25\linewidth}
    \begin{center}
      \begin{tikzpicture}

        % --- Coordinates -------------------------------------------------------
        \coordinate (label1) at (0, 3.5);
        \coordinate (label2) at ($(label1) + 4.0*(\zerowidth, 0.0)$);
        \coordinate (label3) at ($(label2) + (1.5*\zerowidth, -\zeroheight)$);
        % -----------------------------------------------------------------------

        % --- Ancilliary reference points
        \coordinate (line1) at ($(label2) + (-4.0\zerowidth, -2*\zeroheight-0.15 cm)$);
        % -----------------------------------------------------------------------

        % --- Bounding Box ------------------------------------------------------
        \coordinate (bottom) at ($(line1) + 1.0*(0.0, -\zeroheight-\baselineskip-0.5/1.0*\zeroheight)$);
        \coordinate (right) at ($(label2) + (3*\zerowidth, \zeroheight)$);
        \draw [white] (bottom) rectangle (right);
        % -----------------------------------------------------------------------

        % show the box enclosing the divisor
        \draw [thick, rounded corners] ($(label2) + (0.0, \zeroheight)$) -- ($(label2) + (0.0, -\zeroheight)$) -- ($(label2) + 3.0*(\zerowidth, -\zeroheight/3.0)$);

        % show the box for writing the quotient
        \node [rounded corners, rectangle, minimum width=3.0*\zerowidth, minimum height = \zeroheight+\baselineskip, draw, below=0.15 cm of label3] {};
        % -----------------------------------------------------------------------
        
        % --- Text ------------------------------------------------------------

        % Dividend
        \node [right=0.0 cm of label1] (dividend) {\huge 26};

        % Divisor
        \node [right=0.0 cm of label2] (divisor) {\huge 5};
        % -----------------------------------------------------------------------
        
      \end{tikzpicture}
    \end{center}
  \end{minipage}
  \noindent\begin{minipage}{0.25\linewidth}
    \begin{center}
      \begin{tikzpicture}

        % --- Coordinates -------------------------------------------------------
        \coordinate (label1) at (0, 3.5);
        \coordinate (label2) at ($(label1) + 4.0*(\zerowidth, 0.0)$);
        \coordinate (label3) at ($(label2) + (1.5*\zerowidth, -\zeroheight)$);
        % -----------------------------------------------------------------------

        % --- Ancilliary reference points
        \coordinate (line1) at ($(label2) + (-4.0\zerowidth, -2*\zeroheight-0.15 cm)$);
        % -----------------------------------------------------------------------

        % --- Bounding Box ------------------------------------------------------
        \coordinate (bottom) at ($(line1) + 1.0*(0.0, -\zeroheight-\baselineskip-0.5/1.0*\zeroheight)$);
        \coordinate (right) at ($(label2) + (3*\zerowidth, \zeroheight)$);
        \draw [white] (bottom) rectangle (right);
        % -----------------------------------------------------------------------

        % show the box enclosing the divisor
        \draw [thick, rounded corners] ($(label2) + (0.0, \zeroheight)$) -- ($(label2) + (0.0, -\zeroheight)$) -- ($(label2) + 3.0*(\zerowidth, -\zeroheight/3.0)$);

        % show the box for writing the quotient
        \node [rounded corners, rectangle, minimum width=3.0*\zerowidth, minimum height = \zeroheight+\baselineskip, draw, below=0.15 cm of label3] {};
        % -----------------------------------------------------------------------
        
        % --- Text ------------------------------------------------------------

        % Dividend
        \node [right=0.0 cm of label1] (dividend) {\huge 52};

        % Divisor
        \node [right=0.0 cm of label2] (divisor) {\huge 6};
        % -----------------------------------------------------------------------
        
      \end{tikzpicture}
    \end{center}
  \end{minipage}
  \noindent\begin{minipage}{0.25\linewidth}
    \begin{center}
      \begin{tikzpicture}

        % --- Coordinates -------------------------------------------------------
        \coordinate (label1) at (0, 3.5);
        \coordinate (label2) at ($(label1) + 4.0*(\zerowidth, 0.0)$);
        \coordinate (label3) at ($(label2) + (1.5*\zerowidth, -\zeroheight)$);
        % -----------------------------------------------------------------------

        % --- Ancilliary reference points
        \coordinate (line1) at ($(label2) + (-4.0\zerowidth, -2*\zeroheight-0.15 cm)$);
        % -----------------------------------------------------------------------

        % --- Bounding Box ------------------------------------------------------
        \coordinate (bottom) at ($(line1) + 1.0*(0.0, -\zeroheight-\baselineskip-0.5/1.0*\zeroheight)$);
        \coordinate (right) at ($(label2) + (3*\zerowidth, \zeroheight)$);
        \draw [white] (bottom) rectangle (right);
        % -----------------------------------------------------------------------

        % show the box enclosing the divisor
        \draw [thick, rounded corners] ($(label2) + (0.0, \zeroheight)$) -- ($(label2) + (0.0, -\zeroheight)$) -- ($(label2) + 3.0*(\zerowidth, -\zeroheight/3.0)$);

        % show the box for writing the quotient
        \node [rounded corners, rectangle, minimum width=3.0*\zerowidth, minimum height = \zeroheight+\baselineskip, draw, below=0.15 cm of label3] {};
        % -----------------------------------------------------------------------
        
        % --- Text ------------------------------------------------------------

        % Dividend
        \node [right=0.0 cm of label1] (dividend) {\huge 33};

        % Divisor
        \node [right=0.0 cm of label2] (divisor) {\huge 4};
        % -----------------------------------------------------------------------
        
      \end{tikzpicture}
    \end{center}
  \end{minipage}
  \noindent\begin{minipage}{0.25\linewidth}
    \begin{center}
      \begin{tikzpicture}

        % --- Coordinates -------------------------------------------------------
        \coordinate (label1) at (0, 3.5);
        \coordinate (label2) at ($(label1) + 4.0*(\zerowidth, 0.0)$);
        \coordinate (label3) at ($(label2) + (1.5*\zerowidth, -\zeroheight)$);
        % -----------------------------------------------------------------------

        % --- Ancilliary reference points
        \coordinate (line1) at ($(label2) + (-4.0\zerowidth, -2*\zeroheight-0.15 cm)$);
        % -----------------------------------------------------------------------

        % --- Bounding Box ------------------------------------------------------
        \coordinate (bottom) at ($(line1) + 1.0*(0.0, -\zeroheight-\baselineskip-0.5/1.0*\zeroheight)$);
        \coordinate (right) at ($(label2) + (3*\zerowidth, \zeroheight)$);
        \draw [white] (bottom) rectangle (right);
        % -----------------------------------------------------------------------

        % show the box enclosing the divisor
        \draw [thick, rounded corners] ($(label2) + (0.0, \zeroheight)$) -- ($(label2) + (0.0, -\zeroheight)$) -- ($(label2) + 3.0*(\zerowidth, -\zeroheight/3.0)$);

        % show the box for writing the quotient
        \node [rounded corners, rectangle, minimum width=3.0*\zerowidth, minimum height = \zeroheight+\baselineskip, draw, below=0.15 cm of label3] {};
        % -----------------------------------------------------------------------
        
        % --- Text ------------------------------------------------------------

        % Dividend
        \node [right=0.0 cm of label1] (dividend) {\huge 71};

        % Divisor
        \node [right=0.0 cm of label2] (divisor) {\huge 8};
        % -----------------------------------------------------------------------
        
      \end{tikzpicture}
    \end{center}
  \end{minipage}
  
  
  \question
  A continuación se muestran otras divisiones con las siguientes características:

  \begin{itemize}
    
  \item Número de dígitos del dividendo: 2
    
  \item Número de dígitos del divisor: 1
    
  \item Número de dígitos necesarios para representar el cociente: 2
    
  \end{itemize}  
  
  El esquema general se muestra a continuación:

  \noindent\begin{minipage}{0.25\linewidth}
    \begin{center}
      \begin{tikzpicture}

        % in the following, let:
        %
        % nbdvdigits: number of digits required for displaying the
        % dividend
        %
        % nbdrdigits: number of digits required for displaying the
        % divisor
        %
        % nbqdigits: number of digits necessary for writing the
        % quotient
        % 
        % nbrows: number of rows required for making the
        % calculations. This number is exactly equal to twice the
        % number of digits of the quotient, as each digit in the
        % quotient requires two lines for making calculations.

        % parameters:
        %
        % \zerowidth: width of a digit
        % \zeroheight: height of a digit
        %
        % Also, \baselineskip is used to take into account the natural
        % space between lines
        
        % --- Coordinates -------------------------------------------------------
        
        % The first label gets at x=0 and it marks where the dividend
        % should be written.
        %
        % Let Y denote the maximum vertical space required to draw the
        % whole picture. Y is computed as the total number of rows of
        % the figure plus 0.5 to give some slack. The total number of
        % rows required is nbrows + 1 (+1 for displaying the dividend)
        \coordinate (label1) at (0, 5.5);
        \fill [red] (label1) circle (1pt);

        % the location of the divisor is computed by adding the space
        % of *two* digits to nbdvdigits to leave some space between
        % the dividend and the divisor
        \coordinate (label2) at ($(label1) + 4.0*(\zerowidth, 0.0)$);
        \fill [red] (label2) circle (1pt);

        % The third coordinate defines the location which splits the
        % divisor and the quotient. Let W denote the width of the box
        % that encloses the divisor, W = 2 + max (nbdrdigts, nbqdigits)
        %
        % W is computed taking into account the number of digits
        % required for writing both the dividend and the quotient,
        % which is max (nbdrdigits, nbqdigits), plus 2 digits to leave
        % some space around. Then the third label is computed as the
        % midpoint.
        %
        % Also, this label is lowered the height of a digit. 
        \coordinate (label3) at ($(label2) + (2*\zerowidth, -\zeroheight)$);
        \fill [red] (label3) circle (1pt);
        % -----------------------------------------------------------------------

        % --- Ancilliary reference points

        % the first line below the dividend is shown next. It is
        % aligned with the left margin of the whole bounding box, so
        % that the number 3 substracts the same horizontal distance
        % added when computing label2. Twice the height of a digit
        % plus 0.15 cm is substracted *always* so that the first line
        % gets properly aligned with the quotient
        \coordinate (line1) at ($(label2) + (-4.0\zerowidth, -2*\zeroheight-0.15 cm)$);
        \fill [blue] (line1) circle (1pt);

        % the second line below the dividend is shown next. All these
        % lines are computed just by simply leaving a vertical space
        % equal to the sum of the height of a digit and the
        % baselineskip
        \coordinate (line2) at ($(line1) + 1*(0.0, -\zeroheight-\baselineskip)$);
        \fill [blue] (line2) circle (1pt);
        \coordinate (line3) at ($(line1) + 2*(0.0, -\zeroheight-\baselineskip)$);
        \fill [blue] (line3) circle (1pt);
        \coordinate (line4) at ($(line1) + 3*(0.0, -\zeroheight-\baselineskip)$);
        \fill [blue] (line4) circle (1pt);
        % -----------------------------------------------------------------------

        % --- Bounding Box ------------------------------------------------------

        % the bottom of the whole bounding box can be computed from
        % the first line just by decrementing the height as many lines
        % as required for making all the calculations (minus one,
        % because we start in the first line) minus half the
        % height of a digit for the last row.
        %
        % The number of rows required for making all the calculations
        % is twice the number of digits in the quotient, but one has
        % to be substracted because we start from the first line, so
        % that the factor used next is 2*nbqdigits - 1. Note that the
        % same factor is used to divide the last part of the
        % computation of the y-shift so that it is always equal to
        % half the height of a digit
        \coordinate (bottom) at ($(line1) + 3.0*(0.0, -\zeroheight-\baselineskip-0.5/3.0*\zeroheight)$);
        \fill [green] (bottom) circle (1pt);

        % the right border is computed just by adding to the second
        % label as many digits as required to draw both the quotient
        % and the divisor plus two ---+2 to leave some space
        % around. This number was also used above to compute the
        % location of label3 and it is equal to:
        % 2 + max (nbdrdigits, nbqdigits)
        \coordinate (right) at ($(label2) + (4*\zerowidth, \zeroheight)$);
        \fill [green] (right) circle (1pt);
        
        % draw an invisible box used to properly align all sequences
        % \draw [lightgray] (0,0) rectangle (3.5,5.0);
        \draw [lightgray] (bottom) rectangle (right);
        
        % -----------------------------------------------------------------------

        % show the box enclosing the divisor

        % the width of the box has a width equal to 2 + max
        % (nbdrdigits, nbqdigits), and it has a height equal to twice
        % the height of a digit. Note that label2 falls in the middle
        % of the vertical bar and label3 falls in the middle of the
        % horizontal bar
        \draw [thick, rounded corners] ($(label2) + (0.0, \zeroheight)$) -- ($(label2) + (0.0, -\zeroheight)$) -- ($(label2) + 4.0*(\zerowidth, -\zeroheight/4.0)$);

        % show the box for writing the quotient

        % the width of the box equals 2 + max (nbdrdigits, nbqdigits),
        % and its height equals the sum of the height of a digit plus
        % the baselineskip. It is drawn 0.15 cms from the line
        % separating the divisor from the quotient
        \node [rounded corners, rectangle, minimum width=4.0*\zerowidth, minimum height = \zeroheight+\baselineskip, draw, below=0.15 cm of label3] {\textcolor{lightgray}{\huge 11}};

        
        % --- Text

        % Dividend
        \node [right=0.0 cm of label1] (dividend) {\huge 23};

        % Divisor
        \node [right=0.0 cm of label2] (divisor) {\huge 2};

        % calculations
        \node [lightgray,right=0.0cm of line1] (result1) {\huge 2};
        \node [lightgray,right=0.0cm of line2] (result2) {\huge 03};
        \node [lightgray,right=\zerowidth of line3] (result2) {\huge 2};
        \node [lightgray,right=\zerowidth of line4] (result2) {\huge 1};
        
      \end{tikzpicture}
    \end{center}
  \end{minipage}

  A continuación se muestra una vista de la composición de diferentes
  divisiones de este tipo:
  
  \noindent\begin{minipage}{0.25\linewidth}
    \begin{center}
      \begin{tikzpicture}

        % --- Coordinates -------------------------------------------------------
        \coordinate (label1) at (0, 5.5);
        \coordinate (label2) at ($(label1) + 4.0*(\zerowidth, 0.0)$);
        \coordinate (label3) at ($(label2) + (2*\zerowidth, -\zeroheight)$);
        % -----------------------------------------------------------------------

        % --- Ancilliary reference points
        \coordinate (line1) at ($(label2) + (-4.0\zerowidth, -2*\zeroheight-0.15 cm)$);
        % -----------------------------------------------------------------------

        % --- Bounding Box ------------------------------------------------------
        \coordinate (bottom) at ($(line1) + 3.0*(0.0, -\zeroheight-\baselineskip-0.5/3.0*\zeroheight)$);
        \coordinate (right) at ($(label2) + (4*\zerowidth, \zeroheight)$);
        \draw [white] (bottom) rectangle (right);
        % -----------------------------------------------------------------------

        % show the box enclosing the divisor
        \draw [thick, rounded corners] ($(label2) + (0.0, \zeroheight)$) -- ($(label2) + (0.0, -\zeroheight)$) -- ($(label2) + 4.0*(\zerowidth, -\zeroheight/4.0)$);

        % show the box for writing the quotient
        \node [rounded corners, rectangle, minimum width=4.0*\zerowidth, minimum height = \zeroheight+\baselineskip, draw, below=0.15 cm of label3] {};
        % -----------------------------------------------------------------------
        
        % --- Text ------------------------------------------------------------

        % Dividend
        \node [right=0.0 cm of label1] (dividend) {\huge 23};

        % Divisor
        \node [right=0.0 cm of label2] (divisor) {\huge 2};
        % -----------------------------------------------------------------------
        
      \end{tikzpicture}
    \end{center}
  \end{minipage}
  \noindent\begin{minipage}{0.25\linewidth}
    \begin{center}
      \begin{tikzpicture}

        % --- Coordinates -------------------------------------------------------
        \coordinate (label1) at (0, 5.5);
        \coordinate (label2) at ($(label1) + 4.0*(\zerowidth, 0.0)$);
        \coordinate (label3) at ($(label2) + (2*\zerowidth, -\zeroheight)$);
        % -----------------------------------------------------------------------

        % --- Ancilliary reference points
        \coordinate (line1) at ($(label2) + (-4.0\zerowidth, -2*\zeroheight-0.15 cm)$);
        % -----------------------------------------------------------------------

        % --- Bounding Box ------------------------------------------------------
        \coordinate (bottom) at ($(line1) + 3.0*(0.0, -\zeroheight-\baselineskip-0.5/3.0*\zeroheight)$);
        \coordinate (right) at ($(label2) + (4*\zerowidth, \zeroheight)$);
        \draw [white] (bottom) rectangle (right);
        % -----------------------------------------------------------------------

        % show the box enclosing the divisor
        \draw [thick, rounded corners] ($(label2) + (0.0, \zeroheight)$) -- ($(label2) + (0.0, -\zeroheight)$) -- ($(label2) + 4.0*(\zerowidth, -\zeroheight/4.0)$);

        % show the box for writing the quotient
        \node [rounded corners, rectangle, minimum width=4.0*\zerowidth, minimum height = \zeroheight+\baselineskip, draw, below=0.15 cm of label3] {};
        % -----------------------------------------------------------------------
        
        % --- Text ------------------------------------------------------------

        % Dividend
        \node [right=0.0 cm of label1] (dividend) {\huge 93};

        % Divisor
        \node [right=0.0 cm of label2] (divisor) {\huge 6};
        % -----------------------------------------------------------------------
        
      \end{tikzpicture}
    \end{center}
  \end{minipage}
  \noindent\begin{minipage}{0.25\linewidth}
    \begin{center}
      \begin{tikzpicture}

        % --- Coordinates -------------------------------------------------------
        \coordinate (label1) at (0, 5.5);
        \coordinate (label2) at ($(label1) + 4.0*(\zerowidth, 0.0)$);
        \coordinate (label3) at ($(label2) + (2*\zerowidth, -\zeroheight)$);
        % -----------------------------------------------------------------------

        % --- Ancilliary reference points
        \coordinate (line1) at ($(label2) + (-4.0\zerowidth, -2*\zeroheight-0.15 cm)$);
        % -----------------------------------------------------------------------

        % --- Bounding Box ------------------------------------------------------
        \coordinate (bottom) at ($(line1) + 3.0*(0.0, -\zeroheight-\baselineskip-0.5/3.0*\zeroheight)$);
        \coordinate (right) at ($(label2) + (4*\zerowidth, \zeroheight)$);
        \draw [white] (bottom) rectangle (right);
        % -----------------------------------------------------------------------

        % show the box enclosing the divisor
        \draw [thick, rounded corners] ($(label2) + (0.0, \zeroheight)$) -- ($(label2) + (0.0, -\zeroheight)$) -- ($(label2) + 4.0*(\zerowidth, -\zeroheight/4.0)$);

        % show the box for writing the quotient
        \node [rounded corners, rectangle, minimum width=4.0*\zerowidth, minimum height = \zeroheight+\baselineskip, draw, below=0.15 cm of label3] {};
        % -----------------------------------------------------------------------
        
        % --- Text ------------------------------------------------------------

        % Dividend
        \node [right=0.0 cm of label1] (dividend) {\huge 96};

        % Divisor
        \node [right=0.0 cm of label2] (divisor) {\huge 5};
        % -----------------------------------------------------------------------
        
      \end{tikzpicture}
    \end{center}
  \end{minipage}
  \noindent\begin{minipage}{0.25\linewidth}
    \begin{center}
      \begin{tikzpicture}

        % --- Coordinates -------------------------------------------------------
        \coordinate (label1) at (0, 5.5);
        \coordinate (label2) at ($(label1) + 4.0*(\zerowidth, 0.0)$);
        \coordinate (label3) at ($(label2) + (2*\zerowidth, -\zeroheight)$);
        % -----------------------------------------------------------------------

        % --- Ancilliary reference points
        \coordinate (line1) at ($(label2) + (-4.0\zerowidth, -2*\zeroheight-0.15 cm)$);
        % -----------------------------------------------------------------------

        % --- Bounding Box ------------------------------------------------------
        \coordinate (bottom) at ($(line1) + 3.0*(0.0, -\zeroheight-\baselineskip-0.5/3.0*\zeroheight)$);
        \coordinate (right) at ($(label2) + (4*\zerowidth, \zeroheight)$);
        \draw [white] (bottom) rectangle (right);
        % -----------------------------------------------------------------------

        % show the box enclosing the divisor
        \draw [thick, rounded corners] ($(label2) + (0.0, \zeroheight)$) -- ($(label2) + (0.0, -\zeroheight)$) -- ($(label2) + 4.0*(\zerowidth, -\zeroheight/4.0)$);

        % show the box for writing the quotient
        \node [rounded corners, rectangle, minimum width=4.0*\zerowidth, minimum height = \zeroheight+\baselineskip, draw, below=0.15 cm of label3] {};
        % -----------------------------------------------------------------------
        
        % --- Text ------------------------------------------------------------

        % Dividend
        \node [right=0.0 cm of label1] (dividend) {\huge 89};

        % Divisor
        \node [right=0.0 cm of label2] (divisor) {\huge 7};
        % -----------------------------------------------------------------------
        
      \end{tikzpicture}
    \end{center}
  \end{minipage}
  \noindent\begin{minipage}{0.25\linewidth}
    \begin{center}
      \begin{tikzpicture}

        % --- Coordinates -------------------------------------------------------
        \coordinate (label1) at (0, 5.5);
        \coordinate (label2) at ($(label1) + 4.0*(\zerowidth, 0.0)$);
        \coordinate (label3) at ($(label2) + (2*\zerowidth, -\zeroheight)$);
        % -----------------------------------------------------------------------

        % --- Ancilliary reference points
        \coordinate (line1) at ($(label2) + (-4.0\zerowidth, -2*\zeroheight-0.15 cm)$);
        % -----------------------------------------------------------------------

        % --- Bounding Box ------------------------------------------------------
        \coordinate (bottom) at ($(line1) + 3.0*(0.0, -\zeroheight-\baselineskip-0.5/3.0*\zeroheight)$);
        \coordinate (right) at ($(label2) + (4*\zerowidth, \zeroheight)$);
        \draw [white] (bottom) rectangle (right);
        % -----------------------------------------------------------------------

        % show the box enclosing the divisor
        \draw [thick, rounded corners] ($(label2) + (0.0, \zeroheight)$) -- ($(label2) + (0.0, -\zeroheight)$) -- ($(label2) + 4.0*(\zerowidth, -\zeroheight/4.0)$);

        % show the box for writing the quotient
        \node [rounded corners, rectangle, minimum width=4.0*\zerowidth, minimum height = \zeroheight+\baselineskip, draw, below=0.15 cm of label3] {};
        % -----------------------------------------------------------------------
        
        % --- Text ------------------------------------------------------------

        % Dividend
        \node [right=0.0 cm of label1] (dividend) {\huge 57};

        % Divisor
        \node [right=0.0 cm of label2] (divisor) {\huge 3};
        % -----------------------------------------------------------------------
        
      \end{tikzpicture}
    \end{center}
  \end{minipage}
  \noindent\begin{minipage}{0.25\linewidth}
    \begin{center}
      \begin{tikzpicture}

        % --- Coordinates -------------------------------------------------------
        \coordinate (label1) at (0, 5.5);
        \coordinate (label2) at ($(label1) + 4.0*(\zerowidth, 0.0)$);
        \coordinate (label3) at ($(label2) + (2*\zerowidth, -\zeroheight)$);
        % -----------------------------------------------------------------------

        % --- Ancilliary reference points
        \coordinate (line1) at ($(label2) + (-4.0\zerowidth, -2*\zeroheight-0.15 cm)$);
        % -----------------------------------------------------------------------

        % --- Bounding Box ------------------------------------------------------
        \coordinate (bottom) at ($(line1) + 3.0*(0.0, -\zeroheight-\baselineskip-0.5/3.0*\zeroheight)$);
        \coordinate (right) at ($(label2) + (4*\zerowidth, \zeroheight)$);
        \draw [white] (bottom) rectangle (right);
        % -----------------------------------------------------------------------

        % show the box enclosing the divisor
        \draw [thick, rounded corners] ($(label2) + (0.0, \zeroheight)$) -- ($(label2) + (0.0, -\zeroheight)$) -- ($(label2) + 4.0*(\zerowidth, -\zeroheight/4.0)$);

        % show the box for writing the quotient
        \node [rounded corners, rectangle, minimum width=4.0*\zerowidth, minimum height = \zeroheight+\baselineskip, draw, below=0.15 cm of label3] {};
        % -----------------------------------------------------------------------
        
        % --- Text ------------------------------------------------------------

        % Dividend
        \node [right=0.0 cm of label1] (dividend) {\huge 71};

        % Divisor
        \node [right=0.0 cm of label2] (divisor) {\huge 2};
        % -----------------------------------------------------------------------
        
      \end{tikzpicture}
    \end{center}
  \end{minipage}
  \noindent\begin{minipage}{0.25\linewidth}
    \begin{center}
      \begin{tikzpicture}

        % --- Coordinates -------------------------------------------------------
        \coordinate (label1) at (0, 5.5);
        \coordinate (label2) at ($(label1) + 4.0*(\zerowidth, 0.0)$);
        \coordinate (label3) at ($(label2) + (2*\zerowidth, -\zeroheight)$);
        % -----------------------------------------------------------------------

        % --- Ancilliary reference points
        \coordinate (line1) at ($(label2) + (-4.0\zerowidth, -2*\zeroheight-0.15 cm)$);
        % -----------------------------------------------------------------------

        % --- Bounding Box ------------------------------------------------------
        \coordinate (bottom) at ($(line1) + 3.0*(0.0, -\zeroheight-\baselineskip-0.5/3.0*\zeroheight)$);
        \coordinate (right) at ($(label2) + (4*\zerowidth, \zeroheight)$);
        \draw [white] (bottom) rectangle (right);
        % -----------------------------------------------------------------------

        % show the box enclosing the divisor
        \draw [thick, rounded corners] ($(label2) + (0.0, \zeroheight)$) -- ($(label2) + (0.0, -\zeroheight)$) -- ($(label2) + 4.0*(\zerowidth, -\zeroheight/4.0)$);

        % show the box for writing the quotient
        \node [rounded corners, rectangle, minimum width=4.0*\zerowidth, minimum height = \zeroheight+\baselineskip, draw, below=0.15 cm of label3] {};
        % -----------------------------------------------------------------------
        
        % --- Text ------------------------------------------------------------

        % Dividend
        \node [right=0.0 cm of label1] (dividend) {\huge 82};

        % Divisor
        \node [right=0.0 cm of label2] (divisor) {\huge 8};
        % -----------------------------------------------------------------------
        
      \end{tikzpicture}
    \end{center}
  \end{minipage}
  \noindent\begin{minipage}{0.25\linewidth}
    \begin{center}
      \begin{tikzpicture}

        % --- Coordinates -------------------------------------------------------
        \coordinate (label1) at (0, 5.5);
        \coordinate (label2) at ($(label1) + 4.0*(\zerowidth, 0.0)$);
        \coordinate (label3) at ($(label2) + (2*\zerowidth, -\zeroheight)$);
        % -----------------------------------------------------------------------

        % --- Ancilliary reference points
        \coordinate (line1) at ($(label2) + (-4.0\zerowidth, -2*\zeroheight-0.15 cm)$);
        % -----------------------------------------------------------------------

        % --- Bounding Box ------------------------------------------------------
        \coordinate (bottom) at ($(line1) + 3.0*(0.0, -\zeroheight-\baselineskip-0.5/3.0*\zeroheight)$);
        \coordinate (right) at ($(label2) + (4*\zerowidth, \zeroheight)$);
        \draw [white] (bottom) rectangle (right);
        % -----------------------------------------------------------------------

        % show the box enclosing the divisor
        \draw [thick, rounded corners] ($(label2) + (0.0, \zeroheight)$) -- ($(label2) + (0.0, -\zeroheight)$) -- ($(label2) + 4.0*(\zerowidth, -\zeroheight/4.0)$);

        % show the box for writing the quotient
        \node [rounded corners, rectangle, minimum width=4.0*\zerowidth, minimum height = \zeroheight+\baselineskip, draw, below=0.15 cm of label3] {};
        % -----------------------------------------------------------------------
        
        % --- Text ------------------------------------------------------------

        % Dividend
        \node [right=0.0 cm of label1] (dividend) {\huge 91};

        % Divisor
        \node [right=0.0 cm of label2] (divisor) {\huge 9};
        % -----------------------------------------------------------------------
        
      \end{tikzpicture}
    \end{center}
  \end{minipage}
  
  \question
  Por último, se muestran divisiones más complicadas:

  \begin{itemize}
    
  \item Número de dígitos del dividendo: 5
    
  \item Número de dígitos del divisor: 1
    
  \item Número de dígitos necesarios para representar el cociente: 4
    
  \end{itemize}  
  
  El esquema general se muestra a continuación:

  \noindent\begin{minipage}{0.25\linewidth}
    \begin{center}
      \begin{tikzpicture}

        % in the following, let:
        %
        % nbdvdigits: number of digits required for displaying the
        % dividend
        %
        % nbdrdigits: number of digits required for displaying the
        % divisor
        %
        % nbqdigits: number of digits necessary for writing the
        % quotient
        % 
        % nbrows: number of rows required for making the
        % calculations. This number is exactly equal to twice the
        % number of digits of the quotient, as each digit in the
        % quotient requires two lines for making calculations.

        % parameters:
        %
        % \zerowidth: width of a digit
        % \zeroheight: height of a digit
        %
        % Also, \baselineskip is used to take into account the natural
        % space between lines
        
        % --- Coordinates -------------------------------------------------------
        
        % The first label gets at x=0 and it marks where the dividend
        % should be written.
        %
        % Let Y denote the maximum vertical space required to draw the
        % whole picture. Y is computed as the total number of rows of
        % the figure plus 0.5 to give some slack. The total number of
        % rows required is nbrows + 1 (+1 for displaying the dividend)
        \coordinate (label1) at (0, 11.5);
        \fill [red] (label1) circle (1pt);

        % the location of the divisor is computed by adding the space
        % of *two* digits to nbdvdigits to leave some space between
        % the dividend and the divisor
        \coordinate (label2) at ($(label1) + 7.0*(\zerowidth, 0.0)$);
        \fill [red] (label2) circle (1pt);

        % The third coordinate defines the location which splits the
        % divisor and the quotient. Let W denote the width of the box
        % that encloses the divisor, W = 2 + max (nbdrdigts, nbqdigits)
        %
        % W is computed taking into account the number of digits
        % required for writing both the dividend and the quotient,
        % which is max (nbdrdigits, nbqdigits), plus 2 digits to leave
        % some space around. Then the third label is computed as the
        % midpoint.
        %
        % Also, this label is lowered the height of a digit. 
        \coordinate (label3) at ($(label2) + (3.5*\zerowidth, -\zeroheight)$);
        \fill [red] (label3) circle (1pt);
        % -----------------------------------------------------------------------

        % --- Ancilliary reference points

        % the first line below the dividend is shown next. It is
        % aligned with the left margin of the whole bounding box, so
        % that the number 3 substracts the same horizontal distance
        % added when computing label2. Twice the height of a digit
        % plus 0.15 cm is substracted *always* so that the first line
        % gets properly aligned with the quotient
        \coordinate (line1) at ($(label2) + (-7.0\zerowidth, -2*\zeroheight-0.15 cm)$);
        \fill [blue] (line1) circle (1pt);

        % the second line below the dividend is shown next. All these
        % lines are computed just by simply leaving a vertical space
        % equal to the sum of the height of a digit and the
        % baselineskip
        \coordinate (line2) at ($(line1) + 1*(0.0, -\zeroheight-\baselineskip)$);
        \fill [blue] (line2) circle (1pt);
        \coordinate (line3) at ($(line1) + 2*(0.0, -\zeroheight-\baselineskip)$);
        \fill [blue] (line3) circle (1pt);
        \coordinate (line4) at ($(line1) + 3*(0.0, -\zeroheight-\baselineskip)$);
        \fill [blue] (line4) circle (1pt);
        \coordinate (line5) at ($(line1) + 4*(0.0, -\zeroheight-\baselineskip)$);
        \fill [blue] (line5) circle (1pt);
        \coordinate (line6) at ($(line1) + 5*(0.0, -\zeroheight-\baselineskip)$);
        \fill [blue] (line6) circle (1pt);
        \coordinate (line7) at ($(line1) + 6*(0.0, -\zeroheight-\baselineskip)$);
        \fill [blue] (line7) circle (1pt);
        \coordinate (line8) at ($(line1) + 7*(0.0, -\zeroheight-\baselineskip)$);
        \fill [blue] (line8) circle (1pt);
        \coordinate (line9) at ($(line1) + 8*(0.0, -\zeroheight-\baselineskip)$);
        \fill [blue] (line9) circle (1pt);
        \coordinate (line10) at ($(line1) + 9*(0.0, -\zeroheight-\baselineskip)$);
        \fill [blue] (line10) circle (1pt);
        % -----------------------------------------------------------------------

        % --- Bounding Box ------------------------------------------------------

        % the bottom of the whole bounding box can be computed from
        % the first line just by decrementing the height as many lines
        % as required for making all the calculations (minus one,
        % because we start in the first line) minus half the
        % height of a digit for the last row.
        %
        % The number of rows required for making all the calculations
        % is twice the number of digits in the quotient, but one has
        % to be substracted because we start from the first line, so
        % that the factor used next is 2*nbqdigits - 1. Note that the
        % same factor is used to divide the last part of the
        % computation of the y-shift so that it is always equal to
        % half the height of a digit
        \coordinate (bottom) at ($(line1) + 9.0*(0.0, -\zeroheight-\baselineskip-0.5/9.0*\zeroheight)$);
        \fill [green] (bottom) circle (1pt);

        % the right border is computed just by adding to the second
        % label as many digits as required to draw both the quotient
        % and the divisor plus two ---+2 to leave some space
        % around. This number was also used above to compute the
        % location of label3 and it is equal to:
        % 2 + max (nbdrdigits, nbqdigits)
        \coordinate (right) at ($(label2) + (7*\zerowidth, \zeroheight)$);
        \fill [green] (right) circle (1pt);
        
        % draw an invisible box used to properly align all sequences
        % \draw [lightgray] (0,0) rectangle (3.5,5.0);
        \draw [lightgray] (bottom) rectangle (right);
        
        % -----------------------------------------------------------------------

        % show the box enclosing the divisor

        % the width of the box has a width equal to 2 + max
        % (nbdrdigits, nbqdigits), and it has a height equal to twice
        % the height of a digit. Note that label2 falls in the middle
        % of the vertical bar and label3 falls in the middle of the
        % horizontal bar
        \draw [thick, rounded corners] ($(label2) + (0.0, \zeroheight)$) -- ($(label2) + (0.0, -\zeroheight)$) -- ($(label2) + 7.0*(\zerowidth, -\zeroheight/7.0)$);

        % show the box for writing the quotient

        % the width of the box equals 2 + max (nbdrdigits, nbqdigits),
        % and its height equals the sum of the height of a digit plus
        % the baselineskip. It is drawn 0.15 cms from the line
        % separating the divisor from the quotient
        \node [rounded corners, rectangle, minimum width=7.0*\zerowidth, minimum height = \zeroheight+\baselineskip, draw, below=0.15 cm of label3] {\textcolor{lightgray}{\huge 12801}};

        
        % --- Text

        % Dividend
        \node [right=0.0 cm of label1] (dividend) {\huge 76810};

        % Divisor
        \node [right=0.0 cm of label2] (divisor) {\huge 6};

        % calculations
        \node [lightgray,right=0.0cm of line1] (result1) {\huge 6};
        \node [lightgray,right=0.0cm of line2] (result2) {\huge 16};
        \node [lightgray,right=0.0cm of line3] (result3) {\huge 12};
        \node [lightgray,right=\zerowidth of line4] (result4) {\huge 48};
        \node [lightgray,right=\zerowidth of line5] (result5) {\huge 48};
        \node [lightgray,right=2*\zerowidth of line6] (result6) {\huge 01};
        \node [lightgray,right=3*\zerowidth of line7] (result7) {\huge 0};
        \node [lightgray,right=3*\zerowidth of line8] (result8) {\huge 10};
        \node [lightgray,right=4*\zerowidth of line9] (result9) {\huge 6};
        \node [lightgray,right=4*\zerowidth of line10] (result10) {\huge 4};
        
      \end{tikzpicture}
    \end{center}
  \end{minipage}

  A continuación se muestra una vista de la composición de diferentes
  divisiones de este tipo. Nótese que, en este caso, cada
  \textit{minipage} ocupa sólo el 0.33 del ancho de una línea.

  \clearpage
  
  \noindent\begin{minipage}{0.33\linewidth}
    \begin{center}
      \begin{tikzpicture}

        % --- Coordinates -------------------------------------------------------
        \coordinate (label1) at (0, 11.5);
        \coordinate (label2) at ($(label1) + 7.0*(\zerowidth, 0.0)$);
        \coordinate (label3) at ($(label2) + (3.5*\zerowidth, -\zeroheight)$);
        % -----------------------------------------------------------------------

        % --- Ancilliary reference points
        \coordinate (line1) at ($(label2) + (-7.0\zerowidth, -2*\zeroheight-0.15 cm)$);
        % -----------------------------------------------------------------------

        % --- Bounding Box ------------------------------------------------------
        \coordinate (bottom) at ($(line1) + 9.0*(0.0, -\zeroheight-\baselineskip-0.5/9.0*\zeroheight)$);
        \coordinate (right) at ($(label2) + (7*\zerowidth, \zeroheight)$);
        \draw [white] (bottom) rectangle (right);
        % -----------------------------------------------------------------------

        % show the box enclosing the divisor
        \draw [thick, rounded corners] ($(label2) + (0.0, \zeroheight)$) -- ($(label2) + (0.0, -\zeroheight)$) -- ($(label2) + 7.0*(\zerowidth, -\zeroheight/7.0)$);

        % show the box for writing the quotient
        \node [rounded corners, rectangle, minimum width=7.0*\zerowidth, minimum height = \zeroheight+\baselineskip, draw, below=0.15 cm of label3] {};
        % -----------------------------------------------------------------------
        
        % --- Text ------------------------------------------------------------

        % Dividend
        \node [right=0.0 cm of label1] (dividend) {\huge 76810};

        % Divisor
        \node [right=0.0 cm of label2] (divisor) {\huge 6};
        % -----------------------------------------------------------------------
        
      \end{tikzpicture}
    \end{center}
  \end{minipage}
  \noindent\begin{minipage}{0.33\linewidth}
    \begin{center}
      \begin{tikzpicture}

        % --- Coordinates -------------------------------------------------------
        \coordinate (label1) at (0, 11.5);
        \coordinate (label2) at ($(label1) + 7.0*(\zerowidth, 0.0)$);
        \coordinate (label3) at ($(label2) + (3.5*\zerowidth, -\zeroheight)$);
        % -----------------------------------------------------------------------

        % --- Ancilliary reference points
        \coordinate (line1) at ($(label2) + (-7.0\zerowidth, -2*\zeroheight-0.15 cm)$);
        % -----------------------------------------------------------------------

        % --- Bounding Box ------------------------------------------------------
        \coordinate (bottom) at ($(line1) + 9.0*(0.0, -\zeroheight-\baselineskip-0.5/9.0*\zeroheight)$);
        \coordinate (right) at ($(label2) + (7*\zerowidth, \zeroheight)$);
        \draw [white] (bottom) rectangle (right);
        % -----------------------------------------------------------------------

        % show the box enclosing the divisor
        \draw [thick, rounded corners] ($(label2) + (0.0, \zeroheight)$) -- ($(label2) + (0.0, -\zeroheight)$) -- ($(label2) + 7.0*(\zerowidth, -\zeroheight/7.0)$);

        % show the box for writing the quotient
        \node [rounded corners, rectangle, minimum width=7.0*\zerowidth, minimum height = \zeroheight+\baselineskip, draw, below=0.15 cm of label3] {};
        % -----------------------------------------------------------------------
        
        % --- Text ------------------------------------------------------------

        % Dividend
        \node [right=0.0 cm of label1] (dividend) {\huge 49365};

        % Divisor
        \node [right=0.0 cm of label2] (divisor) {\huge 5};
        % -----------------------------------------------------------------------
        
      \end{tikzpicture}
    \end{center}
  \end{minipage}
  \noindent\begin{minipage}{0.33\linewidth}
    \begin{center}
      \begin{tikzpicture}

        % --- Coordinates -------------------------------------------------------
        \coordinate (label1) at (0, 11.5);
        \coordinate (label2) at ($(label1) + 7.0*(\zerowidth, 0.0)$);
        \coordinate (label3) at ($(label2) + (3.5*\zerowidth, -\zeroheight)$);
        % -----------------------------------------------------------------------

        % --- Ancilliary reference points
        \coordinate (line1) at ($(label2) + (-7.0\zerowidth, -2*\zeroheight-0.15 cm)$);
        % -----------------------------------------------------------------------

        % --- Bounding Box ------------------------------------------------------
        \coordinate (bottom) at ($(line1) + 9.0*(0.0, -\zeroheight-\baselineskip-0.5/9.0*\zeroheight)$);
        \coordinate (right) at ($(label2) + (7*\zerowidth, \zeroheight)$);
        \draw [white] (bottom) rectangle (right);
        % -----------------------------------------------------------------------

        % show the box enclosing the divisor
        \draw [thick, rounded corners] ($(label2) + (0.0, \zeroheight)$) -- ($(label2) + (0.0, -\zeroheight)$) -- ($(label2) + 7.0*(\zerowidth, -\zeroheight/7.0)$);

        % show the box for writing the quotient
        \node [rounded corners, rectangle, minimum width=7.0*\zerowidth, minimum height = \zeroheight+\baselineskip, draw, below=0.15 cm of label3] {};
        % -----------------------------------------------------------------------
        
        % --- Text ------------------------------------------------------------

        % Dividend
        \node [right=0.0 cm of label1] (dividend) {\huge 29367};

        % Divisor
        \node [right=0.0 cm of label2] (divisor) {\huge 7};
        % -----------------------------------------------------------------------
        
      \end{tikzpicture}
    \end{center}
  \end{minipage}
  \noindent\begin{minipage}{0.33\linewidth}
    \begin{center}
      \begin{tikzpicture}

        % --- Coordinates -------------------------------------------------------
        \coordinate (label1) at (0, 11.5);
        \coordinate (label2) at ($(label1) + 7.0*(\zerowidth, 0.0)$);
        \coordinate (label3) at ($(label2) + (3.5*\zerowidth, -\zeroheight)$);
        % -----------------------------------------------------------------------

        % --- Ancilliary reference points
        \coordinate (line1) at ($(label2) + (-7.0\zerowidth, -2*\zeroheight-0.15 cm)$);
        % -----------------------------------------------------------------------

        % --- Bounding Box ------------------------------------------------------
        \coordinate (bottom) at ($(line1) + 9.0*(0.0, -\zeroheight-\baselineskip-0.5/9.0*\zeroheight)$);
        \coordinate (right) at ($(label2) + (7*\zerowidth, \zeroheight)$);
        \draw [white] (bottom) rectangle (right);
        % -----------------------------------------------------------------------

        % show the box enclosing the divisor
        \draw [thick, rounded corners] ($(label2) + (0.0, \zeroheight)$) -- ($(label2) + (0.0, -\zeroheight)$) -- ($(label2) + 7.0*(\zerowidth, -\zeroheight/7.0)$);

        % show the box for writing the quotient
        \node [rounded corners, rectangle, minimum width=7.0*\zerowidth, minimum height = \zeroheight+\baselineskip, draw, below=0.15 cm of label3] {};
        % -----------------------------------------------------------------------
        
        % --- Text ------------------------------------------------------------

        % Dividend
        \node [right=0.0 cm of label1] (dividend) {\huge 24384};

        % Divisor
        \node [right=0.0 cm of label2] (divisor) {\huge 3};
        % -----------------------------------------------------------------------
        
      \end{tikzpicture}
    \end{center}
  \end{minipage}
  \noindent\begin{minipage}{0.33\linewidth}
    \begin{center}
      \begin{tikzpicture}

        % --- Coordinates -------------------------------------------------------
        \coordinate (label1) at (0, 11.5);
        \coordinate (label2) at ($(label1) + 7.0*(\zerowidth, 0.0)$);
        \coordinate (label3) at ($(label2) + (3.5*\zerowidth, -\zeroheight)$);
        % -----------------------------------------------------------------------

        % --- Ancilliary reference points
        \coordinate (line1) at ($(label2) + (-7.0\zerowidth, -2*\zeroheight-0.15 cm)$);
        % -----------------------------------------------------------------------

        % --- Bounding Box ------------------------------------------------------
        \coordinate (bottom) at ($(line1) + 9.0*(0.0, -\zeroheight-\baselineskip-0.5/9.0*\zeroheight)$);
        \coordinate (right) at ($(label2) + (7*\zerowidth, \zeroheight)$);
        \draw [white] (bottom) rectangle (right);
        % -----------------------------------------------------------------------

        % show the box enclosing the divisor
        \draw [thick, rounded corners] ($(label2) + (0.0, \zeroheight)$) -- ($(label2) + (0.0, -\zeroheight)$) -- ($(label2) + 7.0*(\zerowidth, -\zeroheight/7.0)$);

        % show the box for writing the quotient
        \node [rounded corners, rectangle, minimum width=7.0*\zerowidth, minimum height = \zeroheight+\baselineskip, draw, below=0.15 cm of label3] {};
        % -----------------------------------------------------------------------
        
        % --- Text ------------------------------------------------------------

        % Dividend
        \node [right=0.0 cm of label1] (dividend) {\huge 17057};

        % Divisor
        \node [right=0.0 cm of label2] (divisor) {\huge 3};
        % -----------------------------------------------------------------------
        
      \end{tikzpicture}
    \end{center}
  \end{minipage}
  \noindent\begin{minipage}{0.33\linewidth}
    \begin{center}
      \begin{tikzpicture}

        % --- Coordinates -------------------------------------------------------
        \coordinate (label1) at (0, 11.5);
        \coordinate (label2) at ($(label1) + 7.0*(\zerowidth, 0.0)$);
        \coordinate (label3) at ($(label2) + (3.5*\zerowidth, -\zeroheight)$);
        % -----------------------------------------------------------------------

        % --- Ancilliary reference points
        \coordinate (line1) at ($(label2) + (-7.0\zerowidth, -2*\zeroheight-0.15 cm)$);
        % -----------------------------------------------------------------------

        % --- Bounding Box ------------------------------------------------------
        \coordinate (bottom) at ($(line1) + 9.0*(0.0, -\zeroheight-\baselineskip-0.5/9.0*\zeroheight)$);
        \coordinate (right) at ($(label2) + (7*\zerowidth, \zeroheight)$);
        \draw [white] (bottom) rectangle (right);
        % -----------------------------------------------------------------------

        % show the box enclosing the divisor
        \draw [thick, rounded corners] ($(label2) + (0.0, \zeroheight)$) -- ($(label2) + (0.0, -\zeroheight)$) -- ($(label2) + 7.0*(\zerowidth, -\zeroheight/7.0)$);

        % show the box for writing the quotient
        \node [rounded corners, rectangle, minimum width=7.0*\zerowidth, minimum height = \zeroheight+\baselineskip, draw, below=0.15 cm of label3] {};
        % -----------------------------------------------------------------------
        
        % --- Text ------------------------------------------------------------

        % Dividend
        \node [right=0.0 cm of label1] (dividend) {\huge 64981};

        % Divisor
        \node [right=0.0 cm of label2] (divisor) {\huge 8};
        % -----------------------------------------------------------------------
        
      \end{tikzpicture}
    \end{center}
  \end{minipage}

\end{questions}

\end{document}

%%% Local Variables:
%%% mode: latex
%%% TeX-master: t
%%% End:
