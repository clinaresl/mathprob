%%
%% main.tex
%% 
%% Made by 
%% Login   <clinares@atlas>
%% 
%% Started on  Sat May  4 14:24:58 2019 
%% Last update Sat May  4 14:24:58 2019 
%%

\documentclass[svgnames,addpoints]{exam}

\usepackage[T1]{fontenc}
\usepackage[utf8]{inputenc}
\usepackage[spanish]{babel}

\usepackage{tb}

\usepackage{amsfonts}
\usepackage{amssymb}
\usepackage{mathtools}

\usepackage{pifont}

\usepackage{cancel}
\usepackage{array}

\usepackage{tikz}
\usepackage{pgflibraryarrows}
\usepackage{pgflibrarysnakes}

\usetikzlibrary{calc,matrix,patterns,fadings,positioning}

\usepackage{xcolor}

\usepackage{array}
\usepackage{eurosym}

\usepackage{booktabs}
\usepackage{url}

\usepackage{rotating}

\newlength{\zerowidth}
\settowidth{\zerowidth}{\huge 0}
\newlength{\zeroheight}
\settoheight{\zeroheight}{\huge 0}

\makeatletter
\def\convertto#1#2{\strip@pt\dimexpr #2*65536/\number\dimexpr 1#1}
\makeatother


\begin{document}

\titulacion{Esquemas}
\asignatura{Multiplicaciones}

\convocatoria{\today}
\tiempo{4 horas}

\begin{tabular}{cc}

  \begin{minipage}{7.5cm}

    \begin{tabular}{c}

      \includegraphics[scale=0.6]{maths-1}\\

    \end{tabular}

  \end{minipage}
  &
  \begin{minipage}{8.5cm}

    \begin{center}

      {\huge \bf \titu}\\
      {\Large \bf \asig}\\ \ \\
      
      \convo

    \end{center}

  \end{minipage}\\ & \\ & \\  & \\

\end{tabular}

{\Large\bf Multiplicaciones}

Todas las representaciones que se muestran a continuación se hacen
empleando las siguientes medidas:

\begin{center}
  \begin{tabular}{ll}
    \texttt{zeroheight}   & \convertto{cm}{\the\zeroheight}\ cm \\
    \texttt{zerowidth}    & \convertto{cm}{\the\zerowidth}\ cm \\
  \end{tabular}
\end{center}

\noindent
y el \texttt{baselineskip} (\convertto{cm}{\the\baselineskip}\
cm). Las dos mostradas arriba deben definirse en el preámbulo del
documento \LaTeX\ (por ejemplo, con un fichero \texttt{.sty} que se
incluya automáticamente en el fichero). Además, cada operación se
muestra en una \textit{minipage} con un ancho igual a un cuarto el
ancho de la página, salvo el último (con dividendos más largos) que
requieren que cada una se haga en un tercio de la longitud de la
línea.

\begin{questions}

  \question Primero, se muestran varias multiplicaciones con las
  siguientes características:

  \begin{itemize}
    
  \item Número de dígitos del primer multiplicando: 1
    
  \item Número de dígitos del segundo multiplicando: 1
    
  \item Número de dígitos necesarios para representar el resultado: 2
    
  \end{itemize}

  El esquema general se muestra a continuación:

  \noindent\begin{minipage}{0.25\linewidth}
    \begin{center}
      \begin{tikzpicture}

        % --- Coordinates -------------------------------------------------------

        \coordinate (origin) at (0,0);
        \coordinate (label1) at ($(origin) + (3.5*\zerowidth, \zeroheight+2*(\zeroheight+\baselineskip)$);
        \fill [red] (label1) circle (1pt);

        \coordinate (label2) at ($(label1) + (0.0, -\zeroheight-\baselineskip)$);
        \fill [red] (label2) circle (1pt);

        \coordinate (label3) at ($(label2) + (-1.5*\zerowidth, 0.0)$);
        \fill [red] (label3) circle (1pt);
        % -----------------------------------------------------------------------

        % --- Ancilliary reference points

        \coordinate (line1) at ($(label2) + (0, -2*\zeroheight-0.15 cm)$);
        \fill [blue] (line1) circle (1pt);        
        
        % -----------------------------------------------------------------------

        % --- Bounding Box ------------------------------------------------------

        % \coordinate (bottom) at (0,0);
        \coordinate (bottom) at ($(label3) + (-2.0*\zerowidth, -2.5*\zeroheight-0.15 cm)$);
        \fill [green] (bottom) circle (1pt);

        \coordinate (right) at ($(label1) + (0, 0.5*\zeroheight)$);
        \fill [green] (right) circle (1pt);
        
        % draw an invisible box used to properly align all sequences
        % \draw [lightgray] (0,0) rectangle (3.5,5.0);
        \draw [lightgray] (bottom) rectangle (right);
        
        % -----------------------------------------------------------------------

        % show the horizontal rule
        \draw [thick] ($(label3) + (-2.0*\zerowidth, -\zeroheight)$) -- ($(label2) + (0, -\zeroheight)$);
        
        % --- Text
        \node [left=0.0 cm of label1] (multiplierA) {\huge 9};
        \node [left=0.0 cm of label2] (multiplierB) {\huge 7};
        \node [left=0.0 cm of label3] (operand) {\huge $\times$};
        
        \node [left=0.0 cm of line1] (result1) {\huge 63};
        
      \end{tikzpicture}
    \end{center}
  \end{minipage}

  A continuación se muestra una vista de la composición de diferentes
  divisiones de este tipo:
  

\end{questions}

\end{document}

%%% Local Variables:
%%% mode: latex
%%% TeX-master: t
%%% End:
