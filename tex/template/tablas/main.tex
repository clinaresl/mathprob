%%
%% main.tex
%% 
%% Made by 
%% Login   <clinares@atlas>
%% 
%% Started on  Sat May  4 14:24:58 2019 
%% Last update Sat May  4 14:24:58 2019 
%%

\documentclass[svgnames,addpoints]{exam}

\usepackage[T1]{fontenc}
\usepackage[utf8]{inputenc}
\usepackage[spanish]{babel}

\usepackage{tb}

\usepackage{amsfonts}
\usepackage{amssymb}
\usepackage{mathtools}

\usepackage{pifont}

\usepackage{cancel}
\usepackage{array}

\usepackage{tikz}
\usepackage{pgflibraryarrows}
\usepackage{pgflibrarysnakes}

\usetikzlibrary{calc,matrix,patterns,fadings,positioning}

\usepackage{xcolor}

\usepackage{array}
\usepackage{eurosym}

\usepackage{booktabs}
\usepackage{url}

\usepackage{rotating}

\newlength{\zerowidth}
\settowidth{\zerowidth}{\huge 0}
\newlength{\zeroheight}
\settoheight{\zeroheight}{\huge 0}

\makeatletter
\def\convertto#1#2{\strip@pt\dimexpr #2*65536/\number\dimexpr 1#1}
\makeatother


\begin{document}

\titulacion{Esquemas}
\asignatura{Tablas de multiplicar}

\convocatoria{\today}
\tiempo{4 horas}

\begin{tabular}{cc}

  \begin{minipage}{7.5cm}

    \begin{tabular}{c}

      \includegraphics[scale=0.15]{tablas-multiplicar}\\

    \end{tabular}

  \end{minipage}
  &
  \begin{minipage}{8.5cm}

    \begin{center}

      {\huge \bf \titu}\\
      {\Large \bf \asig}\\ \ \\
      
      \convo

    \end{center}

  \end{minipage}\\ & \\ & \\  & \\

\end{tabular}

{\Large\bf Tablas de multiplicar}

Todas las representaciones que se muestran a continuación se hacen
empleando las siguientes medidas:

\begin{center}
  \begin{tabular}{ll}
    \texttt{zeroheight}   & \convertto{cm}{\the\zeroheight}\ cm \\
    \texttt{zerowidth}    & \convertto{cm}{\the\zerowidth}\ cm \\
  \end{tabular}
\end{center}

\noindent
y el \texttt{baselineskip} (\convertto{cm}{\the\baselineskip}\
cm). Las dos mostradas arriba deben definirse en el preámbulo del
documento \LaTeX\ (por ejemplo, con un fichero \texttt{.sty} que se
incluya automáticamente en el fichero). Además, cada operación se
muestra en una \textit{minipage} con un ancho igual a un cuarto el
ancho de la página, salvo el último (con dividendos más largos) que
requieren que cada una se haga en un tercio de la longitud de la
línea.

\begin{questions}

  \question Primero, se muestran varias multiplicaciones con las siguientes
  características:

  \begin{itemize}
    
    \item Número maximo de dígitos de la primera columna: 1
    
    \item Número maximo de dígitos de la segunda columna: 1
    
    \item Número maximo de dígitos necesarios para representar cada resultado: 1

    \item Número de multiplicaciones a presentar en la tabla: 4
    
  \end{itemize}

  El esquema general se muestra a continuación:

  \noindent\begin{minipage}{0.33\linewidth}
    \begin{center}
      \begin{tikzpicture}

        % --- Coordinates -----------------------------------------------------

        \coordinate (bottom) at (0,0);
        \fill [blue] (bottom) circle (1pt);

        % el desplazamiento vertical desde el punto inferior izquierdo hasta
        % donde se escribe el primer numero de la última serie incluye una
        % distancia adicional de \baselineskip para facilitar que unas tablas se
        % puedan apilar sobre otras
        \coordinate (label11) at ($(bottom) + (1.5\zerowidth, 0.5\zeroheight+1.5\baselineskip)$);
        \fill [red] (label11) circle (1pt);
        \coordinate (operator1) at ($(label11) + (2.5*\zerowidth, 0.0)$);
        \fill [red] (operator1) circle (1pt);
        \coordinate (label12) at ($(operator1) + (2.5*\zerowidth, 0.0)$);
        \fill [red] (label12) circle (1pt);
        \coordinate (equal1) at ($(label12) + (2.5*\zerowidth, 0.0)$);
        \fill [red] (equal1) circle (1pt);

        % el punto de referencia que marca el punto central de la caja en la que
        % se escribe el resultado se desplaza multiplicando \zerowidth por la
        % mitad del ancho que tendrá la caja más 1. Puesto que en este caso se
        % puede dibujar un sólo digito en la caja de resultado su ancho es 3.0
        % y, por lo tanto, el desplazamiento es 3/2+1=2.5
        \coordinate (answer1) at ($(equal1) + (2.5*\zerowidth, 0.0)$);
        \fill [red] (answer1) circle (1pt);

        \coordinate (label21) at ($(label11) + (0.0, \zeroheight+1.5*\baselineskip)$);
        \fill [red] (label21) circle (1pt);
        \coordinate (operator2) at ($(label21) + (2.5*\zerowidth, 0.0)$);
        \fill [red] (operator2) circle (1pt);
        \coordinate (label22) at ($(operator2) + (2.5*\zerowidth, 0.0)$);
        \fill [red] (label22) circle (1pt);
        \coordinate (equal2) at ($(label22) + (2.5*\zerowidth, 0.0)$);
        \fill [red] (equal2) circle (1pt);
        \coordinate (answer2) at ($(equal2) + (2.5*\zerowidth, 0.0)$);
        \fill [red] (answer2) circle (1pt);

        \coordinate (label31) at ($(label21) + (0.0, \zeroheight+1.5*\baselineskip)$);
        \fill [red] (label31) circle (1pt);
        \coordinate (operator3) at ($(label31) + (2.5*\zerowidth, 0.0)$);
        \fill [red] (operator3) circle (1pt);
        \coordinate (label32) at ($(operator3) + (2.5*\zerowidth, 0.0)$);
        \fill [red] (label32) circle (1pt);
        \coordinate (equal3) at ($(label32) + (2.5*\zerowidth, 0.0)$);
        \fill [red] (equal3) circle (1pt);
        \coordinate (answer3) at ($(equal3) + (2.5*\zerowidth, 0.0)$);
        \fill [red] (answer3) circle (1pt);

        \coordinate (label41) at ($(label31) + (0.0, \zeroheight+1.5*\baselineskip)$);
        \fill [red] (label41) circle (1pt);
        \coordinate (operator4) at ($(label41) + (2.5*\zerowidth, 0.0)$);
        \fill [red] (operator4) circle (1pt);
        \coordinate (label42) at ($(operator4) + (2.5*\zerowidth, 0.0)$);
        \fill [red] (label42) circle (1pt);
        \coordinate (equal4) at ($(label42) + (2.5*\zerowidth, 0.0)$);
        \fill [red] (equal4) circle (1pt);
        \coordinate (answer4) at ($(equal4) + (2.5*\zerowidth, 0.0)$);
        \fill [red] (answer4) circle (1pt);

        % ---------------------------------------------------------------------

        % --- Bounding Box ----------------------------------------------------

        % the distance between the top right corner and the box created for
        % writing the answer in the first row is the same than the distance
        % between the bottom and and the first operand of the last row
        \coordinate (right) at ($(answer4) + (1.5\zerowidth, 0.5\zeroheight + 0.5\baselineskip)$);
        \fill [green] (right) circle (1pt);

        % draw an invisible box used to properly align all sequences
        \draw [lightgray] (bottom) rectangle (right);
       
        % ---------------------------------------------------------------------

        % --- Table -----------------------------------------------------------

        % numbers in the first column
        \draw (label11) node {\huge 4};
        \draw (operator1) node {\huge$\times$};
        \draw (label12) node {\huge 1};
        \draw (equal1) node {\huge$=$};

        % el ancho de una caja en la que se debe escribir un número es 2 más el
        % número de digítos que se escriben por el zerowidth. Como en este caso
        % solo se dibuja un
        % dígito, el ancho es 3 veces el zerowidth
        \draw(answer1) node [rounded corners, rectangle, minimum width=3.0*\zerowidth, minimum height = \zeroheight + \baselineskip, draw] {\textcolor{lightgray}{\huge 4}};

        \draw(label21) node [rounded corners, rectangle, minimum width=3.0*\zerowidth, minimum height = \zeroheight + \baselineskip, draw] {\textcolor{lightgray}{\huge 3}};
        \draw (operator2) node {\huge$\times$};
        \draw (label22) node {\huge 1};
        \draw (equal2) node {\huge$=$};
        \draw(answer2) node [rounded corners, rectangle, minimum width=3.0*\zerowidth, minimum height = \zeroheight + \baselineskip, draw] {\textcolor{lightgray}{\huge 3}};

        \draw (label31) node {\huge 2};
        \draw (operator3) node {\huge$\times$};
        \draw(label32) node [rounded corners, rectangle, minimum width=3.0*\zerowidth, minimum height = \zeroheight + \baselineskip, draw] {\textcolor{lightgray}{\huge 1}};
        \draw (equal3) node {\huge$=$};
        \draw(answer3) node [rounded corners, rectangle, minimum width=3.0*\zerowidth, minimum height = \zeroheight + \baselineskip, draw] {\textcolor{lightgray}{\huge 2}};

        \draw(label41) node [rounded corners, rectangle, minimum width=3.0*\zerowidth, minimum height = \zeroheight + \baselineskip, draw] {\textcolor{lightgray}{\huge 1}};
        \draw (operator4) node {\huge$\times$};
        \draw(label42) node [rounded corners, rectangle, minimum width=3.0*\zerowidth, minimum height = \zeroheight + \baselineskip, draw] {\textcolor{lightgray}{\huge 1}};
        \draw (equal4) node {\huge$=$};
        \draw(answer4) node [rounded corners, rectangle, minimum width=3.0*\zerowidth, minimum height = \zeroheight + \baselineskip, draw] {\textcolor{lightgray}{\huge 1}};

        % ---------------------------------------------------------------------

      \end{tikzpicture}
    \end{center}
  \end{minipage}

  A continuación se muestra el único ejemplo que puede generarse con los
  parámetros anteriores:

  \noindent\begin{minipage}{0.33\linewidth}
    \begin{center}
      \begin{tikzpicture}

        % --- Coordinates -----------------------------------------------------

        \coordinate (bottom) at (0,0);

        \coordinate (label11) at ($(bottom) + (1.5\zerowidth, 0.5\zeroheight+1.5\baselineskip)$);
        \coordinate (operator1) at ($(label11) + (2.5*\zerowidth, 0.0)$);
        \coordinate (label12) at ($(operator1) + (2.5*\zerowidth, 0.0)$);
        \coordinate (equal1) at ($(label12) + (2.5*\zerowidth, 0.0)$);
        \coordinate (answer1) at ($(equal1) + (2.5*\zerowidth, 0.0)$);

        \coordinate (label21) at ($(label11) + (0.0, \zeroheight+1.5*\baselineskip)$);
        \coordinate (operator2) at ($(label21) + (2.5*\zerowidth, 0.0)$);
        \coordinate (label22) at ($(operator2) + (2.5*\zerowidth, 0.0)$);
        \coordinate (equal2) at ($(label22) + (2.5*\zerowidth, 0.0)$);
        \coordinate (answer2) at ($(equal2) + (2.5*\zerowidth, 0.0)$);

        \coordinate (label31) at ($(label21) + (0.0, \zeroheight+1.5*\baselineskip)$);
        \coordinate (operator3) at ($(label31) + (2.5*\zerowidth, 0.0)$);
        \coordinate (label32) at ($(operator3) + (2.5*\zerowidth, 0.0)$);
        \coordinate (equal3) at ($(label32) + (2.5*\zerowidth, 0.0)$);
        \coordinate (answer3) at ($(equal3) + (2.5*\zerowidth, 0.0)$);

        \coordinate (label41) at ($(label31) + (0.0, \zeroheight+1.5*\baselineskip)$);
        \coordinate (operator4) at ($(label41) + (2.5*\zerowidth, 0.0)$);
        \coordinate (label42) at ($(operator4) + (2.5*\zerowidth, 0.0)$);
        \coordinate (equal4) at ($(label42) + (2.5*\zerowidth, 0.0)$);
        \coordinate (answer4) at ($(equal4) + (2.5*\zerowidth, 0.0)$);

        % ---------------------------------------------------------------------

        % --- Bounding Box ----------------------------------------------------

        % the distance between the top right corner and the box created for
        % writing the answer in the first row is the same than the distance
        % between the bottom and and the first operand of the last row
        \coordinate (right) at ($(answer4) + (1.5\zerowidth, 0.5\zeroheight + 0.5\baselineskip)$);

        % draw an invisible box used to properly align all sequences
        \draw [white] (bottom) rectangle (right);

        % ---------------------------------------------------------------------

        % --- Table -----------------------------------------------------------

        % numbers in the first column
        \draw (label11) node {\huge 4};
        \draw (operator1) node {\huge$\times$};
        \draw (label12) node {\huge 1};
        \draw (equal1) node {\huge$=$};
        \draw(answer1) node [rounded corners, rectangle, minimum width=3.0*\zerowidth, minimum height = \zeroheight + \baselineskip, draw] {};

        \draw(label21) node [rounded corners, rectangle, minimum width=3.0*\zerowidth, minimum height = \zeroheight + \baselineskip, draw] {};
        \draw (operator2) node {\huge$\times$};
        \draw (label22) node {\huge 1};
        \draw (equal2) node {\huge$=$};
        \draw(answer2) node [rounded corners, rectangle, minimum width=3.0*\zerowidth, minimum height = \zeroheight + \baselineskip, draw] {};

        \draw (label31) node {\huge 2};
        \draw (operator3) node {\huge$\times$};
        \draw(label32) node [rounded corners, rectangle, minimum width=3.0*\zerowidth, minimum height = \zeroheight + \baselineskip, draw] {};
        \draw (equal3) node {\huge$=$};
        \draw(answer3) node [rounded corners, rectangle, minimum width=3.0*\zerowidth, minimum height = \zeroheight + \baselineskip, draw] {};

        \draw(label41) node [rounded corners, rectangle, minimum width=3.0*\zerowidth, minimum height = \zeroheight + \baselineskip, draw] {};
        \draw (operator4) node {\huge$\times$};
        \draw(label42) node [rounded corners, rectangle, minimum width=3.0*\zerowidth, minimum height = \zeroheight + \baselineskip, draw] {};
        \draw (equal4) node {\huge$=$};
        \draw(answer4) node [rounded corners, rectangle, minimum width=3.0*\zerowidth, minimum height = \zeroheight + \baselineskip, draw] {};

        % ---------------------------------------------------------------------

      \end{tikzpicture}
    \end{center}
  \end{minipage}

  \question Primero, se muestran varias multiplicaciones con las siguientes
  características:

  \begin{itemize}

    \item Número maximo de dígitos de la primera columna: 1

    \item Número maximo de dígitos de la segunda columna: 1

    \item Número maximo de dígitos necesarios para representar cada resultado: 2

    \item Número de multiplicaciones a presentar en la tabla: 4

  \end{itemize}

  El esquema general se muestra a continuación:

  \noindent\begin{minipage}{0.33\linewidth}
    \begin{center}
      \begin{tikzpicture}

        % --- Coordinates -----------------------------------------------------

        \coordinate (bottom) at (0,0);
        \fill [blue] (bottom) circle (1pt);

        % el desplazamiento vertical desde el punto inferior izquierdo hasta
        % donde se escribe el primer numero de la última serie incluye una
        % distancia adicional de \baselineskip para facilitar que unas tablas se
        % puedan apilar sobre otras
        \coordinate (label11) at ($(bottom) + (1.5\zerowidth, 0.5\zeroheight+1.5\baselineskip)$);
        \fill [red] (label11) circle (1pt);
        \coordinate (operator1) at ($(label11) + (2.5*\zerowidth, 0.0)$);
        \fill [red] (operator1) circle (1pt);
        \coordinate (label12) at ($(operator1) + (2.5*\zerowidth, 0.0)$);
        \fill [red] (label12) circle (1pt);
        \coordinate (equal1) at ($(label12) + (2.5*\zerowidth, 0.0)$);
        \fill [red] (equal1) circle (1pt);

        % el punto de referencia que marca el punto central de la caja en la que
        % se escribe el resultado se desplaza multiplicando \zerowidth por la
        % mitad del ancho que tendrá la caja más 1. Puesto que en este caso se
        % pueden dibujar hasta dos dígitos en la caja de resultado su ancho es 4
        % y, por lo tanto, el desplazamiento es 4/2+1=3
        \coordinate (answer1) at ($(equal1) + (3.0*\zerowidth, 0.0)$);
        \fill [red] (answer1) circle (1pt);

        \coordinate (label21) at ($(label11) + (0.0, \zeroheight+1.5*\baselineskip)$);
        \fill [red] (label21) circle (1pt);
        \coordinate (operator2) at ($(label21) + (2.5*\zerowidth, 0.0)$);
        \fill [red] (operator2) circle (1pt);
        \coordinate (label22) at ($(operator2) + (2.5*\zerowidth, 0.0)$);
        \fill [red] (label22) circle (1pt);
        \coordinate (equal2) at ($(label22) + (2.5*\zerowidth, 0.0)$);
        \fill [red] (equal2) circle (1pt);
        \coordinate (answer2) at ($(equal2) + (3.0*\zerowidth, 0.0)$);
        \fill [red] (answer2) circle (1pt);

        \coordinate (label31) at ($(label21) + (0.0, \zeroheight+1.5*\baselineskip)$);
        \fill [red] (label31) circle (1pt);
        \coordinate (operator3) at ($(label31) + (2.5*\zerowidth, 0.0)$);
        \fill [red] (operator3) circle (1pt);
        \coordinate (label32) at ($(operator3) + (2.5*\zerowidth, 0.0)$);
        \fill [red] (label32) circle (1pt);
        \coordinate (equal3) at ($(label32) + (2.5*\zerowidth, 0.0)$);
        \fill [red] (equal3) circle (1pt);
        \coordinate (answer3) at ($(equal3) + (3.0*\zerowidth, 0.0)$);
        \fill [red] (answer3) circle (1pt);

        \coordinate (label41) at ($(label31) + (0.0, \zeroheight+1.5*\baselineskip)$);
        \fill [red] (label41) circle (1pt);
        \coordinate (operator4) at ($(label41) + (2.5*\zerowidth, 0.0)$);
        \fill [red] (operator4) circle (1pt);
        \coordinate (label42) at ($(operator4) + (2.5*\zerowidth, 0.0)$);
        \fill [red] (label42) circle (1pt);
        \coordinate (equal4) at ($(label42) + (2.5*\zerowidth, 0.0)$);
        \fill [red] (equal4) circle (1pt);
        \coordinate (answer4) at ($(equal4) + (3.0*\zerowidth, 0.0)$);
        \fill [red] (answer4) circle (1pt);

        % ---------------------------------------------------------------------

        % --- Bounding Box ----------------------------------------------------

        % the distance between the top right corner and the box created for
        % writing the answer in the first row is the same than the distance
        % between the bottom and and the first operand of the last row
        \coordinate (right) at ($(answer4) + (2.0\zerowidth, 0.5\zeroheight + 0.5\baselineskip)$);
        \fill [green] (right) circle (1pt);

        % draw an invisible box used to properly align all sequences
        \draw [lightgray] (bottom) rectangle (right);

        % ---------------------------------------------------------------------

        % --- Table -----------------------------------------------------------

        % numbers in the first column
        \draw (label11) node {\huge 7};
        \draw (operator1) node {\huge$\times$};
        \draw (label12) node {\huge 4};
        \draw (equal1) node {\huge$=$};
        \draw(answer1) node [rounded corners, rectangle, minimum width=4.0*\zerowidth, minimum height = \zeroheight + \baselineskip, draw] {\textcolor{lightgray}{\huge 28}};

        \draw(label21) node [rounded corners, rectangle, minimum width=3.0*\zerowidth, minimum height = \zeroheight + \baselineskip, draw] {\textcolor{lightgray}{\huge 7}};
        \draw (operator2) node {\huge$\times$};
        \draw (label22) node {\huge 3};
        \draw (equal2) node {\huge$=$};
        \draw(answer2) node [rounded corners, rectangle, minimum width=4.0*\zerowidth, minimum height = \zeroheight + \baselineskip, draw] {\textcolor{lightgray}{\huge 21}};

        \draw (label31) node {\huge 7};
        \draw (operator3) node {\huge$\times$};
        \draw(label32) node [rounded corners, rectangle, minimum width=3.0*\zerowidth, minimum height = \zeroheight + \baselineskip, draw] {\textcolor{lightgray}{\huge 2}};
        \draw (equal3) node {\huge$=$};
        \draw(answer3) node [rounded corners, rectangle, minimum width=4.0*\zerowidth, minimum height = \zeroheight + \baselineskip, draw] {\textcolor{lightgray}{\huge 14}};

        \draw(label41) node [rounded corners, rectangle, minimum width=3.0*\zerowidth, minimum height = \zeroheight + \baselineskip, draw] {\textcolor{lightgray}{\huge 7}};
        \draw (operator4) node {\huge$\times$};
        \draw(label42) node [rounded corners, rectangle, minimum width=3.0*\zerowidth, minimum height = \zeroheight + \baselineskip, draw] {\textcolor{lightgray}{\huge 1}};
        \draw (equal4) node {\huge$=$};
        \draw(answer4) node [rounded corners, rectangle, minimum width=4.0*\zerowidth, minimum height = \zeroheight + \baselineskip, draw] {\textcolor{lightgray}{\huge 7}};

        % ---------------------------------------------------------------------

      \end{tikzpicture}
    \end{center}
  \end{minipage}

  A continuación se muestran algunos ejemplos:

  El esquema general se muestra a continuación:

  \noindent\begin{minipage}{0.33\linewidth}
    \begin{center}
      \begin{tikzpicture}

        % --- Coordinates -----------------------------------------------------

        \coordinate (bottom) at (0,0);

        \coordinate (label11) at ($(bottom) + (1.5\zerowidth, 0.5\zeroheight+1.5\baselineskip)$);
        \coordinate (operator1) at ($(label11) + (2.5*\zerowidth, 0.0)$);
        \coordinate (label12) at ($(operator1) + (2.5*\zerowidth, 0.0)$);
        \coordinate (equal1) at ($(label12) + (2.5*\zerowidth, 0.0)$);
        \coordinate (answer1) at ($(equal1) + (3.0*\zerowidth, 0.0)$);

        \coordinate (label21) at ($(label11) + (0.0, \zeroheight+1.5*\baselineskip)$);
        \coordinate (operator2) at ($(label21) + (2.5*\zerowidth, 0.0)$);
        \coordinate (label22) at ($(operator2) + (2.5*\zerowidth, 0.0)$);
        \coordinate (equal2) at ($(label22) + (2.5*\zerowidth, 0.0)$);
        \coordinate (answer2) at ($(equal2) + (3.0*\zerowidth, 0.0)$);

        \coordinate (label31) at ($(label21) + (0.0, \zeroheight+1.5*\baselineskip)$);
        \coordinate (operator3) at ($(label31) + (2.5*\zerowidth, 0.0)$);
        \coordinate (label32) at ($(operator3) + (2.5*\zerowidth, 0.0)$);
        \coordinate (equal3) at ($(label32) + (2.5*\zerowidth, 0.0)$);
        \coordinate (answer3) at ($(equal3) + (3.0*\zerowidth, 0.0)$);

        \coordinate (label41) at ($(label31) + (0.0, \zeroheight+1.5*\baselineskip)$);
        \coordinate (operator4) at ($(label41) + (2.5*\zerowidth, 0.0)$);
        \coordinate (label42) at ($(operator4) + (2.5*\zerowidth, 0.0)$);
        \coordinate (equal4) at ($(label42) + (2.5*\zerowidth, 0.0)$);
        \coordinate (answer4) at ($(equal4) + (3.0*\zerowidth, 0.0)$);

        % ---------------------------------------------------------------------

        % --- Bounding Box ----------------------------------------------------

        % the distance between the top right corner and the box created for
        % writing the answer in the first row is the same than the distance
        % between the bottom and and the first operand of the last row
        \coordinate (right) at ($(answer4) + (2.0\zerowidth, 0.5\zeroheight + 0.5\baselineskip)$);

        % draw an invisible box used to properly align all sequences
        \draw [white] (bottom) rectangle (right);

        % ---------------------------------------------------------------------

        % --- Table -----------------------------------------------------------

        % numbers in the first column
        \draw (label11) node {\huge 7};
        \draw (operator1) node {\huge$\times$};
        \draw (label12) node {\huge 4};
        \draw (equal1) node {\huge$=$};
        \draw(answer1) node [rounded corners, rectangle, minimum width=4.0*\zerowidth, minimum height = \zeroheight + \baselineskip, draw] {};

        \draw(label21) node [rounded corners, rectangle, minimum width=3.0*\zerowidth, minimum height = \zeroheight + \baselineskip, draw] {};
        \draw (operator2) node {\huge$\times$};
        \draw (label22) node {\huge 3};
        \draw (equal2) node {\huge$=$};
        \draw(answer2) node [rounded corners, rectangle, minimum width=4.0*\zerowidth, minimum height = \zeroheight + \baselineskip, draw] {};

        \draw (label31) node {\huge 7};
        \draw (operator3) node {\huge$\times$};
        \draw(label32) node [rounded corners, rectangle, minimum width=3.0*\zerowidth, minimum height = \zeroheight + \baselineskip, draw] {};
        \draw (equal3) node {\huge$=$};
        \draw(answer3) node [rounded corners, rectangle, minimum width=4.0*\zerowidth, minimum height = \zeroheight + \baselineskip, draw] {};

        \draw(label41) node [rounded corners, rectangle, minimum width=3.0*\zerowidth, minimum height = \zeroheight + \baselineskip, draw] {};
        \draw (operator4) node {\huge$\times$};
        \draw(label42) node [rounded corners, rectangle, minimum width=3.0*\zerowidth, minimum height = \zeroheight + \baselineskip, draw] {};
        \draw (equal4) node {\huge$=$};
        \draw(answer4) node [rounded corners, rectangle, minimum width=4.0*\zerowidth, minimum height = \zeroheight + \baselineskip, draw] {};

        % ---------------------------------------------------------------------

      \end{tikzpicture}
    \end{center}
  \end{minipage}
  \begin{minipage}{0.33\linewidth}
    \begin{center}
      \begin{tikzpicture}

        % --- Coordinates -----------------------------------------------------

        \coordinate (bottom) at (0,0);

        \coordinate (label11) at ($(bottom) + (1.5\zerowidth, 0.5\zeroheight+1.5\baselineskip)$);
        \coordinate (operator1) at ($(label11) + (2.5*\zerowidth, 0.0)$);
        \coordinate (label12) at ($(operator1) + (2.5*\zerowidth, 0.0)$);
        \coordinate (equal1) at ($(label12) + (2.5*\zerowidth, 0.0)$);
        \coordinate (answer1) at ($(equal1) + (3.0*\zerowidth, 0.0)$);

        \coordinate (label21) at ($(label11) + (0.0, \zeroheight+1.5*\baselineskip)$);
        \coordinate (operator2) at ($(label21) + (2.5*\zerowidth, 0.0)$);
        \coordinate (label22) at ($(operator2) + (2.5*\zerowidth, 0.0)$);
        \coordinate (equal2) at ($(label22) + (2.5*\zerowidth, 0.0)$);
        \coordinate (answer2) at ($(equal2) + (3.0*\zerowidth, 0.0)$);

        \coordinate (label31) at ($(label21) + (0.0, \zeroheight+1.5*\baselineskip)$);
        \coordinate (operator3) at ($(label31) + (2.5*\zerowidth, 0.0)$);
        \coordinate (label32) at ($(operator3) + (2.5*\zerowidth, 0.0)$);
        \coordinate (equal3) at ($(label32) + (2.5*\zerowidth, 0.0)$);
        \coordinate (answer3) at ($(equal3) + (3.0*\zerowidth, 0.0)$);

        \coordinate (label41) at ($(label31) + (0.0, \zeroheight+1.5*\baselineskip)$);
        \coordinate (operator4) at ($(label41) + (2.5*\zerowidth, 0.0)$);
        \coordinate (label42) at ($(operator4) + (2.5*\zerowidth, 0.0)$);
        \coordinate (equal4) at ($(label42) + (2.5*\zerowidth, 0.0)$);
        \coordinate (answer4) at ($(equal4) + (3.0*\zerowidth, 0.0)$);

        % ---------------------------------------------------------------------

        % --- Bounding Box ----------------------------------------------------

        % the distance between the top right corner and the box created for
        % writing the answer in the first row is the same than the distance
        % between the bottom and and the first operand of the last row
        \coordinate (right) at ($(answer4) + (2.0\zerowidth, 0.5\zeroheight + 0.5\baselineskip)$);

        % draw an invisible box used to properly align all sequences
        \draw [white] (bottom) rectangle (right);

        % ---------------------------------------------------------------------

        % --- Table -----------------------------------------------------------

        % numbers in the first column
        \draw (label11) node {\huge 5};
        \draw (operator1) node {\huge$\times$};
        \draw (label12) node {\huge 4};
        \draw (equal1) node {\huge$=$};
        \draw(answer1) node [rounded corners, rectangle, minimum width=4.0*\zerowidth, minimum height = \zeroheight + \baselineskip, draw] {};

        \draw(label21) node [rounded corners, rectangle, minimum width=3.0*\zerowidth, minimum height = \zeroheight + \baselineskip, draw] {};
        \draw (operator2) node {\huge$\times$};
        \draw (label22) node {\huge 3};
        \draw (equal2) node {\huge$=$};
        \draw(answer2) node [rounded corners, rectangle, minimum width=4.0*\zerowidth, minimum height = \zeroheight + \baselineskip, draw] {};

        \draw (label31) node {\huge 5};
        \draw (operator3) node {\huge$\times$};
        \draw(label32) node [rounded corners, rectangle, minimum width=3.0*\zerowidth, minimum height = \zeroheight + \baselineskip, draw] {};
        \draw (equal3) node {\huge$=$};
        \draw(answer3) node [rounded corners, rectangle, minimum width=4.0*\zerowidth, minimum height = \zeroheight + \baselineskip, draw] {};

        \draw(label41) node [rounded corners, rectangle, minimum width=3.0*\zerowidth, minimum height = \zeroheight + \baselineskip, draw] {};
        \draw (operator4) node {\huge$\times$};
        \draw(label42) node [rounded corners, rectangle, minimum width=3.0*\zerowidth, minimum height = \zeroheight + \baselineskip, draw] {};
        \draw (equal4) node {\huge$=$};
        \draw(answer4) node [rounded corners, rectangle, minimum width=4.0*\zerowidth, minimum height = \zeroheight + \baselineskip, draw] {};

        % ---------------------------------------------------------------------

      \end{tikzpicture}
    \end{center}
  \end{minipage}
  \begin{minipage}{0.33\linewidth}
    \begin{center}
      \begin{tikzpicture}

        % --- Coordinates -----------------------------------------------------

        \coordinate (bottom) at (0,0);

        \coordinate (label11) at ($(bottom) + (1.5\zerowidth, 0.5\zeroheight+1.5\baselineskip)$);
        \coordinate (operator1) at ($(label11) + (2.5*\zerowidth, 0.0)$);
        \coordinate (label12) at ($(operator1) + (2.5*\zerowidth, 0.0)$);
        \coordinate (equal1) at ($(label12) + (2.5*\zerowidth, 0.0)$);
        \coordinate (answer1) at ($(equal1) + (3.0*\zerowidth, 0.0)$);

        \coordinate (label21) at ($(label11) + (0.0, \zeroheight+1.5*\baselineskip)$);
        \coordinate (operator2) at ($(label21) + (2.5*\zerowidth, 0.0)$);
        \coordinate (label22) at ($(operator2) + (2.5*\zerowidth, 0.0)$);
        \coordinate (equal2) at ($(label22) + (2.5*\zerowidth, 0.0)$);
        \coordinate (answer2) at ($(equal2) + (3.0*\zerowidth, 0.0)$);

        \coordinate (label31) at ($(label21) + (0.0, \zeroheight+1.5*\baselineskip)$);
        \coordinate (operator3) at ($(label31) + (2.5*\zerowidth, 0.0)$);
        \coordinate (label32) at ($(operator3) + (2.5*\zerowidth, 0.0)$);
        \coordinate (equal3) at ($(label32) + (2.5*\zerowidth, 0.0)$);
        \coordinate (answer3) at ($(equal3) + (3.0*\zerowidth, 0.0)$);

        \coordinate (label41) at ($(label31) + (0.0, \zeroheight+1.5*\baselineskip)$);
        \coordinate (operator4) at ($(label41) + (2.5*\zerowidth, 0.0)$);
        \coordinate (label42) at ($(operator4) + (2.5*\zerowidth, 0.0)$);
        \coordinate (equal4) at ($(label42) + (2.5*\zerowidth, 0.0)$);
        \coordinate (answer4) at ($(equal4) + (3.0*\zerowidth, 0.0)$);

        % ---------------------------------------------------------------------

        % --- Bounding Box ----------------------------------------------------

        % the distance between the top right corner and the box created for
        % writing the answer in the first row is the same than the distance
        % between the bottom and and the first operand of the last row
        \coordinate (right) at ($(answer4) + (2.0\zerowidth, 0.5\zeroheight + 0.5\baselineskip)$);

        % draw an invisible box used to properly align all sequences
        \draw [white] (bottom) rectangle (right);

        % ---------------------------------------------------------------------

        % --- Table -----------------------------------------------------------

        % numbers in the first column
        \draw (label11) node {\huge 9};
        \draw (operator1) node {\huge$\times$};
        \draw (label12) node {\huge 4};
        \draw (equal1) node {\huge$=$};
        \draw(answer1) node [rounded corners, rectangle, minimum width=4.0*\zerowidth, minimum height = \zeroheight + \baselineskip, draw] {};

        \draw(label21) node [rounded corners, rectangle, minimum width=3.0*\zerowidth, minimum height = \zeroheight + \baselineskip, draw] {};
        \draw (operator2) node {\huge$\times$};
        \draw (label22) node {\huge 3};
        \draw (equal2) node {\huge$=$};
        \draw(answer2) node [rounded corners, rectangle, minimum width=4.0*\zerowidth, minimum height = \zeroheight + \baselineskip, draw] {};

        \draw (label31) node {\huge 9};
        \draw (operator3) node {\huge$\times$};
        \draw(label32) node [rounded corners, rectangle, minimum width=3.0*\zerowidth, minimum height = \zeroheight + \baselineskip, draw] {};
        \draw (equal3) node {\huge$=$};
        \draw(answer3) node [rounded corners, rectangle, minimum width=4.0*\zerowidth, minimum height = \zeroheight + \baselineskip, draw] {};

        \draw(label41) node [rounded corners, rectangle, minimum width=3.0*\zerowidth, minimum height = \zeroheight + \baselineskip, draw] {};
        \draw (operator4) node {\huge$\times$};
        \draw(label42) node [rounded corners, rectangle, minimum width=3.0*\zerowidth, minimum height = \zeroheight + \baselineskip, draw] {};
        \draw (equal4) node {\huge$=$};
        \draw(answer4) node [rounded corners, rectangle, minimum width=4.0*\zerowidth, minimum height = \zeroheight + \baselineskip, draw] {};

        % ---------------------------------------------------------------------

      \end{tikzpicture}
    \end{center}
  \end{minipage}
  \begin{minipage}{0.33\linewidth}
    \begin{center}
      \begin{tikzpicture}

        % --- Coordinates -----------------------------------------------------

        \coordinate (bottom) at (0,0);

        \coordinate (label11) at ($(bottom) + (1.5\zerowidth, 0.5\zeroheight+1.5\baselineskip)$);
        \coordinate (operator1) at ($(label11) + (2.5*\zerowidth, 0.0)$);
        \coordinate (label12) at ($(operator1) + (2.5*\zerowidth, 0.0)$);
        \coordinate (equal1) at ($(label12) + (2.5*\zerowidth, 0.0)$);
        \coordinate (answer1) at ($(equal1) + (3.0*\zerowidth, 0.0)$);

        \coordinate (label21) at ($(label11) + (0.0, \zeroheight+1.5*\baselineskip)$);
        \coordinate (operator2) at ($(label21) + (2.5*\zerowidth, 0.0)$);
        \coordinate (label22) at ($(operator2) + (2.5*\zerowidth, 0.0)$);
        \coordinate (equal2) at ($(label22) + (2.5*\zerowidth, 0.0)$);
        \coordinate (answer2) at ($(equal2) + (3.0*\zerowidth, 0.0)$);

        \coordinate (label31) at ($(label21) + (0.0, \zeroheight+1.5*\baselineskip)$);
        \coordinate (operator3) at ($(label31) + (2.5*\zerowidth, 0.0)$);
        \coordinate (label32) at ($(operator3) + (2.5*\zerowidth, 0.0)$);
        \coordinate (equal3) at ($(label32) + (2.5*\zerowidth, 0.0)$);
        \coordinate (answer3) at ($(equal3) + (3.0*\zerowidth, 0.0)$);

        \coordinate (label41) at ($(label31) + (0.0, \zeroheight+1.5*\baselineskip)$);
        \coordinate (operator4) at ($(label41) + (2.5*\zerowidth, 0.0)$);
        \coordinate (label42) at ($(operator4) + (2.5*\zerowidth, 0.0)$);
        \coordinate (equal4) at ($(label42) + (2.5*\zerowidth, 0.0)$);
        \coordinate (answer4) at ($(equal4) + (3.0*\zerowidth, 0.0)$);

        % ---------------------------------------------------------------------

        % --- Bounding Box ----------------------------------------------------

        % the distance between the top right corner and the box created for
        % writing the answer in the first row is the same than the distance
        % between the bottom and and the first operand of the last row
        \coordinate (right) at ($(answer4) + (2.0\zerowidth, 0.5\zeroheight + 0.5\baselineskip)$);

        % draw an invisible box used to properly align all sequences
        \draw [white] (bottom) rectangle (right);

        % ---------------------------------------------------------------------

        % --- Table -----------------------------------------------------------

        % numbers in the first column
        \draw (label11) node {\huge 4};
        \draw (operator1) node {\huge$\times$};
        \draw (label12) node {\huge 4};
        \draw (equal1) node {\huge$=$};
        \draw(answer1) node [rounded corners, rectangle, minimum width=4.0*\zerowidth, minimum height = \zeroheight + \baselineskip, draw] {};

        \draw(label21) node [rounded corners, rectangle, minimum width=3.0*\zerowidth, minimum height = \zeroheight + \baselineskip, draw] {};
        \draw (operator2) node {\huge$\times$};
        \draw (label22) node {\huge 3};
        \draw (equal2) node {\huge$=$};
        \draw(answer2) node [rounded corners, rectangle, minimum width=4.0*\zerowidth, minimum height = \zeroheight + \baselineskip, draw] {};

        \draw (label31) node {\huge 4};
        \draw (operator3) node {\huge$\times$};
        \draw(label32) node [rounded corners, rectangle, minimum width=3.0*\zerowidth, minimum height = \zeroheight + \baselineskip, draw] {};
        \draw (equal3) node {\huge$=$};
        \draw(answer3) node [rounded corners, rectangle, minimum width=4.0*\zerowidth, minimum height = \zeroheight + \baselineskip, draw] {};

        \draw(label41) node [rounded corners, rectangle, minimum width=3.0*\zerowidth, minimum height = \zeroheight + \baselineskip, draw] {};
        \draw (operator4) node {\huge$\times$};
        \draw(label42) node [rounded corners, rectangle, minimum width=3.0*\zerowidth, minimum height = \zeroheight + \baselineskip, draw] {};
        \draw (equal4) node {\huge$=$};
        \draw(answer4) node [rounded corners, rectangle, minimum width=4.0*\zerowidth, minimum height = \zeroheight + \baselineskip, draw] {};

        % ---------------------------------------------------------------------

      \end{tikzpicture}
    \end{center}
  \end{minipage}
  \begin{minipage}{0.33\linewidth}
    \begin{center}
      \begin{tikzpicture}

        % --- Coordinates -----------------------------------------------------

        \coordinate (bottom) at (0,0);

        \coordinate (label11) at ($(bottom) + (1.5\zerowidth, 0.5\zeroheight+1.5\baselineskip)$);
        \coordinate (operator1) at ($(label11) + (2.5*\zerowidth, 0.0)$);
        \coordinate (label12) at ($(operator1) + (2.5*\zerowidth, 0.0)$);
        \coordinate (equal1) at ($(label12) + (2.5*\zerowidth, 0.0)$);
        \coordinate (answer1) at ($(equal1) + (3.0*\zerowidth, 0.0)$);

        \coordinate (label21) at ($(label11) + (0.0, \zeroheight+1.5*\baselineskip)$);
        \coordinate (operator2) at ($(label21) + (2.5*\zerowidth, 0.0)$);
        \coordinate (label22) at ($(operator2) + (2.5*\zerowidth, 0.0)$);
        \coordinate (equal2) at ($(label22) + (2.5*\zerowidth, 0.0)$);
        \coordinate (answer2) at ($(equal2) + (3.0*\zerowidth, 0.0)$);

        \coordinate (label31) at ($(label21) + (0.0, \zeroheight+1.5*\baselineskip)$);
        \coordinate (operator3) at ($(label31) + (2.5*\zerowidth, 0.0)$);
        \coordinate (label32) at ($(operator3) + (2.5*\zerowidth, 0.0)$);
        \coordinate (equal3) at ($(label32) + (2.5*\zerowidth, 0.0)$);
        \coordinate (answer3) at ($(equal3) + (3.0*\zerowidth, 0.0)$);

        \coordinate (label41) at ($(label31) + (0.0, \zeroheight+1.5*\baselineskip)$);
        \coordinate (operator4) at ($(label41) + (2.5*\zerowidth, 0.0)$);
        \coordinate (label42) at ($(operator4) + (2.5*\zerowidth, 0.0)$);
        \coordinate (equal4) at ($(label42) + (2.5*\zerowidth, 0.0)$);
        \coordinate (answer4) at ($(equal4) + (3.0*\zerowidth, 0.0)$);

        % ---------------------------------------------------------------------

        % --- Bounding Box ----------------------------------------------------

        % the distance between the top right corner and the box created for
        % writing the answer in the first row is the same than the distance
        % between the bottom and and the first operand of the last row
        \coordinate (right) at ($(answer4) + (2.0\zerowidth, 0.5\zeroheight + 0.5\baselineskip)$);

        % draw an invisible box used to properly align all sequences
        \draw [white] (bottom) rectangle (right);

        % ---------------------------------------------------------------------

        % --- Table -----------------------------------------------------------

        % numbers in the first column
        \draw (label11) node {\huge 6};
        \draw (operator1) node {\huge$\times$};
        \draw (label12) node {\huge 6};
        \draw (equal1) node {\huge$=$};
        \draw(answer1) node [rounded corners, rectangle, minimum width=4.0*\zerowidth, minimum height = \zeroheight + \baselineskip, draw] {};

        \draw(label21) node [rounded corners, rectangle, minimum width=3.0*\zerowidth, minimum height = \zeroheight + \baselineskip, draw] {};
        \draw (operator2) node {\huge$\times$};
        \draw (label22) node {\huge 3};
        \draw (equal2) node {\huge$=$};
        \draw(answer2) node [rounded corners, rectangle, minimum width=4.0*\zerowidth, minimum height = \zeroheight + \baselineskip, draw] {};

        \draw (label31) node {\huge 6};
        \draw (operator3) node {\huge$\times$};
        \draw(label32) node [rounded corners, rectangle, minimum width=3.0*\zerowidth, minimum height = \zeroheight + \baselineskip, draw] {};
        \draw (equal3) node {\huge$=$};
        \draw(answer3) node [rounded corners, rectangle, minimum width=4.0*\zerowidth, minimum height = \zeroheight + \baselineskip, draw] {};

        \draw(label41) node [rounded corners, rectangle, minimum width=3.0*\zerowidth, minimum height = \zeroheight + \baselineskip, draw] {};
        \draw (operator4) node {\huge$\times$};
        \draw(label42) node [rounded corners, rectangle, minimum width=3.0*\zerowidth, minimum height = \zeroheight + \baselineskip, draw] {};
        \draw (equal4) node {\huge$=$};
        \draw(answer4) node [rounded corners, rectangle, minimum width=4.0*\zerowidth, minimum height = \zeroheight + \baselineskip, draw] {};

        % ---------------------------------------------------------------------

      \end{tikzpicture}
    \end{center}
  \end{minipage}
  \begin{minipage}{0.33\linewidth}
    \begin{center}
      \begin{tikzpicture}

        % --- Coordinates -----------------------------------------------------

        \coordinate (bottom) at (0,0);

        \coordinate (label11) at ($(bottom) + (1.5\zerowidth, 0.5\zeroheight+1.5\baselineskip)$);
        \coordinate (operator1) at ($(label11) + (2.5*\zerowidth, 0.0)$);
        \coordinate (label12) at ($(operator1) + (2.5*\zerowidth, 0.0)$);
        \coordinate (equal1) at ($(label12) + (2.5*\zerowidth, 0.0)$);
        \coordinate (answer1) at ($(equal1) + (3.0*\zerowidth, 0.0)$);

        \coordinate (label21) at ($(label11) + (0.0, \zeroheight+1.5*\baselineskip)$);
        \coordinate (operator2) at ($(label21) + (2.5*\zerowidth, 0.0)$);
        \coordinate (label22) at ($(operator2) + (2.5*\zerowidth, 0.0)$);
        \coordinate (equal2) at ($(label22) + (2.5*\zerowidth, 0.0)$);
        \coordinate (answer2) at ($(equal2) + (3.0*\zerowidth, 0.0)$);

        \coordinate (label31) at ($(label21) + (0.0, \zeroheight+1.5*\baselineskip)$);
        \coordinate (operator3) at ($(label31) + (2.5*\zerowidth, 0.0)$);
        \coordinate (label32) at ($(operator3) + (2.5*\zerowidth, 0.0)$);
        \coordinate (equal3) at ($(label32) + (2.5*\zerowidth, 0.0)$);
        \coordinate (answer3) at ($(equal3) + (3.0*\zerowidth, 0.0)$);

        \coordinate (label41) at ($(label31) + (0.0, \zeroheight+1.5*\baselineskip)$);
        \coordinate (operator4) at ($(label41) + (2.5*\zerowidth, 0.0)$);
        \coordinate (label42) at ($(operator4) + (2.5*\zerowidth, 0.0)$);
        \coordinate (equal4) at ($(label42) + (2.5*\zerowidth, 0.0)$);
        \coordinate (answer4) at ($(equal4) + (3.0*\zerowidth, 0.0)$);

        % ---------------------------------------------------------------------

        % --- Bounding Box ----------------------------------------------------

        % the distance between the top right corner and the box created for
        % writing the answer in the first row is the same than the distance
        % between the bottom and and the first operand of the last row
        \coordinate (right) at ($(answer4) + (2.0\zerowidth, 0.5\zeroheight + 0.5\baselineskip)$);

        % draw an invisible box used to properly align all sequences
        \draw [white] (bottom) rectangle (right);

        % ---------------------------------------------------------------------

        % --- Table -----------------------------------------------------------

        % numbers in the first column
        \draw (label11) node {\huge 3};
        \draw (operator1) node {\huge$\times$};
        \draw (label12) node {\huge 3};
        \draw (equal1) node {\huge$=$};
        \draw(answer1) node [rounded corners, rectangle, minimum width=4.0*\zerowidth, minimum height = \zeroheight + \baselineskip, draw] {};

        \draw(label21) node [rounded corners, rectangle, minimum width=3.0*\zerowidth, minimum height = \zeroheight + \baselineskip, draw] {};
        \draw (operator2) node {\huge$\times$};
        \draw (label22) node {\huge 3};
        \draw (equal2) node {\huge$=$};
        \draw(answer2) node [rounded corners, rectangle, minimum width=4.0*\zerowidth, minimum height = \zeroheight + \baselineskip, draw] {};

        \draw (label31) node {\huge 3};
        \draw (operator3) node {\huge$\times$};
        \draw(label32) node [rounded corners, rectangle, minimum width=3.0*\zerowidth, minimum height = \zeroheight + \baselineskip, draw] {};
        \draw (equal3) node {\huge$=$};
        \draw(answer3) node [rounded corners, rectangle, minimum width=4.0*\zerowidth, minimum height = \zeroheight + \baselineskip, draw] {};

        \draw(label41) node [rounded corners, rectangle, minimum width=3.0*\zerowidth, minimum height = \zeroheight + \baselineskip, draw] {};
        \draw (operator4) node {\huge$\times$};
        \draw(label42) node [rounded corners, rectangle, minimum width=3.0*\zerowidth, minimum height = \zeroheight + \baselineskip, draw] {};
        \draw (equal4) node {\huge$=$};
        \draw(answer4) node [rounded corners, rectangle, minimum width=4.0*\zerowidth, minimum height = \zeroheight + \baselineskip, draw] {};

        % ---------------------------------------------------------------------

      \end{tikzpicture}
    \end{center}
  \end{minipage}



\end{questions}

\end{document}

%%% Local Variables:
%%% mode: latex
%%% TeX-master: t
%%% End:
