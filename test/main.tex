

\documentclass[svgnames,addpoints]{exam}



\usepackage[T1]{fontenc}
\usepackage[utf8]{inputenc}
\usepackage[spanish]{babel}

\usepackage{examen}

\usepackage{amsfonts}
\usepackage{amssymb}
\usepackage{mathtools}

\usepackage{pifont}

\usepackage{cancel}
\usepackage{array}

\usepackage{tikz}
\usepackage{pgflibraryarrows}
\usepackage{pgflibrarysnakes}

\usetikzlibrary{matrix,patterns,fadings,positioning}

\usepackage{array}
\usepackage{eurosym}

\usepackage{booktabs}
\usepackage{url}

\usepackage{rotating}



\begin{document}

\titulacion{Grado en Informática}
\asignatura{Heurística y Optimización}

\convocatoria{\today}
\tiempo{4 horas}

\principio



\begin{questions}

\question Adriana, espero que disfrutes haciendo estas
operaciones. Irene y Jorge dicen que eres capaz de hacerlas, pero no
te desanimes si alguna te cuesta, o si te equivocas.

\begin{parts}

  \part[3] ?`Serías capaz de resolver las siguientes sumas?
  
\begin{minipage}{0.25\linewidth}  
  \begin{center}
    \begin{tikzpicture}

      % draw an invisible box used to properly align all sequences
      \draw [white] (0,0) rectangle (3.5,4.0);

      % show the operands
      \node (label1) at (2.5, 3.5) {};
      \node [left=0.0 cm of label1] (num1) {\huge 1};
      \node (label2) at (2.5, 2.5) {};
      \node [left=0.0 cm of label2] (num2) {\huge 4};

      % show a straight line
      \draw [thick] (0.5, 2.0) -- (3.0, 2.0);

      % show the operator
      \node (op) at (1, 2.5) {\huge$+$};

      % show the result
      \node (label3) at (2.75, 1) {};
      \node [rounded corners, rectangle, minimum width=1.5 cm, minimum height = 1.25 cm, draw, left=0.0 cm of label3] {};

    \end{tikzpicture}
  \end{center}
\end{minipage}
\begin{minipage}{0.25\linewidth}  
  \begin{center}
    \begin{tikzpicture}

      % draw an invisible box used to properly align all sequences
      \draw [white] (0,0) rectangle (3.5,4.0);

      % show the operands
      \node (label1) at (2.5, 3.5) {};
      \node [left=0.0 cm of label1] (num1) {\huge 8};
      \node (label2) at (2.5, 2.5) {};
      \node [left=0.0 cm of label2] (num2) {\huge 4};

      % show a straight line
      \draw [thick] (0.5, 2.0) -- (3.0, 2.0);

      % show the operator
      \node (op) at (1, 2.5) {\huge$+$};

      % show the result
      \node (label3) at (2.75, 1) {};
      \node [rounded corners, rectangle, minimum width=1.5 cm, minimum height = 1.25 cm, draw, left=0.0 cm of label3] {};

    \end{tikzpicture}
  \end{center}
\end{minipage}
\begin{minipage}{0.25\linewidth}  
  \begin{center}
    \begin{tikzpicture}

      % draw an invisible box used to properly align all sequences
      \draw [white] (0,0) rectangle (3.5,4.0);

      % show the operands
      \node (label1) at (2.5, 3.5) {};
      \node [left=0.0 cm of label1] (num1) {\huge 4};
      \node (label2) at (2.5, 2.5) {};
      \node [left=0.0 cm of label2] (num2) {\huge 8};

      % show a straight line
      \draw [thick] (0.5, 2.0) -- (3.0, 2.0);

      % show the operator
      \node (op) at (1, 2.5) {\huge$+$};

      % show the result
      \node (label3) at (2.75, 1) {};
      \node [rounded corners, rectangle, minimum width=1.5 cm, minimum height = 1.25 cm, draw, left=0.0 cm of label3] {};

    \end{tikzpicture}
  \end{center}
\end{minipage}
\begin{minipage}{0.25\linewidth}  
  \begin{center}
    \begin{tikzpicture}

      % draw an invisible box used to properly align all sequences
      \draw [white] (0,0) rectangle (3.5,4.0);

      % show the operands
      \node (label1) at (2.5, 3.5) {};
      \node [left=0.0 cm of label1] (num1) {\huge 7};
      \node (label2) at (2.5, 2.5) {};
      \node [left=0.0 cm of label2] (num2) {\huge 8};

      % show a straight line
      \draw [thick] (0.5, 2.0) -- (3.0, 2.0);

      % show the operator
      \node (op) at (1, 2.5) {\huge$+$};

      % show the result
      \node (label3) at (2.75, 1) {};
      \node [rounded corners, rectangle, minimum width=1.5 cm, minimum height = 1.25 cm, draw, left=0.0 cm of label3] {};

    \end{tikzpicture}
  \end{center}
\end{minipage}
\begin{minipage}{0.25\linewidth}  
  \begin{center}
    \begin{tikzpicture}

      % draw an invisible box used to properly align all sequences
      \draw [white] (0,0) rectangle (3.5,4.0);

      % show the operands
      \node (label1) at (2.5, 3.5) {};
      \node [left=0.0 cm of label1] (num1) {\huge 9};
      \node (label2) at (2.5, 2.5) {};
      \node [left=0.0 cm of label2] (num2) {\huge 6};

      % show a straight line
      \draw [thick] (0.5, 2.0) -- (3.0, 2.0);

      % show the operator
      \node (op) at (1, 2.5) {\huge$+$};

      % show the result
      \node (label3) at (2.75, 1) {};
      \node [rounded corners, rectangle, minimum width=1.5 cm, minimum height = 1.25 cm, draw, left=0.0 cm of label3] {};

    \end{tikzpicture}
  \end{center}
\end{minipage}
\begin{minipage}{0.25\linewidth}  
  \begin{center}
    \begin{tikzpicture}

      % draw an invisible box used to properly align all sequences
      \draw [white] (0,0) rectangle (3.5,4.0);

      % show the operands
      \node (label1) at (2.5, 3.5) {};
      \node [left=0.0 cm of label1] (num1) {\huge 4};
      \node (label2) at (2.5, 2.5) {};
      \node [left=0.0 cm of label2] (num2) {\huge 7};

      % show a straight line
      \draw [thick] (0.5, 2.0) -- (3.0, 2.0);

      % show the operator
      \node (op) at (1, 2.5) {\huge$+$};

      % show the result
      \node (label3) at (2.75, 1) {};
      \node [rounded corners, rectangle, minimum width=1.5 cm, minimum height = 1.25 cm, draw, left=0.0 cm of label3] {};

    \end{tikzpicture}
  \end{center}
\end{minipage}
\begin{minipage}{0.25\linewidth}  
  \begin{center}
    \begin{tikzpicture}

      % draw an invisible box used to properly align all sequences
      \draw [white] (0,0) rectangle (3.5,4.0);

      % show the operands
      \node (label1) at (2.5, 3.5) {};
      \node [left=0.0 cm of label1] (num1) {\huge 7};
      \node (label2) at (2.5, 2.5) {};
      \node [left=0.0 cm of label2] (num2) {\huge 2};

      % show a straight line
      \draw [thick] (0.5, 2.0) -- (3.0, 2.0);

      % show the operator
      \node (op) at (1, 2.5) {\huge$+$};

      % show the result
      \node (label3) at (2.75, 1) {};
      \node [rounded corners, rectangle, minimum width=1.5 cm, minimum height = 1.25 cm, draw, left=0.0 cm of label3] {};

    \end{tikzpicture}
  \end{center}
\end{minipage}
\begin{minipage}{0.25\linewidth}  
  \begin{center}
    \begin{tikzpicture}

      % draw an invisible box used to properly align all sequences
      \draw [white] (0,0) rectangle (3.5,4.0);

      % show the operands
      \node (label1) at (2.5, 3.5) {};
      \node [left=0.0 cm of label1] (num1) {\huge 1};
      \node (label2) at (2.5, 2.5) {};
      \node [left=0.0 cm of label2] (num2) {\huge 5};

      % show a straight line
      \draw [thick] (0.5, 2.0) -- (3.0, 2.0);

      % show the operator
      \node (op) at (1, 2.5) {\huge$+$};

      % show the result
      \node (label3) at (2.75, 1) {};
      \node [rounded corners, rectangle, minimum width=1.5 cm, minimum height = 1.25 cm, draw, left=0.0 cm of label3] {};

    \end{tikzpicture}
  \end{center}
\end{minipage}
\begin{minipage}{0.25\linewidth}  
  \begin{center}
    \begin{tikzpicture}

      % draw an invisible box used to properly align all sequences
      \draw [white] (0,0) rectangle (3.5,4.0);

      % show the operands
      \node (label1) at (2.5, 3.5) {};
      \node [left=0.0 cm of label1] (num1) {\huge 1};
      \node (label2) at (2.5, 2.5) {};
      \node [left=0.0 cm of label2] (num2) {\huge 2};

      % show a straight line
      \draw [thick] (0.5, 2.0) -- (3.0, 2.0);

      % show the operator
      \node (op) at (1, 2.5) {\huge$+$};

      % show the result
      \node (label3) at (2.75, 1) {};
      \node [rounded corners, rectangle, minimum width=1.5 cm, minimum height = 1.25 cm, draw, left=0.0 cm of label3] {};

    \end{tikzpicture}
  \end{center}
\end{minipage}
\begin{minipage}{0.25\linewidth}  
  \begin{center}
    \begin{tikzpicture}

      % draw an invisible box used to properly align all sequences
      \draw [white] (0,0) rectangle (3.5,4.0);

      % show the operands
      \node (label1) at (2.5, 3.5) {};
      \node [left=0.0 cm of label1] (num1) {\huge 3};
      \node (label2) at (2.5, 2.5) {};
      \node [left=0.0 cm of label2] (num2) {\huge 5};

      % show a straight line
      \draw [thick] (0.5, 2.0) -- (3.0, 2.0);

      % show the operator
      \node (op) at (1, 2.5) {\huge$+$};

      % show the result
      \node (label3) at (2.75, 1) {};
      \node [rounded corners, rectangle, minimum width=1.5 cm, minimum height = 1.25 cm, draw, left=0.0 cm of label3] {};

    \end{tikzpicture}
  \end{center}
\end{minipage}
\begin{minipage}{0.25\linewidth}  
  \begin{center}
    \begin{tikzpicture}

      % draw an invisible box used to properly align all sequences
      \draw [white] (0,0) rectangle (3.5,4.0);

      % show the operands
      \node (label1) at (2.5, 3.5) {};
      \node [left=0.0 cm of label1] (num1) {\huge 5};
      \node (label2) at (2.5, 2.5) {};
      \node [left=0.0 cm of label2] (num2) {\huge 5};

      % show a straight line
      \draw [thick] (0.5, 2.0) -- (3.0, 2.0);

      % show the operator
      \node (op) at (1, 2.5) {\huge$+$};

      % show the result
      \node (label3) at (2.75, 1) {};
      \node [rounded corners, rectangle, minimum width=1.5 cm, minimum height = 1.25 cm, draw, left=0.0 cm of label3] {};

    \end{tikzpicture}
  \end{center}
\end{minipage}
\begin{minipage}{0.25\linewidth}  
  \begin{center}
    \begin{tikzpicture}

      % draw an invisible box used to properly align all sequences
      \draw [white] (0,0) rectangle (3.5,4.0);

      % show the operands
      \node (label1) at (2.5, 3.5) {};
      \node [left=0.0 cm of label1] (num1) {\huge 6};
      \node (label2) at (2.5, 2.5) {};
      \node [left=0.0 cm of label2] (num2) {\huge 8};

      % show a straight line
      \draw [thick] (0.5, 2.0) -- (3.0, 2.0);

      % show the operator
      \node (op) at (1, 2.5) {\huge$+$};

      % show the result
      \node (label3) at (2.75, 1) {};
      \node [rounded corners, rectangle, minimum width=1.5 cm, minimum height = 1.25 cm, draw, left=0.0 cm of label3] {};

    \end{tikzpicture}
  \end{center}
\end{minipage}


  \part[3] Hey, ya has acabado las del primer apartado. Un poquito más
  difíciles, a ver qué tal se te dan. Si lo consigues, se las
  enseñamos a Darío, a ver si el también quiere intentar hacerlas.
  
\begin{minipage}{0.25\linewidth}  
  \begin{center}
    \begin{tikzpicture}

      % draw an invisible box used to properly align all sequences
      \draw [white] (0,0) rectangle (3.5,4.0);

      % show the operands
      \node (label1) at (2.5, 3.5) {};
      \node [left=0.0 cm of label1] (num1) {\huge 11};
      \node (label2) at (2.5, 2.5) {};
      \node [left=0.0 cm of label2] (num2) {\huge 3};

      % show a straight line
      \draw [thick] (0.5, 2.0) -- (3.0, 2.0);

      % show the operator
      \node (op) at (1, 2.5) {\huge$+$};

      % show the result
      \node (label3) at (2.75, 1) {};
      \node [rounded corners, rectangle, minimum width=1.5 cm, minimum height = 1.25 cm, draw, left=0.0 cm of label3] {};

    \end{tikzpicture}
  \end{center}
\end{minipage}
\begin{minipage}{0.25\linewidth}  
  \begin{center}
    \begin{tikzpicture}

      % draw an invisible box used to properly align all sequences
      \draw [white] (0,0) rectangle (3.5,4.0);

      % show the operands
      \node (label1) at (2.5, 3.5) {};
      \node [left=0.0 cm of label1] (num1) {\huge 11};
      \node (label2) at (2.5, 2.5) {};
      \node [left=0.0 cm of label2] (num2) {\huge 5};

      % show a straight line
      \draw [thick] (0.5, 2.0) -- (3.0, 2.0);

      % show the operator
      \node (op) at (1, 2.5) {\huge$+$};

      % show the result
      \node (label3) at (2.75, 1) {};
      \node [rounded corners, rectangle, minimum width=1.5 cm, minimum height = 1.25 cm, draw, left=0.0 cm of label3] {};

    \end{tikzpicture}
  \end{center}
\end{minipage}
\begin{minipage}{0.25\linewidth}  
  \begin{center}
    \begin{tikzpicture}

      % draw an invisible box used to properly align all sequences
      \draw [white] (0,0) rectangle (3.5,4.0);

      % show the operands
      \node (label1) at (2.5, 3.5) {};
      \node [left=0.0 cm of label1] (num1) {\huge 12};
      \node (label2) at (2.5, 2.5) {};
      \node [left=0.0 cm of label2] (num2) {\huge 4};

      % show a straight line
      \draw [thick] (0.5, 2.0) -- (3.0, 2.0);

      % show the operator
      \node (op) at (1, 2.5) {\huge$+$};

      % show the result
      \node (label3) at (2.75, 1) {};
      \node [rounded corners, rectangle, minimum width=1.5 cm, minimum height = 1.25 cm, draw, left=0.0 cm of label3] {};

    \end{tikzpicture}
  \end{center}
\end{minipage}
\begin{minipage}{0.25\linewidth}  
  \begin{center}
    \begin{tikzpicture}

      % draw an invisible box used to properly align all sequences
      \draw [white] (0,0) rectangle (3.5,4.0);

      % show the operands
      \node (label1) at (2.5, 3.5) {};
      \node [left=0.0 cm of label1] (num1) {\huge 10};
      \node (label2) at (2.5, 2.5) {};
      \node [left=0.0 cm of label2] (num2) {\huge 9};

      % show a straight line
      \draw [thick] (0.5, 2.0) -- (3.0, 2.0);

      % show the operator
      \node (op) at (1, 2.5) {\huge$+$};

      % show the result
      \node (label3) at (2.75, 1) {};
      \node [rounded corners, rectangle, minimum width=1.5 cm, minimum height = 1.25 cm, draw, left=0.0 cm of label3] {};

    \end{tikzpicture}
  \end{center}
\end{minipage}
\begin{minipage}{0.25\linewidth}  
  \begin{center}
    \begin{tikzpicture}

      % draw an invisible box used to properly align all sequences
      \draw [white] (0,0) rectangle (3.5,4.0);

      % show the operands
      \node (label1) at (2.5, 3.5) {};
      \node [left=0.0 cm of label1] (num1) {\huge 13};
      \node (label2) at (2.5, 2.5) {};
      \node [left=0.0 cm of label2] (num2) {\huge 2};

      % show a straight line
      \draw [thick] (0.5, 2.0) -- (3.0, 2.0);

      % show the operator
      \node (op) at (1, 2.5) {\huge$+$};

      % show the result
      \node (label3) at (2.75, 1) {};
      \node [rounded corners, rectangle, minimum width=1.5 cm, minimum height = 1.25 cm, draw, left=0.0 cm of label3] {};

    \end{tikzpicture}
  \end{center}
\end{minipage}
\begin{minipage}{0.25\linewidth}  
  \begin{center}
    \begin{tikzpicture}

      % draw an invisible box used to properly align all sequences
      \draw [white] (0,0) rectangle (3.5,4.0);

      % show the operands
      \node (label1) at (2.5, 3.5) {};
      \node [left=0.0 cm of label1] (num1) {\huge 13};
      \node (label2) at (2.5, 2.5) {};
      \node [left=0.0 cm of label2] (num2) {\huge 1};

      % show a straight line
      \draw [thick] (0.5, 2.0) -- (3.0, 2.0);

      % show the operator
      \node (op) at (1, 2.5) {\huge$+$};

      % show the result
      \node (label3) at (2.75, 1) {};
      \node [rounded corners, rectangle, minimum width=1.5 cm, minimum height = 1.25 cm, draw, left=0.0 cm of label3] {};

    \end{tikzpicture}
  \end{center}
\end{minipage}
\begin{minipage}{0.25\linewidth}  
  \begin{center}
    \begin{tikzpicture}

      % draw an invisible box used to properly align all sequences
      \draw [white] (0,0) rectangle (3.5,4.0);

      % show the operands
      \node (label1) at (2.5, 3.5) {};
      \node [left=0.0 cm of label1] (num1) {\huge 10};
      \node (label2) at (2.5, 2.5) {};
      \node [left=0.0 cm of label2] (num2) {\huge 9};

      % show a straight line
      \draw [thick] (0.5, 2.0) -- (3.0, 2.0);

      % show the operator
      \node (op) at (1, 2.5) {\huge$+$};

      % show the result
      \node (label3) at (2.75, 1) {};
      \node [rounded corners, rectangle, minimum width=1.5 cm, minimum height = 1.25 cm, draw, left=0.0 cm of label3] {};

    \end{tikzpicture}
  \end{center}
\end{minipage}
\begin{minipage}{0.25\linewidth}  
  \begin{center}
    \begin{tikzpicture}

      % draw an invisible box used to properly align all sequences
      \draw [white] (0,0) rectangle (3.5,4.0);

      % show the operands
      \node (label1) at (2.5, 3.5) {};
      \node [left=0.0 cm of label1] (num1) {\huge 11};
      \node (label2) at (2.5, 2.5) {};
      \node [left=0.0 cm of label2] (num2) {\huge 8};

      % show a straight line
      \draw [thick] (0.5, 2.0) -- (3.0, 2.0);

      % show the operator
      \node (op) at (1, 2.5) {\huge$+$};

      % show the result
      \node (label3) at (2.75, 1) {};
      \node [rounded corners, rectangle, minimum width=1.5 cm, minimum height = 1.25 cm, draw, left=0.0 cm of label3] {};

    \end{tikzpicture}
  \end{center}
\end{minipage}
\begin{minipage}{0.25\linewidth}  
  \begin{center}
    \begin{tikzpicture}

      % draw an invisible box used to properly align all sequences
      \draw [white] (0,0) rectangle (3.5,4.0);

      % show the operands
      \node (label1) at (2.5, 3.5) {};
      \node [left=0.0 cm of label1] (num1) {\huge 10};
      \node (label2) at (2.5, 2.5) {};
      \node [left=0.0 cm of label2] (num2) {\huge 4};

      % show a straight line
      \draw [thick] (0.5, 2.0) -- (3.0, 2.0);

      % show the operator
      \node (op) at (1, 2.5) {\huge$+$};

      % show the result
      \node (label3) at (2.75, 1) {};
      \node [rounded corners, rectangle, minimum width=1.5 cm, minimum height = 1.25 cm, draw, left=0.0 cm of label3] {};

    \end{tikzpicture}
  \end{center}
\end{minipage}
\begin{minipage}{0.25\linewidth}  
  \begin{center}
    \begin{tikzpicture}

      % draw an invisible box used to properly align all sequences
      \draw [white] (0,0) rectangle (3.5,4.0);

      % show the operands
      \node (label1) at (2.5, 3.5) {};
      \node [left=0.0 cm of label1] (num1) {\huge 15};
      \node (label2) at (2.5, 2.5) {};
      \node [left=0.0 cm of label2] (num2) {\huge 1};

      % show a straight line
      \draw [thick] (0.5, 2.0) -- (3.0, 2.0);

      % show the operator
      \node (op) at (1, 2.5) {\huge$+$};

      % show the result
      \node (label3) at (2.75, 1) {};
      \node [rounded corners, rectangle, minimum width=1.5 cm, minimum height = 1.25 cm, draw, left=0.0 cm of label3] {};

    \end{tikzpicture}
  \end{center}
\end{minipage}
\begin{minipage}{0.25\linewidth}  
  \begin{center}
    \begin{tikzpicture}

      % draw an invisible box used to properly align all sequences
      \draw [white] (0,0) rectangle (3.5,4.0);

      % show the operands
      \node (label1) at (2.5, 3.5) {};
      \node [left=0.0 cm of label1] (num1) {\huge 13};
      \node (label2) at (2.5, 2.5) {};
      \node [left=0.0 cm of label2] (num2) {\huge 4};

      % show a straight line
      \draw [thick] (0.5, 2.0) -- (3.0, 2.0);

      % show the operator
      \node (op) at (1, 2.5) {\huge$+$};

      % show the result
      \node (label3) at (2.75, 1) {};
      \node [rounded corners, rectangle, minimum width=1.5 cm, minimum height = 1.25 cm, draw, left=0.0 cm of label3] {};

    \end{tikzpicture}
  \end{center}
\end{minipage}
\begin{minipage}{0.25\linewidth}  
  \begin{center}
    \begin{tikzpicture}

      % draw an invisible box used to properly align all sequences
      \draw [white] (0,0) rectangle (3.5,4.0);

      % show the operands
      \node (label1) at (2.5, 3.5) {};
      \node [left=0.0 cm of label1] (num1) {\huge 10};
      \node (label2) at (2.5, 2.5) {};
      \node [left=0.0 cm of label2] (num2) {\huge 5};

      % show a straight line
      \draw [thick] (0.5, 2.0) -- (3.0, 2.0);

      % show the operator
      \node (op) at (1, 2.5) {\huge$+$};

      % show the result
      \node (label3) at (2.75, 1) {};
      \node [rounded corners, rectangle, minimum width=1.5 cm, minimum height = 1.25 cm, draw, left=0.0 cm of label3] {};

    \end{tikzpicture}
  \end{center}
\end{minipage}


\end{parts}

\question !`Vamos a por las restas! Seguro que te salen todas, pero
deberás prestar atención

\begin{parts}

\part[4] Las siguientes resta son todas entre números pequeños. Acabas
de aprender a hacerlas en el colegio, asi que estas Navidades vamos a
practicar las fáciles

\begin{minipage}{0.25\linewidth}  
  \begin{center}
    \begin{tikzpicture}

      % draw an invisible box used to properly align all sequences
      \draw [white] (0,0) rectangle (3.5,4.0);

      % show the operands
      \node (label1) at (2.5, 3.5) {};
      \node [left=0.0 cm of label1] (num1) {\huge 9};
      \node (label2) at (2.5, 2.5) {};
      \node [left=0.0 cm of label2] (num2) {\huge 8};

      % show a straight line
      \draw [thick] (0.5, 2.0) -- (3.0, 2.0);

      % show the operator
      \node (op) at (1, 2.5) {\huge$-$};

      % show the result
      \node (label3) at (2.75, 1) {};
      \node [rounded corners, rectangle, minimum width=1.5 cm, minimum height = 1.25 cm, draw, left=0.0 cm of label3] {};

    \end{tikzpicture}
  \end{center}
\end{minipage}
\begin{minipage}{0.25\linewidth}  
  \begin{center}
    \begin{tikzpicture}

      % draw an invisible box used to properly align all sequences
      \draw [white] (0,0) rectangle (3.5,4.0);

      % show the operands
      \node (label1) at (2.5, 3.5) {};
      \node [left=0.0 cm of label1] (num1) {\huge 5};
      \node (label2) at (2.5, 2.5) {};
      \node [left=0.0 cm of label2] (num2) {\huge 2};

      % show a straight line
      \draw [thick] (0.5, 2.0) -- (3.0, 2.0);

      % show the operator
      \node (op) at (1, 2.5) {\huge$-$};

      % show the result
      \node (label3) at (2.75, 1) {};
      \node [rounded corners, rectangle, minimum width=1.5 cm, minimum height = 1.25 cm, draw, left=0.0 cm of label3] {};

    \end{tikzpicture}
  \end{center}
\end{minipage}
\begin{minipage}{0.25\linewidth}  
  \begin{center}
    \begin{tikzpicture}

      % draw an invisible box used to properly align all sequences
      \draw [white] (0,0) rectangle (3.5,4.0);

      % show the operands
      \node (label1) at (2.5, 3.5) {};
      \node [left=0.0 cm of label1] (num1) {\huge 2};
      \node (label2) at (2.5, 2.5) {};
      \node [left=0.0 cm of label2] (num2) {\huge 1};

      % show a straight line
      \draw [thick] (0.5, 2.0) -- (3.0, 2.0);

      % show the operator
      \node (op) at (1, 2.5) {\huge$-$};

      % show the result
      \node (label3) at (2.75, 1) {};
      \node [rounded corners, rectangle, minimum width=1.5 cm, minimum height = 1.25 cm, draw, left=0.0 cm of label3] {};

    \end{tikzpicture}
  \end{center}
\end{minipage}
\begin{minipage}{0.25\linewidth}  
  \begin{center}
    \begin{tikzpicture}

      % draw an invisible box used to properly align all sequences
      \draw [white] (0,0) rectangle (3.5,4.0);

      % show the operands
      \node (label1) at (2.5, 3.5) {};
      \node [left=0.0 cm of label1] (num1) {\huge 9};
      \node (label2) at (2.5, 2.5) {};
      \node [left=0.0 cm of label2] (num2) {\huge 8};

      % show a straight line
      \draw [thick] (0.5, 2.0) -- (3.0, 2.0);

      % show the operator
      \node (op) at (1, 2.5) {\huge$-$};

      % show the result
      \node (label3) at (2.75, 1) {};
      \node [rounded corners, rectangle, minimum width=1.5 cm, minimum height = 1.25 cm, draw, left=0.0 cm of label3] {};

    \end{tikzpicture}
  \end{center}
\end{minipage}
\begin{minipage}{0.25\linewidth}  
  \begin{center}
    \begin{tikzpicture}

      % draw an invisible box used to properly align all sequences
      \draw [white] (0,0) rectangle (3.5,4.0);

      % show the operands
      \node (label1) at (2.5, 3.5) {};
      \node [left=0.0 cm of label1] (num1) {\huge 9};
      \node (label2) at (2.5, 2.5) {};
      \node [left=0.0 cm of label2] (num2) {\huge 8};

      % show a straight line
      \draw [thick] (0.5, 2.0) -- (3.0, 2.0);

      % show the operator
      \node (op) at (1, 2.5) {\huge$-$};

      % show the result
      \node (label3) at (2.75, 1) {};
      \node [rounded corners, rectangle, minimum width=1.5 cm, minimum height = 1.25 cm, draw, left=0.0 cm of label3] {};

    \end{tikzpicture}
  \end{center}
\end{minipage}
\begin{minipage}{0.25\linewidth}  
  \begin{center}
    \begin{tikzpicture}

      % draw an invisible box used to properly align all sequences
      \draw [white] (0,0) rectangle (3.5,4.0);

      % show the operands
      \node (label1) at (2.5, 3.5) {};
      \node [left=0.0 cm of label1] (num1) {\huge 8};
      \node (label2) at (2.5, 2.5) {};
      \node [left=0.0 cm of label2] (num2) {\huge 6};

      % show a straight line
      \draw [thick] (0.5, 2.0) -- (3.0, 2.0);

      % show the operator
      \node (op) at (1, 2.5) {\huge$-$};

      % show the result
      \node (label3) at (2.75, 1) {};
      \node [rounded corners, rectangle, minimum width=1.5 cm, minimum height = 1.25 cm, draw, left=0.0 cm of label3] {};

    \end{tikzpicture}
  \end{center}
\end{minipage}
\begin{minipage}{0.25\linewidth}  
  \begin{center}
    \begin{tikzpicture}

      % draw an invisible box used to properly align all sequences
      \draw [white] (0,0) rectangle (3.5,4.0);

      % show the operands
      \node (label1) at (2.5, 3.5) {};
      \node [left=0.0 cm of label1] (num1) {\huge 9};
      \node (label2) at (2.5, 2.5) {};
      \node [left=0.0 cm of label2] (num2) {\huge 2};

      % show a straight line
      \draw [thick] (0.5, 2.0) -- (3.0, 2.0);

      % show the operator
      \node (op) at (1, 2.5) {\huge$-$};

      % show the result
      \node (label3) at (2.75, 1) {};
      \node [rounded corners, rectangle, minimum width=1.5 cm, minimum height = 1.25 cm, draw, left=0.0 cm of label3] {};

    \end{tikzpicture}
  \end{center}
\end{minipage}
\begin{minipage}{0.25\linewidth}  
  \begin{center}
    \begin{tikzpicture}

      % draw an invisible box used to properly align all sequences
      \draw [white] (0,0) rectangle (3.5,4.0);

      % show the operands
      \node (label1) at (2.5, 3.5) {};
      \node [left=0.0 cm of label1] (num1) {\huge 9};
      \node (label2) at (2.5, 2.5) {};
      \node [left=0.0 cm of label2] (num2) {\huge 3};

      % show a straight line
      \draw [thick] (0.5, 2.0) -- (3.0, 2.0);

      % show the operator
      \node (op) at (1, 2.5) {\huge$-$};

      % show the result
      \node (label3) at (2.75, 1) {};
      \node [rounded corners, rectangle, minimum width=1.5 cm, minimum height = 1.25 cm, draw, left=0.0 cm of label3] {};

    \end{tikzpicture}
  \end{center}
\end{minipage}
\begin{minipage}{0.25\linewidth}  
  \begin{center}
    \begin{tikzpicture}

      % draw an invisible box used to properly align all sequences
      \draw [white] (0,0) rectangle (3.5,4.0);

      % show the operands
      \node (label1) at (2.5, 3.5) {};
      \node [left=0.0 cm of label1] (num1) {\huge 5};
      \node (label2) at (2.5, 2.5) {};
      \node [left=0.0 cm of label2] (num2) {\huge 4};

      % show a straight line
      \draw [thick] (0.5, 2.0) -- (3.0, 2.0);

      % show the operator
      \node (op) at (1, 2.5) {\huge$-$};

      % show the result
      \node (label3) at (2.75, 1) {};
      \node [rounded corners, rectangle, minimum width=1.5 cm, minimum height = 1.25 cm, draw, left=0.0 cm of label3] {};

    \end{tikzpicture}
  \end{center}
\end{minipage}
\begin{minipage}{0.25\linewidth}  
  \begin{center}
    \begin{tikzpicture}

      % draw an invisible box used to properly align all sequences
      \draw [white] (0,0) rectangle (3.5,4.0);

      % show the operands
      \node (label1) at (2.5, 3.5) {};
      \node [left=0.0 cm of label1] (num1) {\huge 6};
      \node (label2) at (2.5, 2.5) {};
      \node [left=0.0 cm of label2] (num2) {\huge 3};

      % show a straight line
      \draw [thick] (0.5, 2.0) -- (3.0, 2.0);

      % show the operator
      \node (op) at (1, 2.5) {\huge$-$};

      % show the result
      \node (label3) at (2.75, 1) {};
      \node [rounded corners, rectangle, minimum width=1.5 cm, minimum height = 1.25 cm, draw, left=0.0 cm of label3] {};

    \end{tikzpicture}
  \end{center}
\end{minipage}
\begin{minipage}{0.25\linewidth}  
  \begin{center}
    \begin{tikzpicture}

      % draw an invisible box used to properly align all sequences
      \draw [white] (0,0) rectangle (3.5,4.0);

      % show the operands
      \node (label1) at (2.5, 3.5) {};
      \node [left=0.0 cm of label1] (num1) {\huge 4};
      \node (label2) at (2.5, 2.5) {};
      \node [left=0.0 cm of label2] (num2) {\huge 3};

      % show a straight line
      \draw [thick] (0.5, 2.0) -- (3.0, 2.0);

      % show the operator
      \node (op) at (1, 2.5) {\huge$-$};

      % show the result
      \node (label3) at (2.75, 1) {};
      \node [rounded corners, rectangle, minimum width=1.5 cm, minimum height = 1.25 cm, draw, left=0.0 cm of label3] {};

    \end{tikzpicture}
  \end{center}
\end{minipage}
\begin{minipage}{0.25\linewidth}  
  \begin{center}
    \begin{tikzpicture}

      % draw an invisible box used to properly align all sequences
      \draw [white] (0,0) rectangle (3.5,4.0);

      % show the operands
      \node (label1) at (2.5, 3.5) {};
      \node [left=0.0 cm of label1] (num1) {\huge 9};
      \node (label2) at (2.5, 2.5) {};
      \node [left=0.0 cm of label2] (num2) {\huge 3};

      % show a straight line
      \draw [thick] (0.5, 2.0) -- (3.0, 2.0);

      % show the operator
      \node (op) at (1, 2.5) {\huge$-$};

      % show the result
      \node (label3) at (2.75, 1) {};
      \node [rounded corners, rectangle, minimum width=1.5 cm, minimum height = 1.25 cm, draw, left=0.0 cm of label3] {};

    \end{tikzpicture}
  \end{center}
\end{minipage}


\end{parts}

  \question {\bf }, como sé que te gusta jugar a calcular los números
  anteriores y posteriores a uno que siempre te digo, he hecho estos
  ejercicios para tí. Espero que te gusten mi niña.

  \begin{parts}

    \part[3] Primero, ¿serías capaz de calcular el número anterior a
    cada uno de los siguientes? No es más difícil que cuando jugamos a
    esto paseando por la calle:

\begin{minipage}{0.25\linewidth}  
  \begin{center}
    \begin{tikzpicture}

      % draw an invisible box used to properly align all sequences
      \draw [white] (0,0) rectangle (3,1.5);

      % show the area to draw the answer
      \node [rounded corners, rectangle, minimum width=1.5 cm, minimum height = 1.25 cm, draw] at (0.5, 0.75) {};

      % show the index
      \node at (2, 0.75) {\huge 8};

    \end{tikzpicture}
  \end{center}
\end{minipage}
\begin{minipage}{0.25\linewidth}  
  \begin{center}
    \begin{tikzpicture}

      % draw an invisible box used to properly align all sequences
      \draw [white] (0,0) rectangle (3,1.5);

      % show the area to draw the answer
      \node [rounded corners, rectangle, minimum width=1.5 cm, minimum height = 1.25 cm, draw] at (0.5, 0.75) {};

      % show the index
      \node at (2, 0.75) {\huge 2};

    \end{tikzpicture}
  \end{center}
\end{minipage}
\begin{minipage}{0.25\linewidth}  
  \begin{center}
    \begin{tikzpicture}

      % draw an invisible box used to properly align all sequences
      \draw [white] (0,0) rectangle (3,1.5);

      % show the area to draw the answer
      \node [rounded corners, rectangle, minimum width=1.5 cm, minimum height = 1.25 cm, draw] at (0.5, 0.75) {};

      % show the index
      \node at (2, 0.75) {\huge 1};

    \end{tikzpicture}
  \end{center}
\end{minipage}
\begin{minipage}{0.25\linewidth}  
  \begin{center}
    \begin{tikzpicture}

      % draw an invisible box used to properly align all sequences
      \draw [white] (0,0) rectangle (3,1.5);

      % show the area to draw the answer
      \node [rounded corners, rectangle, minimum width=1.5 cm, minimum height = 1.25 cm, draw] at (0.5, 0.75) {};

      % show the index
      \node at (2, 0.75) {\huge 7};

    \end{tikzpicture}
  \end{center}
\end{minipage}
\begin{minipage}{0.25\linewidth}  
  \begin{center}
    \begin{tikzpicture}

      % draw an invisible box used to properly align all sequences
      \draw [white] (0,0) rectangle (3,1.5);

      % show the area to draw the answer
      \node [rounded corners, rectangle, minimum width=1.5 cm, minimum height = 1.25 cm, draw] at (0.5, 0.75) {};

      % show the index
      \node at (2, 0.75) {\huge 8};

    \end{tikzpicture}
  \end{center}
\end{minipage}
\begin{minipage}{0.25\linewidth}  
  \begin{center}
    \begin{tikzpicture}

      % draw an invisible box used to properly align all sequences
      \draw [white] (0,0) rectangle (3,1.5);

      % show the area to draw the answer
      \node [rounded corners, rectangle, minimum width=1.5 cm, minimum height = 1.25 cm, draw] at (0.5, 0.75) {};

      % show the index
      \node at (2, 0.75) {\huge 5};

    \end{tikzpicture}
  \end{center}
\end{minipage}
\begin{minipage}{0.25\linewidth}  
  \begin{center}
    \begin{tikzpicture}

      % draw an invisible box used to properly align all sequences
      \draw [white] (0,0) rectangle (3,1.5);

      % show the area to draw the answer
      \node [rounded corners, rectangle, minimum width=1.5 cm, minimum height = 1.25 cm, draw] at (0.5, 0.75) {};

      % show the index
      \node at (2, 0.75) {\huge 4};

    \end{tikzpicture}
  \end{center}
\end{minipage}
\begin{minipage}{0.25\linewidth}  
  \begin{center}
    \begin{tikzpicture}

      % draw an invisible box used to properly align all sequences
      \draw [white] (0,0) rectangle (3,1.5);

      % show the area to draw the answer
      \node [rounded corners, rectangle, minimum width=1.5 cm, minimum height = 1.25 cm, draw] at (0.5, 0.75) {};

      % show the index
      \node at (2, 0.75) {\huge 4};

    \end{tikzpicture}
  \end{center}
\end{minipage}
\begin{minipage}{0.25\linewidth}  
  \begin{center}
    \begin{tikzpicture}

      % draw an invisible box used to properly align all sequences
      \draw [white] (0,0) rectangle (3,1.5);

      % show the area to draw the answer
      \node [rounded corners, rectangle, minimum width=1.5 cm, minimum height = 1.25 cm, draw] at (0.5, 0.75) {};

      % show the index
      \node at (2, 0.75) {\huge 5};

    \end{tikzpicture}
  \end{center}
\end{minipage}
\begin{minipage}{0.25\linewidth}  
  \begin{center}
    \begin{tikzpicture}

      % draw an invisible box used to properly align all sequences
      \draw [white] (0,0) rectangle (3,1.5);

      % show the area to draw the answer
      \node [rounded corners, rectangle, minimum width=1.5 cm, minimum height = 1.25 cm, draw] at (0.5, 0.75) {};

      % show the index
      \node at (2, 0.75) {\huge 1};

    \end{tikzpicture}
  \end{center}
\end{minipage}
\begin{minipage}{0.25\linewidth}  
  \begin{center}
    \begin{tikzpicture}

      % draw an invisible box used to properly align all sequences
      \draw [white] (0,0) rectangle (3,1.5);

      % show the area to draw the answer
      \node [rounded corners, rectangle, minimum width=1.5 cm, minimum height = 1.25 cm, draw] at (0.5, 0.75) {};

      % show the index
      \node at (2, 0.75) {\huge 9};

    \end{tikzpicture}
  \end{center}
\end{minipage}
\begin{minipage}{0.25\linewidth}  
  \begin{center}
    \begin{tikzpicture}

      % draw an invisible box used to properly align all sequences
      \draw [white] (0,0) rectangle (3,1.5);

      % show the area to draw the answer
      \node [rounded corners, rectangle, minimum width=1.5 cm, minimum height = 1.25 cm, draw] at (0.5, 0.75) {};

      % show the index
      \node at (2, 0.75) {\huge 5};

    \end{tikzpicture}
  \end{center}
\end{minipage}
\begin{minipage}{0.25\linewidth}  
  \begin{center}
    \begin{tikzpicture}

      % draw an invisible box used to properly align all sequences
      \draw [white] (0,0) rectangle (3,1.5);

      % show the area to draw the answer
      \node [rounded corners, rectangle, minimum width=1.5 cm, minimum height = 1.25 cm, draw] at (0.5, 0.75) {};

      % show the index
      \node at (2, 0.75) {\huge 6};

    \end{tikzpicture}
  \end{center}
\end{minipage}
\begin{minipage}{0.25\linewidth}  
  \begin{center}
    \begin{tikzpicture}

      % draw an invisible box used to properly align all sequences
      \draw [white] (0,0) rectangle (3,1.5);

      % show the area to draw the answer
      \node [rounded corners, rectangle, minimum width=1.5 cm, minimum height = 1.25 cm, draw] at (0.5, 0.75) {};

      % show the index
      \node at (2, 0.75) {\huge 5};

    \end{tikzpicture}
  \end{center}
\end{minipage}
\begin{minipage}{0.25\linewidth}  
  \begin{center}
    \begin{tikzpicture}

      % draw an invisible box used to properly align all sequences
      \draw [white] (0,0) rectangle (3,1.5);

      % show the area to draw the answer
      \node [rounded corners, rectangle, minimum width=1.5 cm, minimum height = 1.25 cm, draw] at (0.5, 0.75) {};

      % show the index
      \node at (2, 0.75) {\huge 7};

    \end{tikzpicture}
  \end{center}
\end{minipage}
\begin{minipage}{0.25\linewidth}  
  \begin{center}
    \begin{tikzpicture}

      % draw an invisible box used to properly align all sequences
      \draw [white] (0,0) rectangle (3,1.5);

      % show the area to draw the answer
      \node [rounded corners, rectangle, minimum width=1.5 cm, minimum height = 1.25 cm, draw] at (0.5, 0.75) {};

      % show the index
      \node at (2, 0.75) {\huge 4};

    \end{tikzpicture}
  \end{center}
\end{minipage}


    \part[3] !`Lo has hecho muy bien {\it }! Y estoy seguro que
    también eres capaz de calcular el número posterior a cada uno de
    los siguientes:

\begin{minipage}{0.25\linewidth}  
  \begin{center}
    \begin{tikzpicture}

      % draw an invisible box used to properly align all sequences
      \draw [white] (0,0) rectangle (3,1.5);

      % show the index
      \node at (0.5, 0.75) {\huge 1};

      % show the area to draw the answer
      \node [rounded corners, rectangle, minimum width=1.5 cm, minimum height = 1.25 cm, draw] at (2, 0.75) {};

    \end{tikzpicture}
  \end{center}
\end{minipage}
\begin{minipage}{0.25\linewidth}  
  \begin{center}
    \begin{tikzpicture}

      % draw an invisible box used to properly align all sequences
      \draw [white] (0,0) rectangle (3,1.5);

      % show the index
      \node at (0.5, 0.75) {\huge 1};

      % show the area to draw the answer
      \node [rounded corners, rectangle, minimum width=1.5 cm, minimum height = 1.25 cm, draw] at (2, 0.75) {};

    \end{tikzpicture}
  \end{center}
\end{minipage}
\begin{minipage}{0.25\linewidth}  
  \begin{center}
    \begin{tikzpicture}

      % draw an invisible box used to properly align all sequences
      \draw [white] (0,0) rectangle (3,1.5);

      % show the index
      \node at (0.5, 0.75) {\huge 7};

      % show the area to draw the answer
      \node [rounded corners, rectangle, minimum width=1.5 cm, minimum height = 1.25 cm, draw] at (2, 0.75) {};

    \end{tikzpicture}
  \end{center}
\end{minipage}
\begin{minipage}{0.25\linewidth}  
  \begin{center}
    \begin{tikzpicture}

      % draw an invisible box used to properly align all sequences
      \draw [white] (0,0) rectangle (3,1.5);

      % show the index
      \node at (0.5, 0.75) {\huge 8};

      % show the area to draw the answer
      \node [rounded corners, rectangle, minimum width=1.5 cm, minimum height = 1.25 cm, draw] at (2, 0.75) {};

    \end{tikzpicture}
  \end{center}
\end{minipage}
\begin{minipage}{0.25\linewidth}  
  \begin{center}
    \begin{tikzpicture}

      % draw an invisible box used to properly align all sequences
      \draw [white] (0,0) rectangle (3,1.5);

      % show the index
      \node at (0.5, 0.75) {\huge 2};

      % show the area to draw the answer
      \node [rounded corners, rectangle, minimum width=1.5 cm, minimum height = 1.25 cm, draw] at (2, 0.75) {};

    \end{tikzpicture}
  \end{center}
\end{minipage}
\begin{minipage}{0.25\linewidth}  
  \begin{center}
    \begin{tikzpicture}

      % draw an invisible box used to properly align all sequences
      \draw [white] (0,0) rectangle (3,1.5);

      % show the index
      \node at (0.5, 0.75) {\huge 3};

      % show the area to draw the answer
      \node [rounded corners, rectangle, minimum width=1.5 cm, minimum height = 1.25 cm, draw] at (2, 0.75) {};

    \end{tikzpicture}
  \end{center}
\end{minipage}
\begin{minipage}{0.25\linewidth}  
  \begin{center}
    \begin{tikzpicture}

      % draw an invisible box used to properly align all sequences
      \draw [white] (0,0) rectangle (3,1.5);

      % show the index
      \node at (0.5, 0.75) {\huge 2};

      % show the area to draw the answer
      \node [rounded corners, rectangle, minimum width=1.5 cm, minimum height = 1.25 cm, draw] at (2, 0.75) {};

    \end{tikzpicture}
  \end{center}
\end{minipage}
\begin{minipage}{0.25\linewidth}  
  \begin{center}
    \begin{tikzpicture}

      % draw an invisible box used to properly align all sequences
      \draw [white] (0,0) rectangle (3,1.5);

      % show the index
      \node at (0.5, 0.75) {\huge 5};

      % show the area to draw the answer
      \node [rounded corners, rectangle, minimum width=1.5 cm, minimum height = 1.25 cm, draw] at (2, 0.75) {};

    \end{tikzpicture}
  \end{center}
\end{minipage}
\begin{minipage}{0.25\linewidth}  
  \begin{center}
    \begin{tikzpicture}

      % draw an invisible box used to properly align all sequences
      \draw [white] (0,0) rectangle (3,1.5);

      % show the index
      \node at (0.5, 0.75) {\huge 5};

      % show the area to draw the answer
      \node [rounded corners, rectangle, minimum width=1.5 cm, minimum height = 1.25 cm, draw] at (2, 0.75) {};

    \end{tikzpicture}
  \end{center}
\end{minipage}
\begin{minipage}{0.25\linewidth}  
  \begin{center}
    \begin{tikzpicture}

      % draw an invisible box used to properly align all sequences
      \draw [white] (0,0) rectangle (3,1.5);

      % show the index
      \node at (0.5, 0.75) {\huge 9};

      % show the area to draw the answer
      \node [rounded corners, rectangle, minimum width=1.5 cm, minimum height = 1.25 cm, draw] at (2, 0.75) {};

    \end{tikzpicture}
  \end{center}
\end{minipage}
\begin{minipage}{0.25\linewidth}  
  \begin{center}
    \begin{tikzpicture}

      % draw an invisible box used to properly align all sequences
      \draw [white] (0,0) rectangle (3,1.5);

      % show the index
      \node at (0.5, 0.75) {\huge 1};

      % show the area to draw the answer
      \node [rounded corners, rectangle, minimum width=1.5 cm, minimum height = 1.25 cm, draw] at (2, 0.75) {};

    \end{tikzpicture}
  \end{center}
\end{minipage}
\begin{minipage}{0.25\linewidth}  
  \begin{center}
    \begin{tikzpicture}

      % draw an invisible box used to properly align all sequences
      \draw [white] (0,0) rectangle (3,1.5);

      % show the index
      \node at (0.5, 0.75) {\huge 6};

      % show the area to draw the answer
      \node [rounded corners, rectangle, minimum width=1.5 cm, minimum height = 1.25 cm, draw] at (2, 0.75) {};

    \end{tikzpicture}
  \end{center}
\end{minipage}
\begin{minipage}{0.25\linewidth}  
  \begin{center}
    \begin{tikzpicture}

      % draw an invisible box used to properly align all sequences
      \draw [white] (0,0) rectangle (3,1.5);

      % show the index
      \node at (0.5, 0.75) {\huge 4};

      % show the area to draw the answer
      \node [rounded corners, rectangle, minimum width=1.5 cm, minimum height = 1.25 cm, draw] at (2, 0.75) {};

    \end{tikzpicture}
  \end{center}
\end{minipage}
\begin{minipage}{0.25\linewidth}  
  \begin{center}
    \begin{tikzpicture}

      % draw an invisible box used to properly align all sequences
      \draw [white] (0,0) rectangle (3,1.5);

      % show the index
      \node at (0.5, 0.75) {\huge 7};

      % show the area to draw the answer
      \node [rounded corners, rectangle, minimum width=1.5 cm, minimum height = 1.25 cm, draw] at (2, 0.75) {};

    \end{tikzpicture}
  \end{center}
\end{minipage}
\begin{minipage}{0.25\linewidth}  
  \begin{center}
    \begin{tikzpicture}

      % draw an invisible box used to properly align all sequences
      \draw [white] (0,0) rectangle (3,1.5);

      % show the index
      \node at (0.5, 0.75) {\huge 8};

      % show the area to draw the answer
      \node [rounded corners, rectangle, minimum width=1.5 cm, minimum height = 1.25 cm, draw] at (2, 0.75) {};

    \end{tikzpicture}
  \end{center}
\end{minipage}
\begin{minipage}{0.25\linewidth}  
  \begin{center}
    \begin{tikzpicture}

      % draw an invisible box used to properly align all sequences
      \draw [white] (0,0) rectangle (3,1.5);

      % show the index
      \node at (0.5, 0.75) {\huge 4};

      % show the area to draw the answer
      \node [rounded corners, rectangle, minimum width=1.5 cm, minimum height = 1.25 cm, draw] at (2, 0.75) {};

    \end{tikzpicture}
  \end{center}
\end{minipage}


    \part[4] Y ahora mucho más difícil. ?`Serías capaz de calcular el
    anterior y el posterior a los siguientes números que son mucho más
    grandes? Estoy seguro que te divertirás intentándolo:

\begin{minipage}{0.25\linewidth}  
  \begin{center}
    \begin{tikzpicture}

      % draw an invisible box used to properly align all sequences
      \draw [white] (0,0) rectangle (3,1.5);

      % show the area to draw the answer
      \node [rounded corners, rectangle, minimum width=1.5 cm, minimum height = 1.25 cm, draw] at (0.5, 0.75) {};

      % show the index
      \node at (2, 0.75) {\huge 9};

    \end{tikzpicture}
  \end{center}
\end{minipage}
\begin{minipage}{0.25\linewidth}  
  \begin{center}
    \begin{tikzpicture}

      % draw an invisible box used to properly align all sequences
      \draw [white] (0,0) rectangle (3,1.5);

      % show the area to draw the answer
      \node [rounded corners, rectangle, minimum width=1.5 cm, minimum height = 1.25 cm, draw] at (0.5, 0.75) {};

      % show the index
      \node at (2, 0.75) {\huge 95};

    \end{tikzpicture}
  \end{center}
\end{minipage}
\begin{minipage}{0.25\linewidth}  
  \begin{center}
    \begin{tikzpicture}

      % draw an invisible box used to properly align all sequences
      \draw [white] (0,0) rectangle (3,1.5);

      % show the area to draw the answer
      \node [rounded corners, rectangle, minimum width=1.5 cm, minimum height = 1.25 cm, draw] at (0.5, 0.75) {};

      % show the index
      \node at (2, 0.75) {\huge 5};

    \end{tikzpicture}
  \end{center}
\end{minipage}
\begin{minipage}{0.25\linewidth}  
  \begin{center}
    \begin{tikzpicture}

      % draw an invisible box used to properly align all sequences
      \draw [white] (0,0) rectangle (3,1.5);

      % show the area to draw the answer
      \node [rounded corners, rectangle, minimum width=1.5 cm, minimum height = 1.25 cm, draw] at (0.5, 0.75) {};

      % show the index
      \node at (2, 0.75) {\huge 79};

    \end{tikzpicture}
  \end{center}
\end{minipage}
\begin{minipage}{0.25\linewidth}  
  \begin{center}
    \begin{tikzpicture}

      % draw an invisible box used to properly align all sequences
      \draw [white] (0,0) rectangle (3,1.5);

      % show the area to draw the answer
      \node [rounded corners, rectangle, minimum width=1.5 cm, minimum height = 1.25 cm, draw] at (0.5, 0.75) {};

      % show the index
      \node at (2, 0.75) {\huge 82};

    \end{tikzpicture}
  \end{center}
\end{minipage}
\begin{minipage}{0.25\linewidth}  
  \begin{center}
    \begin{tikzpicture}

      % draw an invisible box used to properly align all sequences
      \draw [white] (0,0) rectangle (3,1.5);

      % show the area to draw the answer
      \node [rounded corners, rectangle, minimum width=1.5 cm, minimum height = 1.25 cm, draw] at (0.5, 0.75) {};

      % show the index
      \node at (2, 0.75) {\huge 99};

    \end{tikzpicture}
  \end{center}
\end{minipage}
\begin{minipage}{0.25\linewidth}  
  \begin{center}
    \begin{tikzpicture}

      % draw an invisible box used to properly align all sequences
      \draw [white] (0,0) rectangle (3,1.5);

      % show the area to draw the answer
      \node [rounded corners, rectangle, minimum width=1.5 cm, minimum height = 1.25 cm, draw] at (0.5, 0.75) {};

      % show the index
      \node at (2, 0.75) {\huge 46};

    \end{tikzpicture}
  \end{center}
\end{minipage}
\begin{minipage}{0.25\linewidth}  
  \begin{center}
    \begin{tikzpicture}

      % draw an invisible box used to properly align all sequences
      \draw [white] (0,0) rectangle (3,1.5);

      % show the area to draw the answer
      \node [rounded corners, rectangle, minimum width=1.5 cm, minimum height = 1.25 cm, draw] at (0.5, 0.75) {};

      % show the index
      \node at (2, 0.75) {\huge 85};

    \end{tikzpicture}
  \end{center}
\end{minipage}
\begin{minipage}{0.25\linewidth}  
  \begin{center}
    \begin{tikzpicture}

      % draw an invisible box used to properly align all sequences
      \draw [white] (0,0) rectangle (3,1.5);

      % show the area to draw the answer
      \node [rounded corners, rectangle, minimum width=1.5 cm, minimum height = 1.25 cm, draw] at (0.5, 0.75) {};

      % show the index
      \node at (2, 0.75) {\huge 36};

    \end{tikzpicture}
  \end{center}
\end{minipage}
\begin{minipage}{0.25\linewidth}  
  \begin{center}
    \begin{tikzpicture}

      % draw an invisible box used to properly align all sequences
      \draw [white] (0,0) rectangle (3,1.5);

      % show the area to draw the answer
      \node [rounded corners, rectangle, minimum width=1.5 cm, minimum height = 1.25 cm, draw] at (0.5, 0.75) {};

      % show the index
      \node at (2, 0.75) {\huge 93};

    \end{tikzpicture}
  \end{center}
\end{minipage}
\begin{minipage}{0.25\linewidth}  
  \begin{center}
    \begin{tikzpicture}

      % draw an invisible box used to properly align all sequences
      \draw [white] (0,0) rectangle (3,1.5);

      % show the area to draw the answer
      \node [rounded corners, rectangle, minimum width=1.5 cm, minimum height = 1.25 cm, draw] at (0.5, 0.75) {};

      % show the index
      \node at (2, 0.75) {\huge 41};

    \end{tikzpicture}
  \end{center}
\end{minipage}
\begin{minipage}{0.25\linewidth}  
  \begin{center}
    \begin{tikzpicture}

      % draw an invisible box used to properly align all sequences
      \draw [white] (0,0) rectangle (3,1.5);

      % show the area to draw the answer
      \node [rounded corners, rectangle, minimum width=1.5 cm, minimum height = 1.25 cm, draw] at (0.5, 0.75) {};

      % show the index
      \node at (2, 0.75) {\huge 27};

    \end{tikzpicture}
  \end{center}
\end{minipage}

\begin{minipage}{0.25\linewidth}  
  \begin{center}
    \begin{tikzpicture}

      % draw an invisible box used to properly align all sequences
      \draw [white] (0,0) rectangle (3,1.5);

      % show the index
      \node at (0.5, 0.75) {\huge 42};

      % show the area to draw the answer
      \node [rounded corners, rectangle, minimum width=1.5 cm, minimum height = 1.25 cm, draw] at (2, 0.75) {};

    \end{tikzpicture}
  \end{center}
\end{minipage}
\begin{minipage}{0.25\linewidth}  
  \begin{center}
    \begin{tikzpicture}

      % draw an invisible box used to properly align all sequences
      \draw [white] (0,0) rectangle (3,1.5);

      % show the index
      \node at (0.5, 0.75) {\huge 53};

      % show the area to draw the answer
      \node [rounded corners, rectangle, minimum width=1.5 cm, minimum height = 1.25 cm, draw] at (2, 0.75) {};

    \end{tikzpicture}
  \end{center}
\end{minipage}
\begin{minipage}{0.25\linewidth}  
  \begin{center}
    \begin{tikzpicture}

      % draw an invisible box used to properly align all sequences
      \draw [white] (0,0) rectangle (3,1.5);

      % show the index
      \node at (0.5, 0.75) {\huge 23};

      % show the area to draw the answer
      \node [rounded corners, rectangle, minimum width=1.5 cm, minimum height = 1.25 cm, draw] at (2, 0.75) {};

    \end{tikzpicture}
  \end{center}
\end{minipage}
\begin{minipage}{0.25\linewidth}  
  \begin{center}
    \begin{tikzpicture}

      % draw an invisible box used to properly align all sequences
      \draw [white] (0,0) rectangle (3,1.5);

      % show the index
      \node at (0.5, 0.75) {\huge 59};

      % show the area to draw the answer
      \node [rounded corners, rectangle, minimum width=1.5 cm, minimum height = 1.25 cm, draw] at (2, 0.75) {};

    \end{tikzpicture}
  \end{center}
\end{minipage}
\begin{minipage}{0.25\linewidth}  
  \begin{center}
    \begin{tikzpicture}

      % draw an invisible box used to properly align all sequences
      \draw [white] (0,0) rectangle (3,1.5);

      % show the index
      \node at (0.5, 0.75) {\huge 49};

      % show the area to draw the answer
      \node [rounded corners, rectangle, minimum width=1.5 cm, minimum height = 1.25 cm, draw] at (2, 0.75) {};

    \end{tikzpicture}
  \end{center}
\end{minipage}
\begin{minipage}{0.25\linewidth}  
  \begin{center}
    \begin{tikzpicture}

      % draw an invisible box used to properly align all sequences
      \draw [white] (0,0) rectangle (3,1.5);

      % show the index
      \node at (0.5, 0.75) {\huge 39};

      % show the area to draw the answer
      \node [rounded corners, rectangle, minimum width=1.5 cm, minimum height = 1.25 cm, draw] at (2, 0.75) {};

    \end{tikzpicture}
  \end{center}
\end{minipage}
\begin{minipage}{0.25\linewidth}  
  \begin{center}
    \begin{tikzpicture}

      % draw an invisible box used to properly align all sequences
      \draw [white] (0,0) rectangle (3,1.5);

      % show the index
      \node at (0.5, 0.75) {\huge 32};

      % show the area to draw the answer
      \node [rounded corners, rectangle, minimum width=1.5 cm, minimum height = 1.25 cm, draw] at (2, 0.75) {};

    \end{tikzpicture}
  \end{center}
\end{minipage}
\begin{minipage}{0.25\linewidth}  
  \begin{center}
    \begin{tikzpicture}

      % draw an invisible box used to properly align all sequences
      \draw [white] (0,0) rectangle (3,1.5);

      % show the index
      \node at (0.5, 0.75) {\huge 39};

      % show the area to draw the answer
      \node [rounded corners, rectangle, minimum width=1.5 cm, minimum height = 1.25 cm, draw] at (2, 0.75) {};

    \end{tikzpicture}
  \end{center}
\end{minipage}
\begin{minipage}{0.25\linewidth}  
  \begin{center}
    \begin{tikzpicture}

      % draw an invisible box used to properly align all sequences
      \draw [white] (0,0) rectangle (3,1.5);

      % show the index
      \node at (0.5, 0.75) {\huge 64};

      % show the area to draw the answer
      \node [rounded corners, rectangle, minimum width=1.5 cm, minimum height = 1.25 cm, draw] at (2, 0.75) {};

    \end{tikzpicture}
  \end{center}
\end{minipage}
\begin{minipage}{0.25\linewidth}  
  \begin{center}
    \begin{tikzpicture}

      % draw an invisible box used to properly align all sequences
      \draw [white] (0,0) rectangle (3,1.5);

      % show the index
      \node at (0.5, 0.75) {\huge 15};

      % show the area to draw the answer
      \node [rounded corners, rectangle, minimum width=1.5 cm, minimum height = 1.25 cm, draw] at (2, 0.75) {};

    \end{tikzpicture}
  \end{center}
\end{minipage}
\begin{minipage}{0.25\linewidth}  
  \begin{center}
    \begin{tikzpicture}

      % draw an invisible box used to properly align all sequences
      \draw [white] (0,0) rectangle (3,1.5);

      % show the index
      \node at (0.5, 0.75) {\huge 54};

      % show the area to draw the answer
      \node [rounded corners, rectangle, minimum width=1.5 cm, minimum height = 1.25 cm, draw] at (2, 0.75) {};

    \end{tikzpicture}
  \end{center}
\end{minipage}
\begin{minipage}{0.25\linewidth}  
  \begin{center}
    \begin{tikzpicture}

      % draw an invisible box used to properly align all sequences
      \draw [white] (0,0) rectangle (3,1.5);

      % show the index
      \node at (0.5, 0.75) {\huge 84};

      % show the area to draw the answer
      \node [rounded corners, rectangle, minimum width=1.5 cm, minimum height = 1.25 cm, draw] at (2, 0.75) {};

    \end{tikzpicture}
  \end{center}
\end{minipage}


\end{parts}

\end{questions}


\end{document}
