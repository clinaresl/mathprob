

\documentclass[svgnames,addpoints]{exam}



\usepackage[T1]{fontenc}
\usepackage[utf8]{inputenc}
\usepackage[spanish]{babel}

\usepackage{examen}

\usepackage{amsfonts}
\usepackage{amssymb}
\usepackage{mathtools}

\usepackage{pifont}

\usepackage{cancel}
\usepackage{array}

\usepackage{tikz}
\usepackage{pgflibraryarrows}
\usepackage{pgflibrarysnakes}

\usetikzlibrary{matrix,patterns,fadings,positioning}

\usepackage{array}
\usepackage{eurosym}

\usepackage{booktabs}
\usepackage{url}

\usepackage{rotating}



\begin{document}

\titulacion{Grado en Informática}
\asignatura{Heurística y Optimización}

\convocatoria{\today}
\tiempo{4 horas}

\principio



\begin{questions}

  \question {\bf Roberto}, como sé que te gusta jugar a calcular los números
  anteriores y posteriores a uno que siempre te digo, he hecho estos
  ejercicios para tí. Espero que te gusten mi niña.

  \begin{parts}

    \part[3] Primero, ¿serías capaz de calcular el número anterior a
    cada uno de los siguientes? No es más difícil que cuando jugamos a
    esto paseando por la calle:

\begin{minipage}{0.25\linewidth}  
  \begin{center}
    \begin{tikzpicture}

      % show the area to draw the answer
      \node (label2) at (0, 0) {};
      \node [rectangle, minimum width=1.2 cm, minimum height = 1 cm, draw, left = 0.0 cm of label2] {};

      % show the index
      \node (label1) at (1.2, 0) {};
      \node [left = 0.0 cm of label1] (num1) {\huge 3};

    \end{tikzpicture}
  \end{center}
\end{minipage}
\begin{minipage}{0.25\linewidth}  
  \begin{center}
    \begin{tikzpicture}

      % show the area to draw the answer
      \node (label2) at (0, 0) {};
      \node [rectangle, minimum width=1.2 cm, minimum height = 1 cm, draw, left = 0.0 cm of label2] {};

      % show the index
      \node (label1) at (1.2, 0) {};
      \node [left = 0.0 cm of label1] (num1) {\huge 8};

    \end{tikzpicture}
  \end{center}
\end{minipage}
\begin{minipage}{0.25\linewidth}  
  \begin{center}
    \begin{tikzpicture}

      % show the area to draw the answer
      \node (label2) at (0, 0) {};
      \node [rectangle, minimum width=1.2 cm, minimum height = 1 cm, draw, left = 0.0 cm of label2] {};

      % show the index
      \node (label1) at (1.2, 0) {};
      \node [left = 0.0 cm of label1] (num1) {\huge 3};

    \end{tikzpicture}
  \end{center}
\end{minipage}
\begin{minipage}{0.25\linewidth}  
  \begin{center}
    \begin{tikzpicture}

      % show the area to draw the answer
      \node (label2) at (0, 0) {};
      \node [rectangle, minimum width=1.2 cm, minimum height = 1 cm, draw, left = 0.0 cm of label2] {};

      % show the index
      \node (label1) at (1.2, 0) {};
      \node [left = 0.0 cm of label1] (num1) {\huge 8};

    \end{tikzpicture}
  \end{center}
\end{minipage}
\begin{minipage}{0.25\linewidth}  
  \begin{center}
    \begin{tikzpicture}

      % show the area to draw the answer
      \node (label2) at (0, 0) {};
      \node [rectangle, minimum width=1.2 cm, minimum height = 1 cm, draw, left = 0.0 cm of label2] {};

      % show the index
      \node (label1) at (1.2, 0) {};
      \node [left = 0.0 cm of label1] (num1) {\huge 4};

    \end{tikzpicture}
  \end{center}
\end{minipage}
\begin{minipage}{0.25\linewidth}  
  \begin{center}
    \begin{tikzpicture}

      % show the area to draw the answer
      \node (label2) at (0, 0) {};
      \node [rectangle, minimum width=1.2 cm, minimum height = 1 cm, draw, left = 0.0 cm of label2] {};

      % show the index
      \node (label1) at (1.2, 0) {};
      \node [left = 0.0 cm of label1] (num1) {\huge 5};

    \end{tikzpicture}
  \end{center}
\end{minipage}
\begin{minipage}{0.25\linewidth}  
  \begin{center}
    \begin{tikzpicture}

      % show the area to draw the answer
      \node (label2) at (0, 0) {};
      \node [rectangle, minimum width=1.2 cm, minimum height = 1 cm, draw, left = 0.0 cm of label2] {};

      % show the index
      \node (label1) at (1.2, 0) {};
      \node [left = 0.0 cm of label1] (num1) {\huge 5};

    \end{tikzpicture}
  \end{center}
\end{minipage}
\begin{minipage}{0.25\linewidth}  
  \begin{center}
    \begin{tikzpicture}

      % show the area to draw the answer
      \node (label2) at (0, 0) {};
      \node [rectangle, minimum width=1.2 cm, minimum height = 1 cm, draw, left = 0.0 cm of label2] {};

      % show the index
      \node (label1) at (1.2, 0) {};
      \node [left = 0.0 cm of label1] (num1) {\huge 9};

    \end{tikzpicture}
  \end{center}
\end{minipage}
\begin{minipage}{0.25\linewidth}  
  \begin{center}
    \begin{tikzpicture}

      % show the area to draw the answer
      \node (label2) at (0, 0) {};
      \node [rectangle, minimum width=1.2 cm, minimum height = 1 cm, draw, left = 0.0 cm of label2] {};

      % show the index
      \node (label1) at (1.2, 0) {};
      \node [left = 0.0 cm of label1] (num1) {\huge 4};

    \end{tikzpicture}
  \end{center}
\end{minipage}
\begin{minipage}{0.25\linewidth}  
  \begin{center}
    \begin{tikzpicture}

      % show the area to draw the answer
      \node (label2) at (0, 0) {};
      \node [rectangle, minimum width=1.2 cm, minimum height = 1 cm, draw, left = 0.0 cm of label2] {};

      % show the index
      \node (label1) at (1.2, 0) {};
      \node [left = 0.0 cm of label1] (num1) {\huge 2};

    \end{tikzpicture}
  \end{center}
\end{minipage}
\begin{minipage}{0.25\linewidth}  
  \begin{center}
    \begin{tikzpicture}

      % show the area to draw the answer
      \node (label2) at (0, 0) {};
      \node [rectangle, minimum width=1.2 cm, minimum height = 1 cm, draw, left = 0.0 cm of label2] {};

      % show the index
      \node (label1) at (1.2, 0) {};
      \node [left = 0.0 cm of label1] (num1) {\huge 2};

    \end{tikzpicture}
  \end{center}
\end{minipage}
\begin{minipage}{0.25\linewidth}  
  \begin{center}
    \begin{tikzpicture}

      % show the area to draw the answer
      \node (label2) at (0, 0) {};
      \node [rectangle, minimum width=1.2 cm, minimum height = 1 cm, draw, left = 0.0 cm of label2] {};

      % show the index
      \node (label1) at (1.2, 0) {};
      \node [left = 0.0 cm of label1] (num1) {\huge 3};

    \end{tikzpicture}
  \end{center}
\end{minipage}


    \part[3] !`Lo has hecho muy bien {\it Roberto}! Y estoy seguro que
    también eres capaz de calcular el número posterior a cada uno de
    los siguientes:

\begin{minipage}{0.25\linewidth}  
  \begin{center}
    \begin{tikzpicture}

      % show the index
      \node (label1) at (0, 0) {};
      \node [left = 0.0 cm of label1] (num1) {\huge 6};

      % show the area to draw the answer
      \node (label2) at (1.2, 0) {};
      \node [rectangle, minimum width=1.2 cm, minimum height = 1 cm, draw, left = 0.0 cm of label2] {};

    \end{tikzpicture}
  \end{center}
\end{minipage}
\begin{minipage}{0.25\linewidth}  
  \begin{center}
    \begin{tikzpicture}

      % show the index
      \node (label1) at (0, 0) {};
      \node [left = 0.0 cm of label1] (num1) {\huge 7};

      % show the area to draw the answer
      \node (label2) at (1.2, 0) {};
      \node [rectangle, minimum width=1.2 cm, minimum height = 1 cm, draw, left = 0.0 cm of label2] {};

    \end{tikzpicture}
  \end{center}
\end{minipage}
\begin{minipage}{0.25\linewidth}  
  \begin{center}
    \begin{tikzpicture}

      % show the index
      \node (label1) at (0, 0) {};
      \node [left = 0.0 cm of label1] (num1) {\huge 9};

      % show the area to draw the answer
      \node (label2) at (1.2, 0) {};
      \node [rectangle, minimum width=1.2 cm, minimum height = 1 cm, draw, left = 0.0 cm of label2] {};

    \end{tikzpicture}
  \end{center}
\end{minipage}
\begin{minipage}{0.25\linewidth}  
  \begin{center}
    \begin{tikzpicture}

      % show the index
      \node (label1) at (0, 0) {};
      \node [left = 0.0 cm of label1] (num1) {\huge 4};

      % show the area to draw the answer
      \node (label2) at (1.2, 0) {};
      \node [rectangle, minimum width=1.2 cm, minimum height = 1 cm, draw, left = 0.0 cm of label2] {};

    \end{tikzpicture}
  \end{center}
\end{minipage}
\begin{minipage}{0.25\linewidth}  
  \begin{center}
    \begin{tikzpicture}

      % show the index
      \node (label1) at (0, 0) {};
      \node [left = 0.0 cm of label1] (num1) {\huge 3};

      % show the area to draw the answer
      \node (label2) at (1.2, 0) {};
      \node [rectangle, minimum width=1.2 cm, minimum height = 1 cm, draw, left = 0.0 cm of label2] {};

    \end{tikzpicture}
  \end{center}
\end{minipage}
\begin{minipage}{0.25\linewidth}  
  \begin{center}
    \begin{tikzpicture}

      % show the index
      \node (label1) at (0, 0) {};
      \node [left = 0.0 cm of label1] (num1) {\huge 6};

      % show the area to draw the answer
      \node (label2) at (1.2, 0) {};
      \node [rectangle, minimum width=1.2 cm, minimum height = 1 cm, draw, left = 0.0 cm of label2] {};

    \end{tikzpicture}
  \end{center}
\end{minipage}
\begin{minipage}{0.25\linewidth}  
  \begin{center}
    \begin{tikzpicture}

      % show the index
      \node (label1) at (0, 0) {};
      \node [left = 0.0 cm of label1] (num1) {\huge 6};

      % show the area to draw the answer
      \node (label2) at (1.2, 0) {};
      \node [rectangle, minimum width=1.2 cm, minimum height = 1 cm, draw, left = 0.0 cm of label2] {};

    \end{tikzpicture}
  \end{center}
\end{minipage}
\begin{minipage}{0.25\linewidth}  
  \begin{center}
    \begin{tikzpicture}

      % show the index
      \node (label1) at (0, 0) {};
      \node [left = 0.0 cm of label1] (num1) {\huge 7};

      % show the area to draw the answer
      \node (label2) at (1.2, 0) {};
      \node [rectangle, minimum width=1.2 cm, minimum height = 1 cm, draw, left = 0.0 cm of label2] {};

    \end{tikzpicture}
  \end{center}
\end{minipage}
\begin{minipage}{0.25\linewidth}  
  \begin{center}
    \begin{tikzpicture}

      % show the index
      \node (label1) at (0, 0) {};
      \node [left = 0.0 cm of label1] (num1) {\huge 8};

      % show the area to draw the answer
      \node (label2) at (1.2, 0) {};
      \node [rectangle, minimum width=1.2 cm, minimum height = 1 cm, draw, left = 0.0 cm of label2] {};

    \end{tikzpicture}
  \end{center}
\end{minipage}
\begin{minipage}{0.25\linewidth}  
  \begin{center}
    \begin{tikzpicture}

      % show the index
      \node (label1) at (0, 0) {};
      \node [left = 0.0 cm of label1] (num1) {\huge 8};

      % show the area to draw the answer
      \node (label2) at (1.2, 0) {};
      \node [rectangle, minimum width=1.2 cm, minimum height = 1 cm, draw, left = 0.0 cm of label2] {};

    \end{tikzpicture}
  \end{center}
\end{minipage}
\begin{minipage}{0.25\linewidth}  
  \begin{center}
    \begin{tikzpicture}

      % show the index
      \node (label1) at (0, 0) {};
      \node [left = 0.0 cm of label1] (num1) {\huge 6};

      % show the area to draw the answer
      \node (label2) at (1.2, 0) {};
      \node [rectangle, minimum width=1.2 cm, minimum height = 1 cm, draw, left = 0.0 cm of label2] {};

    \end{tikzpicture}
  \end{center}
\end{minipage}
\begin{minipage}{0.25\linewidth}  
  \begin{center}
    \begin{tikzpicture}

      % show the index
      \node (label1) at (0, 0) {};
      \node [left = 0.0 cm of label1] (num1) {\huge 2};

      % show the area to draw the answer
      \node (label2) at (1.2, 0) {};
      \node [rectangle, minimum width=1.2 cm, minimum height = 1 cm, draw, left = 0.0 cm of label2] {};

    \end{tikzpicture}
  \end{center}
\end{minipage}


    \part[4] Y ahora mucho más difícil. ?`Serías capaz de calcular el
    anterior y el posterior a los siguientes números que son mucho más
    grandes? Estoy seguro que te divertirás intentándolo:

\begin{minipage}{0.25\linewidth}  
  \begin{center}
    \begin{tikzpicture}

      % show the area to draw the answer
      \node (label2) at (0, 0) {};
      \node [rectangle, minimum width=1.2 cm, minimum height = 1 cm, draw, left = 0.0 cm of label2] {};

      % show the index
      \node (label1) at (1.2, 0) {};
      \node [left = 0.0 cm of label1] (num1) {\huge 75};

    \end{tikzpicture}
  \end{center}
\end{minipage}
\begin{minipage}{0.25\linewidth}  
  \begin{center}
    \begin{tikzpicture}

      % show the area to draw the answer
      \node (label2) at (0, 0) {};
      \node [rectangle, minimum width=1.2 cm, minimum height = 1 cm, draw, left = 0.0 cm of label2] {};

      % show the index
      \node (label1) at (1.2, 0) {};
      \node [left = 0.0 cm of label1] (num1) {\huge 18};

    \end{tikzpicture}
  \end{center}
\end{minipage}
\begin{minipage}{0.25\linewidth}  
  \begin{center}
    \begin{tikzpicture}

      % show the area to draw the answer
      \node (label2) at (0, 0) {};
      \node [rectangle, minimum width=1.2 cm, minimum height = 1 cm, draw, left = 0.0 cm of label2] {};

      % show the index
      \node (label1) at (1.2, 0) {};
      \node [left = 0.0 cm of label1] (num1) {\huge 36};

    \end{tikzpicture}
  \end{center}
\end{minipage}
\begin{minipage}{0.25\linewidth}  
  \begin{center}
    \begin{tikzpicture}

      % show the area to draw the answer
      \node (label2) at (0, 0) {};
      \node [rectangle, minimum width=1.2 cm, minimum height = 1 cm, draw, left = 0.0 cm of label2] {};

      % show the index
      \node (label1) at (1.2, 0) {};
      \node [left = 0.0 cm of label1] (num1) {\huge 66};

    \end{tikzpicture}
  \end{center}
\end{minipage}
\begin{minipage}{0.25\linewidth}  
  \begin{center}
    \begin{tikzpicture}

      % show the area to draw the answer
      \node (label2) at (0, 0) {};
      \node [rectangle, minimum width=1.2 cm, minimum height = 1 cm, draw, left = 0.0 cm of label2] {};

      % show the index
      \node (label1) at (1.2, 0) {};
      \node [left = 0.0 cm of label1] (num1) {\huge 13};

    \end{tikzpicture}
  \end{center}
\end{minipage}
\begin{minipage}{0.25\linewidth}  
  \begin{center}
    \begin{tikzpicture}

      % show the area to draw the answer
      \node (label2) at (0, 0) {};
      \node [rectangle, minimum width=1.2 cm, minimum height = 1 cm, draw, left = 0.0 cm of label2] {};

      % show the index
      \node (label1) at (1.2, 0) {};
      \node [left = 0.0 cm of label1] (num1) {\huge 44};

    \end{tikzpicture}
  \end{center}
\end{minipage}
\begin{minipage}{0.25\linewidth}  
  \begin{center}
    \begin{tikzpicture}

      % show the area to draw the answer
      \node (label2) at (0, 0) {};
      \node [rectangle, minimum width=1.2 cm, minimum height = 1 cm, draw, left = 0.0 cm of label2] {};

      % show the index
      \node (label1) at (1.2, 0) {};
      \node [left = 0.0 cm of label1] (num1) {\huge 78};

    \end{tikzpicture}
  \end{center}
\end{minipage}
\begin{minipage}{0.25\linewidth}  
  \begin{center}
    \begin{tikzpicture}

      % show the area to draw the answer
      \node (label2) at (0, 0) {};
      \node [rectangle, minimum width=1.2 cm, minimum height = 1 cm, draw, left = 0.0 cm of label2] {};

      % show the index
      \node (label1) at (1.2, 0) {};
      \node [left = 0.0 cm of label1] (num1) {\huge 92};

    \end{tikzpicture}
  \end{center}
\end{minipage}
\begin{minipage}{0.25\linewidth}  
  \begin{center}
    \begin{tikzpicture}

      % show the area to draw the answer
      \node (label2) at (0, 0) {};
      \node [rectangle, minimum width=1.2 cm, minimum height = 1 cm, draw, left = 0.0 cm of label2] {};

      % show the index
      \node (label1) at (1.2, 0) {};
      \node [left = 0.0 cm of label1] (num1) {\huge 73};

    \end{tikzpicture}
  \end{center}
\end{minipage}
\begin{minipage}{0.25\linewidth}  
  \begin{center}
    \begin{tikzpicture}

      % show the area to draw the answer
      \node (label2) at (0, 0) {};
      \node [rectangle, minimum width=1.2 cm, minimum height = 1 cm, draw, left = 0.0 cm of label2] {};

      % show the index
      \node (label1) at (1.2, 0) {};
      \node [left = 0.0 cm of label1] (num1) {\huge 33};

    \end{tikzpicture}
  \end{center}
\end{minipage}
\begin{minipage}{0.25\linewidth}  
  \begin{center}
    \begin{tikzpicture}

      % show the area to draw the answer
      \node (label2) at (0, 0) {};
      \node [rectangle, minimum width=1.2 cm, minimum height = 1 cm, draw, left = 0.0 cm of label2] {};

      % show the index
      \node (label1) at (1.2, 0) {};
      \node [left = 0.0 cm of label1] (num1) {\huge 76};

    \end{tikzpicture}
  \end{center}
\end{minipage}
\begin{minipage}{0.25\linewidth}  
  \begin{center}
    \begin{tikzpicture}

      % show the area to draw the answer
      \node (label2) at (0, 0) {};
      \node [rectangle, minimum width=1.2 cm, minimum height = 1 cm, draw, left = 0.0 cm of label2] {};

      % show the index
      \node (label1) at (1.2, 0) {};
      \node [left = 0.0 cm of label1] (num1) {\huge 15};

    \end{tikzpicture}
  \end{center}
\end{minipage}

\begin{minipage}{0.25\linewidth}  
  \begin{center}
    \begin{tikzpicture}

      % show the index
      \node (label1) at (0, 0) {};
      \node [left = 0.0 cm of label1] (num1) {\huge 90};

      % show the area to draw the answer
      \node (label2) at (1.2, 0) {};
      \node [rectangle, minimum width=1.2 cm, minimum height = 1 cm, draw, left = 0.0 cm of label2] {};

    \end{tikzpicture}
  \end{center}
\end{minipage}
\begin{minipage}{0.25\linewidth}  
  \begin{center}
    \begin{tikzpicture}

      % show the index
      \node (label1) at (0, 0) {};
      \node [left = 0.0 cm of label1] (num1) {\huge 18};

      % show the area to draw the answer
      \node (label2) at (1.2, 0) {};
      \node [rectangle, minimum width=1.2 cm, minimum height = 1 cm, draw, left = 0.0 cm of label2] {};

    \end{tikzpicture}
  \end{center}
\end{minipage}
\begin{minipage}{0.25\linewidth}  
  \begin{center}
    \begin{tikzpicture}

      % show the index
      \node (label1) at (0, 0) {};
      \node [left = 0.0 cm of label1] (num1) {\huge 43};

      % show the area to draw the answer
      \node (label2) at (1.2, 0) {};
      \node [rectangle, minimum width=1.2 cm, minimum height = 1 cm, draw, left = 0.0 cm of label2] {};

    \end{tikzpicture}
  \end{center}
\end{minipage}
\begin{minipage}{0.25\linewidth}  
  \begin{center}
    \begin{tikzpicture}

      % show the index
      \node (label1) at (0, 0) {};
      \node [left = 0.0 cm of label1] (num1) {\huge 13};

      % show the area to draw the answer
      \node (label2) at (1.2, 0) {};
      \node [rectangle, minimum width=1.2 cm, minimum height = 1 cm, draw, left = 0.0 cm of label2] {};

    \end{tikzpicture}
  \end{center}
\end{minipage}
\begin{minipage}{0.25\linewidth}  
  \begin{center}
    \begin{tikzpicture}

      % show the index
      \node (label1) at (0, 0) {};
      \node [left = 0.0 cm of label1] (num1) {\huge 27};

      % show the area to draw the answer
      \node (label2) at (1.2, 0) {};
      \node [rectangle, minimum width=1.2 cm, minimum height = 1 cm, draw, left = 0.0 cm of label2] {};

    \end{tikzpicture}
  \end{center}
\end{minipage}
\begin{minipage}{0.25\linewidth}  
  \begin{center}
    \begin{tikzpicture}

      % show the index
      \node (label1) at (0, 0) {};
      \node [left = 0.0 cm of label1] (num1) {\huge 6};

      % show the area to draw the answer
      \node (label2) at (1.2, 0) {};
      \node [rectangle, minimum width=1.2 cm, minimum height = 1 cm, draw, left = 0.0 cm of label2] {};

    \end{tikzpicture}
  \end{center}
\end{minipage}
\begin{minipage}{0.25\linewidth}  
  \begin{center}
    \begin{tikzpicture}

      % show the index
      \node (label1) at (0, 0) {};
      \node [left = 0.0 cm of label1] (num1) {\huge 88};

      % show the area to draw the answer
      \node (label2) at (1.2, 0) {};
      \node [rectangle, minimum width=1.2 cm, minimum height = 1 cm, draw, left = 0.0 cm of label2] {};

    \end{tikzpicture}
  \end{center}
\end{minipage}
\begin{minipage}{0.25\linewidth}  
  \begin{center}
    \begin{tikzpicture}

      % show the index
      \node (label1) at (0, 0) {};
      \node [left = 0.0 cm of label1] (num1) {\huge 32};

      % show the area to draw the answer
      \node (label2) at (1.2, 0) {};
      \node [rectangle, minimum width=1.2 cm, minimum height = 1 cm, draw, left = 0.0 cm of label2] {};

    \end{tikzpicture}
  \end{center}
\end{minipage}
\begin{minipage}{0.25\linewidth}  
  \begin{center}
    \begin{tikzpicture}

      % show the index
      \node (label1) at (0, 0) {};
      \node [left = 0.0 cm of label1] (num1) {\huge 3};

      % show the area to draw the answer
      \node (label2) at (1.2, 0) {};
      \node [rectangle, minimum width=1.2 cm, minimum height = 1 cm, draw, left = 0.0 cm of label2] {};

    \end{tikzpicture}
  \end{center}
\end{minipage}
\begin{minipage}{0.25\linewidth}  
  \begin{center}
    \begin{tikzpicture}

      % show the index
      \node (label1) at (0, 0) {};
      \node [left = 0.0 cm of label1] (num1) {\huge 94};

      % show the area to draw the answer
      \node (label2) at (1.2, 0) {};
      \node [rectangle, minimum width=1.2 cm, minimum height = 1 cm, draw, left = 0.0 cm of label2] {};

    \end{tikzpicture}
  \end{center}
\end{minipage}
\begin{minipage}{0.25\linewidth}  
  \begin{center}
    \begin{tikzpicture}

      % show the index
      \node (label1) at (0, 0) {};
      \node [left = 0.0 cm of label1] (num1) {\huge 97};

      % show the area to draw the answer
      \node (label2) at (1.2, 0) {};
      \node [rectangle, minimum width=1.2 cm, minimum height = 1 cm, draw, left = 0.0 cm of label2] {};

    \end{tikzpicture}
  \end{center}
\end{minipage}
\begin{minipage}{0.25\linewidth}  
  \begin{center}
    \begin{tikzpicture}

      % show the index
      \node (label1) at (0, 0) {};
      \node [left = 0.0 cm of label1] (num1) {\huge 14};

      % show the area to draw the answer
      \node (label2) at (1.2, 0) {};
      \node [rectangle, minimum width=1.2 cm, minimum height = 1 cm, draw, left = 0.0 cm of label2] {};

    \end{tikzpicture}
  \end{center}
\end{minipage}


\end{parts}

\end{questions}


\end{document}
